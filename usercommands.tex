% this file defines custom commands


\renewcommand{\vec}[1]{\boldsymbol{ \mathsf{#1} } }         % for vectors
\newcommand{\mat}[1]{\boldsymbol{\mathsf{#1}}}           % for matrices
%\newcommand{\tensortwo}[1]{\mathscr{#1}}           % for matrices
\newcommand{\tensorx}[1]  {\mathcal{#1}}           % for higher order tensor
\newcommand{\fspace}[1]{\mathcal{#1}}           % for matrices

\newcommand{\derivtime}[1]{\dot{#1}}           % time derivative
\newcommand{\dderivtime}[1]{\ddot{#1}}         % double time derivative

\newcommand{\dd}{\mathrm{d}}                    % differential d
\newcommand{\pd}{\partial}                      % partial differentiation d


\newcommand{\tsum}{{\textstyle\sum\limits}}     % for small sums in large equations
\newcommand{\pfrac}[2]{\frac{\pd #1}{\pd #2}}   % \pfrac{f(x,y)}{x} for partial derivative of f(x,y) with respect to x
\renewcommand{\dfrac}[2]{\frac{\dd #1}{\dd #2}} % \dfrac{f(x,y)}{x} for total derivative of f(x,y) with respect to x
\newcommand{\grad}{\nabla}                      % for the gradient
\newcommand{\norm}[1]{\| #1 \|}                 % \norm{x} for the norm of x
\newcommand{\abs}[1]{| #1 |}                    % \abs{x} for the absolute value of x
\newcommand{\is}[1]{#1^{(1)}}                    % \abs{x} for slave indication
\newcommand{\im}[1]{#1^{(2)}}                    % \abs{x} for master indication


% suggested text abbreviations - extend at will
\newcommand{\dampmat}{\mathcal{\d C}}  % put around regions that are not to be compiled


%
\newcommand{\insertblankpage}{\mbox{}\thispagestyle{empty}\addtocounter{page}{-1}\newpage}
\newcommand{\BibTeX}{$\mathrm{B{\scriptstyle{IB}} \! T\!_{\displaystyle E} \! X}$}