
\section{Introduction}\label{chapt:introduction}


\subsection{Motivation}\label{sec:motivation}
Thanks to the advent of massively parallel high-performance computers, \ac{CFD} is nowadays capable of accurately predicting the flow characteristics of a large variety of real-world problems. Typically, it involves numerical analysis and algorithms in order to analyze the characteristics of a given input setting.\\
In contrast to a pure scientific point of view, which put the main objective in gaining fundamental insight into the flow phenomenon, an engineer typically sees the computational methods as a mean of designing and improving his product. Therefore he is not only satisfied with an sufficiently accurate approximation of the flow variables around his design objective, one could imagine an aircraft or an automobile, but he seeks to use that information to improve his product.
For trivial cases it might be intuitive to identify opportunities for improvement, in a highly complex system with mutual interactions, however, this is not only time consuming but it might even be not possible at all.\\

The effort thus is to automatically determine the gradient of specific variables of interest with respect to design parameters. More vividly speaking, this could be the change of lift of a given airfoil with respect to its front edge curvature or its twist. In fact there are not restrictions when it comes to the choice of the design parameters. In fact one can also think of purely abstract ones, as we will describe in Section \ref{sec:design_model}. However, the appropriate choice of this variables is the key-point in getting a satisfactory results. Details of this will be discussed in Section~\ref{sec:design_model}.
The numerical methodology of calculating this gradients is known as \ac{SA}.\\

There are different approaches to do \ac{SA}. Most intuitively, one can approximate the sensitivity of a target value with respect to some design parameter by a simple \ac{FD} approach. Figuratively speaking, for the airfoil example above, this would mean, that on could run two(or more) simulations with slightly different edge curvatures and then obtain the sensitivity of the lift with respect to that design decision as the difference of the absolute lift values of both simulation divided by the difference in absolute curvature. This approach is appealing, since it can be done with a standard flow solver. However, to get one sensitivity result, at least two simulations now have to be run. What is more, finite differencing not only introduces an approximation error, the appropriate choice of the stencil size is also difficult. Moreover, two slightly different meshes are required, or at least a mean to perturb the original one according to the design parameter choice.\\
Another approach is automatic differentiation. It is different form numerical differentiation in the sense that its error is only determined by the finite accuracy of the machine, but like finite differencing, automatic differentiation of a function works on a specific input parameter set. Therefore it has to be done repeatedly over and over again.
An elegant alternative that solves both issues it symbolic derivation, which means that the equations of interest are derived by specific variables in a fully general manner, leading to a symbolic formula for the derivatives. Once obtained, this formula can then simply be evaluated by the code. We therefore no longer need multiple function evaluations nor is it necessary to do the derivation process over and over again. The payoff lies in the complexity. On can imagine that in a complex 3D fluid simulation, potentially with viscous, inviscid and turbulence-closure terms, this derivation is cumbersome and involves multiple chain rules.

The main purpose of \ac{SA} lies in optimization. Given the a set of design parameters on can imagine that the methodology above is able to give reliable values for out target function, e.g. airfoil lift, with respect to these parameters, e.g. airfoil thickness at different points. Based on this information it is now possible to slightly modify the airfoil, according to these new results. After that the while process starts again until it finally converges to the optimal configuration.
\subsection{Objective}\label{sec:objective}

The main purpose of this project is the implementation of the \ac{SA} method into the \ac{CFD} code AERO-F~\cite{Aerof}..
Both analytical methods and numerical schemes for the computation of
aeroelastic sensitivities in the presence of turbulent viscous flows will be considered.
The scope of this thesis covers both, the direct~\ref{sec:direct_sa} and adjoint approaches~\ref{sec:adjoint_sa}, within the context of Eulerian, \ac{ALE} and Embedded formulations~\ref{sec:fluid_equations_basic}..
The main effort of this thesis concentrates on the computation of the contributions of the viscous terms to the aforementioned sensitivities, as well as a generalization of the technique to embedded methods.