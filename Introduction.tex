
\section{Introduction}\label{chapt:introduction}


\subsection{Motivation}\label{sec:motivation}
Thanks to the advent of massivly parallel high-perfomance computers, Computational Fluid Dynamics(CFD) is nowadays capable of accurately predicting the flow characterisitics of a large variety of real-world problems. Typically, it involves numerical analysis and algorithms in order to analyze the characteritics of a given input setting.\\
In contrast to a pure scientific point of view, which main objective lies in gaining fundamental insight into the flow phenomenon an engineer typically sees the computational methods as a mean of designing and improving his product. Therefor he is not only statisfied with an sufficently accurate approximation of the flow varaibles around his design objective, one could imagine an aircraft or an automobile, but he seeks to use that information to improve his product.
For trivial cases it might be intuitive to identify opportunities for improvement, in a highly complex system with mutaul interactions however, this is not only time consuming but it might even be not possible at all.\\
The effort thus is to automatically determine the gradient of specific veriables of interest with respect to design paramaters. More vividly speaking, this could be the change of lift of a given airfoil with respect to its front edge curvature or its twist. In fact there are not restrictions when it comes to the chocie of the design parameters. In fact one can also think of purely abstract ones, as we will describe in Section REF. However, the appropriate choice of this variables is the keypoint in getting a statisfactory results. Details of this will be discussed in Section REF.
The numerical methodology of calculating this Gradients is known as Sensitivity Analysis(SA).\\
There are different approaches to do SA. Most intuitvely, one can approximate the sensitivity of a target value with respect to some design parameter by a simple Finite Difference approach. Figuratively speaking, for the airfoil example above, this would mean, that on could run two(or more) simulations with slighlt different edge curvatures and then obtain the sensitivitiv of the lift with respect to that design decision as the difference of the absolute lift values of both simulation divided by the difference in absolute curvature. This approach is appealing, since it can be done with a standard flow solver. However, to get one sensitivity result, at least two simulations now have to be run. What is more, finite differencin not only introduces an approximation error, the appropriate choice of the stencil size is also difficult. MOreover, two slightly different meshes are required, or at least a mean to perturb the original one according to the design parameter choice.\\
Another approace is automatic differentiation. It is different form numerical diffeetiation in the sense that its error is only determined by the finite accuracy of the machine, but like finite differencing, automaic differentiation of a function works on a specific input parameter set. Therefor it has to be done repeatedly over and over again.
An elegant alternative that solves both issues it symbolic derivation, which means that the equations of interest are derived by specific variables in a fully general manner, leading to a symbolic formulat for the derivatives. Once obtained, this formula can then simply be evaluated by the code. We therefor no longer need multiple function evfaluations nor is it neccessary to do the derivation process over and over again. The payoff lies in the complexity. On can imagine that in a complex 3D fluid simulation, potentially with viscous, inviscid and turbulence-closure terms, this derivation is cumbersome and involves multiple chain rules.

The main purpose of SA lies in Optimization. Given the a set of design parameters on can imagine that the methodology above is able to give reliable values for out target function, e.g. airfoil lift, with respect to these parameters, e.g. airfoil thickness at different points. Based on this information it is now possible to slighlty modify the airfoil, according to these new results. After that the whiole process starts again after we finally converge to the optimal configuration of the design parameters with respect to those parameters.
\subsection{Objective}\label{sec:objective}

The main purpose of this project is the implementation of the Sensitivity Analysis method in a CFD code REF.
Both analytical methods and numerical schemes for the computation of
aeroelastic sensitivities in the presence of turbulent viscous flows will be considered.
The scope of this thesis covers both, the direct and adjoint approaches~REF, within the context of Eulerian, Arbitray Lagrangian Eulerian and Embedded formulations~REF.
The main effort of this thesis concentrates on the computation of the contributions of the viscoues terms to the afforementioned sensitivities.