\documentclass[../main.tex]{subfiles}
%\usepackage{import}
\begin{document}
\setlength{\delimitershortfall}{0pt}
\section{About this template}\label{sec:about}



\subsection{Basics}

A generic, aeroelastic optimization problem can be written as
\begin{align}\label{eq:generic_optimization_problem}
&min_s \costfunc(\absvars) \\%TODO subscribt s
&\eqctr(\absvar)=\vec{0},~~~\eqctr \in \mathbb{R}^{\numeqctr}\\
&\neqctr(\absvar)>\vec{0},~~~\neqctr \in \mathbb{R}^{\numneqctr}
\end{align}



The Lagrangian function looks like
\begin{align}\label{eq:lagrangian_of_optimization}
\Lagfunc(\absvars,\lagmultseq,\lagmultsneq)=\costfunc(\absvars))=\T{\lagmultseq} \eqctr(\absvars) + \T{\lagmultsneq}\neqctr(\absvars) \\
\end{align}

The Saddle point formulation of the Lagrangian
\begin{align}\label{eq:saddlepoint_optimization}
\pdfrac{\Lagfunc}{\absvars}&=\pdfrac{\costfunc}{\absvars}=\T{\lagmultseq}\pdfrac{\eqctr}{\absvars}+\T{\lagmultsneq}\pdfrac{\neqctr}{\absvars} \\
\pdfrac{\Lagfunc}{\lagmultseq}&=\eqctr=\vec{0} \\
\pdfrac{\Lagfunc}{\lagmultsneq}&=\T{\lagmultsneq}\neqctr=\vec{0}
\end{align}


\def\PPLagfuncBYabsvars{\ppdfrac{\Lagfunc}{\absvars}}
\def\PLagfuncBYabsvars{\pdfrac{\Lagfunc}{\absvars}}
\def\PneqctrBYabsvars{\pdfrac{\neqctr}{\absvar}}
\def\PeqctrBYabsvars{\pdfrac{\eqctr}{\absvar}}

\def\incrabsvars{\Delta \absvars}
\def\incrlagmultsneq{\Delta \lagmultsneq}
\def\incrlagmultseq{\Delta \lagmultseq}
\begin{align}\label{eq:saddlepoint_newtonform}
\begin{bmatrix}
\it{\PPLagfuncBYabsvars}                & \it{\PneqctrBYabsvars} & \it{\PeqctrBYabsvars} \\
\it{\lagmultsneq}\it{\PneqctrBYabsvars} & \it{\eqctr}            & \vec{0}               \\
\it{\PeqctrBYabsvars}                   & \vec{0}                & \vec{0}
\end{bmatrix}
	\begin{bmatrix}
	\incrabsvars \\
	\incrlagmultsneq \\
	\incrlagmultseq
	\end{bmatrix}\
	=
    -\begin{bmatrix}
		\it{\PLagfuncBYabsvars} \\
		\T{\it{\lagmultsneq}} \it{\neqctr} \\
		\it{\eqctr}
		\end{bmatrix}
\end{align}

\begin{align}\label{eq:quadratic_minimization_problem}
min_{\absvars}( \frac{1}{2} \T{\incrabsvars} \it{\PPLagfuncBYabsvars} \incrabsvars + \frac{\partial \it{\costfunc}}{\partial \absvars} ) \\
\it{\PneqctrBYabsvars} \incrabsvars+\it{\neqctr} \geq \vec{0} \\
\it{\PeqctrBYabsvars} \incrabsvars+\it{\eqctr} = \vec{0}
\end{align}

\begin{align}\label{eq:}
\begin{bmatrix}
\absvars^{(k+1)} \\
\lagmultseq^{(k+1)} \\
\lagmultsneq^{(k+1)}
\end{bmatrix} =
  \begin{bmatrix}
  \it{\absvars} \\
  \it{\lagmultseq} \\
  \it{\lagmultsneq} \\
  \end{bmatrix} +
    \it{\alpha}
    \begin{bmatrix}
    \it{\incrabsvars} \\
    \it{\incrlagmultseq} \\
    \it{\incrlagmultsneq} \\
    \end{bmatrix}
\end{align}



The discrete form ot the steady state can be written as:
\begin{align}\label{eq:3field_basic}
\EOSstruct (\absvars,\structdisp,\mms,\fstate) = \vec{0} \\
\EOSmesh   (\absvars,\structdisp,\mms)         = \vec{0} \\
\EOSfluid  (\absvars,\mms,\fstate)             = \vec{0}
\end{align}

\def\DoptcritJBYabsvarI{\frac{d \optcrit_j}{d \absvar_i}}
\def\PoptcritJBYabsvarI{\pdfrac{\optcrit_j}{\absvar_i}}

\def\PoptcritJBYstructdisp{\pdfrac{\optcrit_j}{\structdisp}}
\def\PoptcritJBYmms       {\pdfrac{\optcrit_j}{\mms}}
\def\PoptcritJBYfstate    {\pdfrac{\optcrit_j}{\fstate}}

\def\DstructdispBYabsvarI{\frac{d \structdisp}{d \absvar_i} }
\def\DmmsBYabsvarI       {\frac{d \mms       }{d \absvar_i} }
\def\DfstateBYabsvarI    {\frac{d \fstate    }{d \absvar_i} }

The derivative of the optimization criterion$\optcrit_j$ with respect to the optimization variable $\absvar_i$ gives:
\begin{align}
\DoptcritJBYabsvarI =  \PoptcritJBYabsvarI    +
\PoptcritJBYstructdisp \DstructdispBYabsvarI  +
\PoptcritJBYmms        \DmmsBYabsvarI  +
\PoptcritJBYfstate     \DfstateBYabsvarI
\end{align}


Differentiation of the governing eqations \eqref{eq:3field_basic} gives:
\def\PEOSstructBYabsvarI{\pdfrac{\EOSstruct} {\absvar_i}}
\def\PEOSmeshBYabsvarI  {\pdfrac{\EOSmesh}  {\absvar_i}}
\def\PEOSfluidBYabsvarI{\pdfrac{\EOSfluid}{\absvar_i}}

\def\PEOSstructBYstructdisp{\pdfrac{\EOSstruct} {\structdisp}}
\def\PEOSstructBYmms{\pdfrac{\EOSstruct}{\mms}}
\def\PEOSstructBYfstate    {\vec{0}}

\def\PEOSmeshBYstructdisp{\pdfrac{\EOSmesh}{\structdisp}}
\def\PEOSmeshBYmms       {\pdfrac{\EOSmesh} {\mms}}
\def\PEOSmeshBYfstate    {\pdfrac{\EOSmesh} {\fstate}}

\def\PEOSfluidBYstructdisp{\vec{0}}
\def\PEOSfluidBYmms       {\pdfrac{\EOSfluid} {\mms}}
\def\PEOSfluidBYfstate    {\pdfrac{\EOSfluid} {\fstate}}

\def\PstructdispBYabsvarI{\pdfrac{\structdisp} {\absvar_i}}
\def\PmmsBYabsvarI       {\pdfrac{\mms}        {\absvar_i}}
\def\PfstateBYabsvarI    {\pdfrac{\fstate}     {\absvar_i}}



\def\jacobian3field{
  \begin{bmatrix}  
  \PEOSstructBYstructdisp & \PEOSstructBYmms & \PEOSstructBYfstate \\[0.5em]
  \PEOSmeshBYstructdisp   & \PEOSmeshBYmms   & \PEOSmeshBYfstate   \\[0.5em]
  \PEOSfluidBYstructdisp  & \PEOSfluidBYmms  & \PEOSfluidBYfstate
  \end{bmatrix}}


\begin{align}\label{eq:govering_eqautions_derivative}
\begin{bmatrix}
\PEOSstructBYabsvarI \\[0.5em]
\PEOSmeshBYabsvarI   \\[0.5em]
\PEOSfluidBYabsvarI
\end{bmatrix} +
  \begin{bmatrix}
  \PEOSstructBYstructdisp & \PEOSstructBYmms & \PEOSstructBYfstate \\[0.5em]
  \PEOSmeshBYstructdisp   & \PEOSmeshBYmms   & \PEOSmeshBYfstate   \\[0.5em]
  \PEOSfluidBYstructdisp  & \PEOSfluidBYmms  & \PEOSfluidBYfstate
  \end{bmatrix}
    \begin{bmatrix}
    \PstructdispBYabsvarI \\[0.5em]
    \PmmsBYabsvarI \\[0.5em]
    \PfstateBYabsvarI
    \end{bmatrix} = \vec{0}
\end{align}


The total derivative of the optimization criterion with respect to the abstract variables vcan be expressed as:
\begin{align}\label{eq:totderiv_optcritBYabsvar}
\DoptcritJBYabsvarI = \PoptcritJBYabsvarI -
\T{\begin{bmatrix}
\PoptcritJBYstructdisp \\[0.5em]
\PoptcritJBYmms        \\[0.5em]
\PoptcritJBYfstate     \\[0.5em]
\end{bmatrix}}
  \inv{\tensor{A}}
  \begin{bmatrix}
  \PEOSstructBYabsvarI \\[0.5em]
  \PEOSmeshBYabsvarI   \\[0.5em]
  \PEOSfluidBYabsvarI
  \end{bmatrix}
\end{align}


\begin{align}\label{eq:firststep_direct}
\begin{bmatrix}
\PstructdispBYabsvarI \\[0.5em]
\PmmsBYabsvarI   \\[0.5em]
\PfstateBYabsvarI
\end{bmatrix}
=
  -\inv{\tensor{A}}
  \begin{bmatrix}
  \PEOSstructBYabsvarI \\[0.5em]
  \PEOSmeshBYabsvarI   \\[0.5em]
  \PEOSfluidBYabsvarI
  \end{bmatrix}
\end{align}


Computation of the adjjoint solutions
\begin{align}\label{eq:firststep_adjoint}
\begin{bmatrix}
\adjoints_{\structdisp} \\
\adjoints_{\mms}        \\
\adjoints_{\fstate}
\end{bmatrix}=
  \tensor{A}^{-T}
  \begin{bmatrix}
  \PoptcritJBYstructdisp \\[0.5em]
  \PoptcritJBYmms        \\[0.5em]
  \PoptcritJBYfstate     \\[0.5em]
  \end{bmatrix}
\end{align}

Then these can be injected into REF
\begin{align}\label{eq:secondstep_adjoint}
\DoptcritJBYabsvarI = \PoptcritJBYabsvarI -
\T{\begin{bmatrix}
\adjoints_{\structdisp} \\
\adjoints_{\mms}        \\
\adjoints_{\fstate}
\end{bmatrix}}_j
  \begin{bmatrix}
  \PEOSstructBYabsvarI \\[0.5em]
  \PEOSmeshBYabsvarI   \\[0.5em]
  \PEOSfluidBYabsvarI
  \end{bmatrix}
\end{align}


The A matrix for the direct approach in ALE formulation looks like
\def\AoneoneALE{\stiffmat}
\def\AonetwoALE{\pdfrac{\sload}{\mms}}
\def\AonethreeALE{\pdfrac{\sload}{\fstate}}

\def\AtwooneALE{\begin{bmatrix}
                \stiffmat_{\Omega \Gamma} \ifaceprojStoF \\
                \ifaceprojStoF
                \end{bmatrix}
               }
\def\AtwotwoALE{
               \begin{bmatrix}
               \stiffmat_{\Omega \Omega} & \vec{0}  \\
               \vec{0}                   & \dmat{I}
               \end{bmatrix}
               }
\def\AtwothreeALE{\vec{0}}

\def\AthreeoneALE{\vec{0}}
\def\AthreetwoALE{\pdfrac{\dmat{F}_2}{\mms}}
\def\AthreethreeALE{\jactwo}                       %TODO find better shortcurt name
\begin{align}\label{eq:Amatrix_ALE}
\tensor{A}=
\begin{bmatrix}
\AoneoneALE    &  \AonetwoALE    &  \AonethreeALE  \\[0.5em]
\AtwooneALE    &  \AtwotwoALE    &  \AtwothreeALE  \\[0.5em]
\AthreeoneALE  &  \AthreetwoALE  &  \AthreethreeALE
\end{bmatrix}
\end{align}



TODO some formulkas are missing here
\\
The underrelaxation of the structural displacement update
\begin{align}\label{eq:underrelax_structdisp}
\tfrac{\its{\structdisp}}{\absvar_i} =
(1-\theta)
\tfrac{\its{\structdisp}}{\absvar_i} +
\theta \tfrac{\fic{\structdisp}}{\absvar_i}
\end{align}

where $\fic{\structdisp}$ is obtained from:
\begin{align}\label{eq:fictious_structdisp}
\stiffmat \tfrac{\fic{\structdisp}}{\absvar_i} =
\pdfrac{TODO}{\absvar_i} + \pdfrac{\its{\ifaceprojStoF}}{\absvar_i} -
\pdfrac{\stiffmat}{\absvar_i} \structdisp
\end{align}


Compute
\begin{equation}\label{eq:TODO}
\tfrac{\its{\structdisp}_T}{\absvar_i} =
\ifaceprojStoF \tfrac{\its{\structdisp}}{\absvar_i}
\end{equation}

The fluid mesh motion is computed by solving
\begin{align}\label{eq:mms_domain}
\fic{\stiffmat}_{\Omega \Omega}  \tfrac{\its{\mms}_{\Omega}}{\absvar_i} =
-\fic{\stiffmat}_{\Omega \Gamma} \tfrac{\its{\mms}_{\Gamma}}{\absvar_i}
\end{align}
with
\begin{align}
\tfrac{\its{\mms}_{\Gamma}}{\absvar_i} =
\tfrac{\its{\mms}_{\Gamma}}{\absvar_i}
\end{align}

The derivatives of the fluid state variables are computed by
\begin{align}
\jactwo \tfrac{\itss{\fstate}}{\absvar_i} =
\pdfrac{\tensor{F}_2}{\absvar_i} - \pdfrac{\tensor{F}_2}{\mms} \its{\tfrac{\mms}{\absvar_i}}
\end{align}


The derivative of the fluid load with respect to the abstarct variuables is computed as
\begin{align}\label{eq:deriv_floadBYabsvar}
\pdfrac{\itss{\fload}}{\absvar_i} =
\pdfrac{\itss{\fload}}{\mms}    \tfrac{\its{\mms}}   {\absvar_i} +
\pdfrac{\itss{\fload}}{\fstate} \tfrac{\its{\fstate}}{\absvar_i}
\end{align}

and compute project it onto the structure via
\begin{align}
\pdfrac{\itss{\sload}}{\absvar_i} = \ifaceprojFtoS \pdfrac{\itss{\fload}}{\absvar_i}
\end{align}

The convergence is monitores via
\begin{align}
\begin{split}
  \norm{ \stiffmat \tfrac{ \itss{\fic{\structdisp}} }{\absvar_i}  -  \pdfrac{TODO}{\absvar_i} - \pdfrac{\itss{\sload}}{\absvar_i}  +  \pdfrac{\stiffmat}{\absvar_i}}_{2} &
  \leq \\
  \tolsa \norm{ \stiffmat \tfrac{ \itn{\fic{\structdisp}}  }{\absvar_i}  -  \pdfrac{TODO}{\absvar_i} - \pdfrac{\itn{\sload}}{\absvar_i}  +  \pdfrac{\stiffmat}{\absvar_i}}_{2} &
\end{split} &
\\
\begin{split}
  \norm{\jactwo \tfrac{\itss{\fstate}}{\absvar_i} + \pdfrac{\tensor{F}_2}{\absvar_i} + \pdfrac{\tensor{F}_2}{\mms} \itss{\tfrac{\mms}{\absvar_i} }   }_{2} &
  \leq \\
  \tolsa \norm{\jactwo \tfrac{\itn{\fstate}}{\absvar_i} + \pdfrac{\tensor{F}_2}{\absvar_i} + \pdfrac{\tensor{F}_2}{\mms} \itn{\tfrac{\mms}{\absvar_i} }   }_{2} &
\end{split} &
\end{align}




\end{document}