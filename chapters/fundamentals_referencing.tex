\documentclass[../main.tex]{subfiles}

\begin{document}

\section{Fundamentals of \LaTeX II: referencing}\label{sec:referencing}

\subsection{Sections, equations, figures, and tables}
\label{sec:refseceq}

Everything that can be referenced should have a label. Otherwise use the
non-referenceable version, e.g. the \verb!align*! environment instead of
\verb!align!. This also means every proper section of your thesis should have a
label. Furthermore, every equation that has a number should be important
enough to be referenced and thus gets a label. Sometimes this can also be a
sentence like ``Equation (4) states the key result of this section.'' to
emphasise its importance. If the equation is not referenced, it is not
important and should not even have a number - use \verb!align*!. Figures and
tables always have a label und should always be referenced at least
once. Some readers don't look at them before they are referenced in the
text. Of course, things that cannot be referenced should
never have a label. 

The label should have a certain structure and start with a specifier
according to the object it references, followed by a colon and the actual name
of the label. An example would be \verb!\label{sec:intro}! for the label of
the first section ``Introduction''. The specifiers are given in table
\ref{tab:labelspec}. Referencing a labelled objects is done by the command
\verb!\ref!, e.g. \verb!\ref{sec:intro}!. Equations should always be
referenced by \verb!\eqref!, which automatically puts the brackets around the
equation number (comes with an ams-package already included in this
template).

\begin{table}[h]
  \centering
  \begin{tabular}{|l|l|}\hline
    \bf object & \bf specifier \\\hline
    (sub-)section & sec\\
    equation & eq\\
    figure & fig\\
    table & tab\\ 
    (appendix) & (app)\\ \hline
  \end{tabular}
  \caption{Object specifiers for labels.}
  \label{tab:labelspec}
\end{table}

\subsection{Citations}
\label{sec:cite}

For citations you have to use \BibTeX. For citing a reference, you use
\verb!\cite{LabelOfRef}! As label, it is suggested to use the surname of the
first author, directly followed by the year, and, if there is more than one
publication of that author in the that year, a lowercase letter in
alphabetical order. For instance: \verb!\cite{hughes1987}! for article
\cite{hughes1987}. Others might prefer other systems, which is fine as long
as it is consistent. If you use only a few references, adding them by hand to
your \BibTeX-file seems the easiest. If there are more references you
might want to create your \BibTeX-file by a reference software like Jabref.

\end{document}