\documentclass[../main.tex]{subfiles}
%\usepackage{algorithm}
%\usepackage{algorithmic}
\everymath{\displaystyle}
\def\arraystretch{2.0}
\begin{document}
\setlength{\delimitershortfall}{0pt}

\def\incrabsvars{\Delta \absvars}
\def\incrlagmultsneq{\Delta \lagmultsneq}
\def\incrlagmultseq{\Delta \lagmultseq}

\def\PPLagfuncBYabsvars{\ppdfrac{\Lagfunc}{\absvars}}
\def\PLagfuncBYabsvars{\pdfrac{\Lagfunc}{\absvars}}
\def\PneqctrBYabsvars{\pdfrac{\neqctr}{\absvar}}
\def\PeqctrBYabsvars{\pdfrac{\eqctr}{\absvar}}

\def\DoptcritJBYabsvarI{\frac{d \optcrit_{j}}{d \absvar_{i}}}
\def\PoptcritJBYabsvarI{\pdfrac{\optcrit_{j}}{\absvar_{i}}}

\def\PoptcritJBYstructdisp{\pdfrac{\optcrit_{j}}{\structdisp}}
\def\PoptcritJBYmms       {\pdfrac{\optcrit_{j}}{\mms}}
\def\PoptcritJBYfstate    {\pdfrac{\optcrit_{j}}{\fstate}}

\def\DoptcritJBYstructdisp{\tfrac{\optcrit_{j}}{\structdisp}}
\def\DoptcritJBYmms       {\tfrac{\optcrit_{j}}{\mms}}
\def\DoptcritJBYfstate    {\tfrac{\optcrit_{j}}{\fstate}}

\def\DstructdispBYabsvarI{\frac{d \structdisp}{d \absvar_{i}} }
\def\DmmsBYabsvarI       {\frac{d \mms       }{d \absvar_{i}} }
\def\DfstateBYabsvarI    {\frac{d \fstate    }{d \absvar_{i}} }

\def\PstructdispBYabsvarI{\pdfrac{d \structdisp}{d \absvar_{i}} }
\def\PmmsBYabsvarI       {\pdfrac{d \mms       }{d \absvar_{i}} }
\def\PfstateBYabsvarI    {\pdfrac{d \fstate    }{d \absvar_{i}} }

\def\PEOSstructBYabsvarI{\pdfrac{\EOSstruct} {\absvar_i}}
\def\PEOSmeshBYabsvarI  {\pdfrac{\EOSmesh}  {\absvar_i}}
\def\PEOSfluidBYabsvarI{\pdfrac{\EOSfluid}{\absvar_i}}

\def\DEOSstructBYabsvarI{\tfrac{\EOSstruct} {\absvar_i}}
\def\DEOSmeshBYabsvarI  {\tfrac{\EOSmesh}  {\absvar_i}}
\def\DEOSfluidBYabsvarI{\tfrac{\EOSfluid}{\absvar_i}}

\def\PEOSstructBYstructdisp{\pdfrac{\EOSstruct} {\structdisp}}
\def\PEOSstructBYmms{\pdfrac{\EOSstruct}{\mms}}
\def\PEOSstructBYfstate    {\pdfrac{\EOSstruct}{\fstate}}

\def\PEOSmeshBYstructdisp{\pdfrac{\EOSmesh}{\structdisp}}
\def\PEOSmeshBYmms       {\pdfrac{\EOSmesh} {\mms}}
\def\PEOSmeshBYfstate    {\vec{0}}

\def\PEOSfluidBYstructdisp{\vec{0}}
\def\PEOSfluidBYmms       {\pdfrac{\EOSfluid} {\mms}}
\def\PEOSfluidBYfstate    {\pdfrac{\EOSfluid} {\fstate}}

\def\PstructdispBYabsvarI{\pdfrac{\structdisp} {\absvar_i}}
\def\PmmsBYabsvarI       {\pdfrac{\mms}        {\absvar_i}}
\def\PfstateBYabsvarI    {\pdfrac{\fstate}     {\absvar_i}}



\section{Fluid \acl{SA}}\label{sec:about}
As explained in section \ref{sec:optimization}, optimization is based on the calculations of gradients, also denoted as \acf{SA}. When it comes to \ac{SA} within the context of \ac{CFD}, one has to distinguish between \ac{SA} of the non-coupled fluid problem and the \ac{SA} in the fully coupled aeroelastic case. The latter case is much more involved, since it requires additional terms from the Structure and mesh motion equations. In the following we will therefore begin with the \ac{SA} of a regular fluid problem. A generalization to a coupled FSI problem will follow.
 \\
As mentioned in Section~\ref{sec:optimization}, we typically deal with an optimization criteria $\optcrit_j$, that is in this case dependent on the fluid state variables $\fstate$ that themselves may depend on abstract variables $\absvar_i$
\begin{align}
\optcrit_j=\optcrit_j(\fstate(\absvar_i))
\end{align}



\begin{align}
\DoptcritJBYabsvarI\bigg\rvert_{\fstate_0}=
\underbrace{\PoptcritJBYabsvarI\bigg\rvert_{\fstate_0}}_{
                                                        \substack{
                                                                 \text{directy derived from} \\
                                                                 \text{the definition of $\optcrit$}
                                                                 } 
                                                        }  +
\underbrace{\PoptcritJBYfstate\bigg\rvert_{\fstate_0}}_ {
                                                        \substack{
                                                                 \text{derived analytically or~~}\\
                                                                 \text{by \ac{FD}}
                                                                 }
                                                        }
\underbrace{\DfstateBYabsvarI\bigg\rvert_{\fstate_0}}_  {
                                                        \substack{
                                                                 \text{derived from dynamic}\\
                                                                 \text{fluid equilibrium}
                                                                 }
                                                        }
\end{align}

\begin{align}
\EOSfluid(\fstate(\absvar_i),\mms(\absvar_i),\absvar_i)=\vec{0}
\end{align}
Therefore, the total derivative must be zero as well:
\begin{align}
\DEOSfluidBYabsvarI=\vec{0}=
\PEOSfluidBYabsvarI+
\PEOSfluidBYfstate\DfstateBYabsvarI+
\PEOSfluidBYmms\DmmsBYabsvarI
\end{align}
Therefore the total derivative of the fluid state with respect to the shape variable, needed in equation REF, can be obtained by solving
\begin{align}
\PEOSfluidBYfstate\DfstateBYabsvarI= -
\PEOSfluidBYabsvarI -
\PEOSfluidBYmms\DmmsBYabsvarI
\end{align}
In this equation, $\PEOSfluidBYabsvarI$ can be computed analytically or by \ac{FD}. The derivative of the mesh motion with respect to the abstract variables is often denoted as "shape gradient". It can be divided into two components:
\begin{itemize}
\item The interface component $\tfrac{\mms_\Gamma}{\absvar_i}$, which is associated with the grid points lying on the fluid boundary
\item The interior component $\tfrac{\mms_\Omega}{\absvar_i}$, which is associated with the grid points located in the interior $\Omega$ of the computational domain.
\end{itemize}
The interface component is determined by the structure. Having obtained this one, the interior component can be computed by solving an auxiliary, fictitious Dirichlet problem:
\begin{align}
\tfrac{\mms_\Omega}{\absvar_i}=-\left[\inv{\fic{\stiffmat}_{\Omega\Omega}} \fic{\stiffmat}_{\Omega\Gamma}\right] \tfrac{\mms_\Gamma}{\absvar_i}
\end{align}
where $\fic{\stiffmat}$ is a pseudo stiffness matrix that can be obtained by a simple spring analogy or similar approaches.
For the later introduced Embedded framework, $\tfrac{\mms}{\absvar_i}$ is the position vector of the embedded discrete surface
 \\
In summary, the derivative of an optimization criteria with respect to an abstract variable can be computed as
\begin{align}
\tfrac{\optcrit_j}{\absvar_i}\bigg\rvert_{\fstate_0} &=
-\tfrac{\optcrit_j}{\fstate}\bigg\rvert_{\fstate_0}
\inv{\left[\pdfrac{\EOSfluid}{\fstate}\right]}
\left(
  \pdfrac{\EOSfluid}{\fstate}\bigg\rvert_{\fstate_0} +
  \begin{bmatrix}
    \alpha\pdfrac{\EOSfluid}{\mms_\Omega}\bigg\rvert_{\fstate_0}~
    \pdfrac{\EOSfluid}{\mms_\Gamma}\bigg\rvert_{\fstate_0}
  \end{bmatrix}
  \begin{bmatrix}
    \alpha \inv{\fic{\stiffmat}_{\Omega\Omega}} \fic{\stiffmat}_{\Omega\Gamma} \\
    \eye
  \end{bmatrix}
  \tfrac{\mms_\Gamma}{\absvar_i}
\right)\\
\alpha&=
\begin{cases}
  1\text{  in \ac{ALE} framework}\\
  0\text{  in Embedded framework}
\end{cases}
\end{align}

\pagebreak


\section{Aero-elastic \acl{SA}}\label{sec:aeroelastic_sa}
The \ac{SA} approach applied in this thesis is based on the work of~\cite{Sobieszczanski1990}, for deriving the \ac{GSE} of coupled systems. As introduced by the authors of\cite{Maute2001}, we utilize the three-field formulation of \cite{Farhat1995}.




The derivative of the optimization criterion~$\optcrit_j$, as introduced in Equation~\eqref{eq:optimization_criteria}, with respect to the optimization variable $\absvar_i$ gives:
\begin{align}
\DoptcritJBYabsvarI &=  \PoptcritJBYabsvarI    +
\PoptcritJBYstructdisp \hleblue{\DstructdispBYabsvarI}  +
\PoptcritJBYmms        \hleblue{\DmmsBYabsvarI}  +
\PoptcritJBYfstate     \hleblue{\DfstateBYabsvarI}
\\
&=
\PoptcritJBYabsvarI+
\T{\begin{bmatrix}
\PoptcritJBYstructdisp \\[0.5em]
\PoptcritJBYmms        \\[0.5em]
\PoptcritJBYfstate     \\[0.5em]
\end{bmatrix}}
\cdot
\begin{bmatrix}
\DstructdispBYabsvarI \\[0.5em]
\DmmsBYabsvarI        \\[0.5em]
\DfstateBYabsvarI     \\[0.5em]
\end{bmatrix}
\end{align}
where the partial derivatives$,\PoptcritJBYstructdisp,\PoptcritJBYmms\text{ and } \PoptcritJBYfstate$ can be directly evaluated within the discretized structure and fluid model through the relation between structural, aerodynamic design and abstract optimization parameters defied in the design model~\ref{sec:design_model}.\\
The cumbersome part are the derivatives $\DstructdispBYabsvarI,\DmmsBYabsvarI\text{~and~}\DfstateBYabsvarI$
To obtain them, the governing equations \eqref{eq:3field_basic}\textbf{TODO write down governing equations} have to be derived:




%Differentiation of the governing equations
\def\PEOSstructBYabsvarI{\pdfrac{\EOSstruct} {\absvar_i}}
\def\PEOSmeshBYabsvarI  {\pdfrac{\EOSmesh}  {\absvar_i}}
\def\PEOSfluidBYabsvarI{\pdfrac{\EOSfluid}{\absvar_i}}

\def\PEOSstructBYstructdisp{\pdfrac{\EOSstruct} {\structdisp}}
\def\PEOSstructBYmms{\pdfrac{\EOSstruct}{\mms}}
\def\PEOSstructBYfstate    {\pdfrac{\EOSstruct}{\fstate}}

\def\PEOSmeshBYstructdisp{\pdfrac{\EOSmesh}{\structdisp}}
\def\PEOSmeshBYmms       {\pdfrac{\EOSmesh} {\mms}}
\def\PEOSmeshBYfstate    {\vec{0}}

\def\PEOSfluidBYstructdisp{\vec{0}}
\def\PEOSfluidBYmms       {\pdfrac{\EOSfluid} {\mms}}
\def\PEOSfluidBYfstate    {\pdfrac{\EOSfluid} {\fstate}}

\def\PstructdispBYabsvarI{\pdfrac{\structdisp} {\absvar_i}}
\def\PmmsBYabsvarI       {\pdfrac{\mms}        {\absvar_i}}
\def\PfstateBYabsvarI    {\pdfrac{\fstate}     {\absvar_i}}

\begin{align}\label{eq:govering_eqautions_derivative}
\def\arraystretch{2.0}
\begin{bmatrix}
\PEOSstructBYabsvarI \\[0.5em]
\PEOSmeshBYabsvarI   \\[0.5em]
\PEOSfluidBYabsvarI
\end{bmatrix} +
  \underbrace{\begin{bmatrix}
  \PEOSstructBYstructdisp & \PEOSstructBYmms & \PEOSstructBYfstate \\[0.5em]
  \PEOSmeshBYstructdisp   & \PEOSmeshBYmms   & \PEOSmeshBYfstate   \\[0.5em]
  \PEOSfluidBYstructdisp  & \PEOSfluidBYmms  & \PEOSfluidBYfstate
  \end{bmatrix}}_{\tensor{A}}
    \hleblue{\begin{bmatrix}
    \PstructdispBYabsvarI \\
    \PmmsBYabsvarI        \\
    \PfstateBYabsvarI     \\
    \end{bmatrix}} = \vec{0}
\end{align}
In this equations $\textstyle{\PEOSstructBYabsvarI}$ and $\textstyle \PEOSfluidBYabsvarI$ can be again directly evaluated using the relation specified in the design model. The matrix of first derivatives~$\tensor{A}$ is from now on denoted as the "Jacobian of the optimization problem".
\\
Combining the previous two equations, it follows that the total derivative of the optimization criterion with respect to the abstract variables can be expressed as:
\begin{align}\label{eq:totderiv_optcritBYabsvar}
\DoptcritJBYabsvarI = \PoptcritJBYabsvarI -
\underbrace{\T{\begin{bmatrix}
\PoptcritJBYstructdisp \\[0.5em]
\PoptcritJBYmms        \\[0.5em]
\PoptcritJBYfstate     \\[0.5em]
\end{bmatrix}}}_{n_{\optcrits}\times n_{eq}}
  \underbrace{\inv{\tensor{A}}}_{n_{eq}\times n_{eq}}
  \underbrace{\begin{bmatrix}
  \PEOSstructBYabsvarI \\[0.5em]
  \PEOSmeshBYabsvarI   \\[0.5em]
  \PEOSfluidBYabsvarI
  \end{bmatrix}}_{n_{eq}\times n_{\absvars}}
\end{align}
Where $n_{eq}$ is the total number of equations(e.g. five fluid state equations for the compressible NSG in 3D, three equations of the mesh motions and another three equations for the structure motion), $n_{q}$ is the number of optimization criteria and $n_s$ is the number of abstract variables.

\subsection{Direct vs. adjoint approach}
Equation~\ref{eq:totderiv_optcritBYabsvar} suggests, that there are two alternatives to compute vector-matrix-vector product above.\\

\paragraph{Direct approach}
Firstly, one could first compute the derivatives of the aeroelastic response for each abstract variable and perform the matrix product with $\tensor{A}$:

\begin{align}\label{eq:direct_approach}
\begin{bmatrix}
\DstructdispBYabsvarI \\[0.5em]
\DmmsBYabsvarI   \\[0.5em]
\DfstateBYabsvarI \\[0.5em]
\end{bmatrix}
&=
  -\inv{\tensor{A}}
  \begin{bmatrix}
  \PEOSstructBYabsvarI \\[0.5em]
  \PEOSmeshBYabsvarI   \\[0.5em]
  \PEOSfluidBYabsvarI
  \end{bmatrix}
\text{~~~and then}\\
\DoptcritJBYabsvarI &= \PoptcritJBYabsvarI -
\T{\begin{bmatrix}
\PoptcritJBYstructdisp \\[0.5em]
\PoptcritJBYmms        \\[0.5em]
\PoptcritJBYfstate     \\[0.5em]
\end{bmatrix}}
  \begin{bmatrix}
  \DstructdispBYabsvarI \\[0.5em]
  \DmmsBYabsvarI   \\[0.5em]
  \DfstateBYabsvarI \\[0.5em]
  \end{bmatrix}
\end{align}
Where the total complexity can be approximated as $\order{n_{eq}^2n_{\absvars}+n_{\optcrits}n_{eq}n_{\absvars}}$



\paragraph{Adjoint approach}
Secondly, one could also first compute the derivatives of the optimization criteria and multiply with the Jacobian before substituting this into Equation~\eqref{eq:totderiv_optcritBYabsvar}:
\begin{align}\label{eq:firststep_adjoint}
\begin{bmatrix}
\adjoints_{\structdisp} \\
\adjoints_{\mms}        \\
\adjoints_{\fstate}
\end{bmatrix}&=
  \tensor{A}^{-T}
  \begin{bmatrix}
  \PoptcritJBYstructdisp \\[0.5em]
  \PoptcritJBYmms        \\[0.5em]
  \PoptcritJBYfstate     \\[0.5em]
  \end{bmatrix}
\\
\DoptcritJBYabsvarI &= \PoptcritJBYabsvarI -
\T{\begin{bmatrix}
\adjoints_{\structdisp} \\
\adjoints_{\mms}        \\
\adjoints_{\fstate}
\end{bmatrix}}_j
  \begin{bmatrix}
  \PEOSstructBYabsvarI \\[0.5em]
  \PEOSmeshBYabsvarI   \\[0.5em]
  \PEOSfluidBYabsvarI
  \end{bmatrix}
\end{align}
Where the total complexity can be approximated as $\order{n_{eq}^2n_{\optcrits}+n_{\optcrits}n_{eq}n_{\absvars}}$\\


If one or the other approach is to be preferred depends in on the optimization setup, particularly the number of optimization criteria and the number of optimization variables. Looking at the orders above, one can conclude that if the number of abstract parameters $n_{\absvars}$ is smaller than the number of optimization criteria, the direct approach is more efficient, otherwise the adjoint approach is to be preferred. Additionally the on can argue, that the relevant term in the orders above is the one with $n_{eq}^2$  since it dominates the sum. Therefore, depending on whether $n_s$ or $n_q$ is bigger, the direct or the adjoint method is to be preferred.


\subsubsection{Direct \acl{SA} for the Euler equations}\label{sec:direct_sa}


The A matrix for the direct approach in ALE formulation looks like
\def\AoneoneALE{\stiffmat}
\def\AonetwoALE{\pdfrac{\sload}{\mms}}
\def\AonethreeALE{\pdfrac{\sload}{\fstate}}

\def\AtwooneALE{\begin{bmatrix}
                \stiffmat_{\Omega \Gamma} \ifaceprojStoF \\
                \ifaceprojStoF
                \end{bmatrix}
               }
\def\AtwotwoALE{
               \begin{bmatrix}
               \stiffmat_{\Omega \Omega} & \vec{0}  \\
               \vec{0}                   & \dmat{I}
               \end{bmatrix}
               }
\def\AtwothreeALE{\vec{0}}

\def\AthreeoneALE{\vec{0}}
\def\AthreetwoALE{\pdfrac{\dmat{F}_2}{\mms_\Omega}}
\def\AthreethreeALE{\jactwo}                       %TODO find better shortcurt name
\begin{align}\label{eq:Amatrix_ALE}
\tensor{A}=
\begin{bmatrix}
  \PEOSstructBYstructdisp & \PEOSstructBYmms & \PEOSstructBYfstate \\[0.5em]
  \PEOSmeshBYstructdisp   & \PEOSmeshBYmms   & \PEOSmeshBYfstate   \\[0.5em]
  \PEOSfluidBYstructdisp  & \PEOSfluidBYmms  & \PEOSfluidBYfstate  \\[0.5em]
\end{bmatrix}=
  \begin{bmatrix}
  \AoneoneALE    &  \AonetwoALE    &  \AonethreeALE  \\[0.5em]
  \AtwooneALE    &  \AtwotwoALE    &  \AtwothreeALE  \\[0.5em]
  \AthreeoneALE  &  \AthreetwoALE  &  \AthreethreeALE
  \end{bmatrix}
\end{align}
Where, $\AthreethreeALE$ is the Jacobian of the second order row flux. It shall be noted that constructing this Jacobian is not a trivial issue and takes up a lot of computational resources, especially for \ac{FV}, as described in \cite{Farhat1995}. Investigation into whether this term can be approximated at first order were carried out in \cite{Maute2001} and \cite{Maute2003}.\\
Furthermore, \cite{Piperno2000} considered replacing the two mesh motion related matrices $\AonetwoALE$ and $\AthreetwoALE$ by a transpirational boundary condition. The consequences of this approach are also investigated in \cite{Maute2001} and \cite{Maute2003}.\\
This thesis, however, does not use any of this simplifications.
~\\
The derivation of the sensitivities, can be achieved in a staggered scheme, very similar to the one, described in \ref{sec:3field}\textbf{TODO}. It consists of five steps.

\paragraph{1) Update the structural displacement sensitivity to a new time step}
BY differentiating equations~\eqref{eq:3field_structure} and \eqref{eq:eos_struct} and applying an under relaxation, we can obtain
\begin{align}\label{eq:underrelax_structdisp}
\tfrac{\its{\structdisp}}{\absvar_i} =
(1-\theta)
\tfrac{\its{\structdisp}}{\absvar_i} +
\theta \tfrac{\fic{\structdisp}}{\absvar_i}
\end{align}

where $\fic{\structdisp}$ is obtained from:
\begin{align}\label{eq:fictious_structdisp}
\stiffmat \tfrac{\fic{\structdisp}}{\absvar_i} =
\pdfrac{TODO}{\absvar_i} + \pdfrac{\its{\ifaceprojStoF}}{\absvar_i} -
\pdfrac{\stiffmat}{\absvar_i} \structdisp
\end{align}


\paragraph{2) Transfer sensitivity of structure motion to the interface}
\begin{equation}\label{eq:interface_projections}
\tfrac{\its{\structdisp}_T}{\absvar_i} =
\ifaceprojStoF \tfrac{\its{\structdisp}}{\absvar_i}
\end{equation}


\paragraph{3) Compute derivative of fluid mesh motion}
The fluid mesh motion is computed by solving the pseudo Dirichlet problem as described in~\cite{Farhat1995}. By design, the fictions stiffness matrix $\fic{\stiffmat}$ does not depend on the abstract optimization variables $\absvars$
\begin{align}\label{eq:mms_domain}
\fic{\stiffmat}_{\Omega \Omega}  \tfrac{\its{\mms}_{\Omega}}{\absvar_i} =
-\fic{\stiffmat}_{\Omega \Gamma} \tfrac{\its{\mms}_{\Gamma}}{\absvar_i}
\end{align}
with
\begin{align}
\tfrac{\its{\mms}_{\Gamma}}{\absvar_i} =
\tfrac{\its{\mms}_{\Gamma}}{\absvar_i}
\end{align}


\paragraph{4) Compute the sensitivity of the fluid state variables}
The derivatives of the fluid state variables are computed by
\begin{align}
\jactwo \tfrac{\itss{\fstate}}{\absvar_i} =
\pdfrac{\tensor{F}_2}{\absvar_i} - \pdfrac{\tensor{F}_2}{\mms} \its{\tfrac{\mms}{\absvar_i}}
\end{align}

\paragraph{5) Compute the sensitivity of the structure load vector}
The derivative of the fluid load with respect to the abstract variables can be computed by the third of Equations~\eqref{eq:direct_approach} with the definition of $\tensor{A}$ as specified in \eqref{eq:Amatrix_ALE}.
\begin{align}\label{eq:deriv_floadBYabsvar}
\pdfrac{\itss{\fload}}{\absvar_i} =
\pdfrac{\itss{\fload}}{\mms}    \tfrac{\its{\mms}}   {\absvar_i} +
\pdfrac{\itss{\fload}}{\fstate} \tfrac{\its{\fstate}}{\absvar_i}
\end{align}

and compute project it onto the structure via
\begin{align}
\pdfrac{\itss{\sload}}{\absvar_i} = \ifaceprojFtoS \pdfrac{\itss{\fload}}{\absvar_i}
\end{align}
~\\
~\\
The convergence of the staggered algorithm can be monitored via
\begin{align}
\begin{split}
  \norm{ \stiffmat \tfrac{ \itss{\fic{\structdisp}} }{\absvar_i}  -  \pdfrac{TODO}{\absvar_i} - \pdfrac{\itss{\sload}}{\absvar_i}  +  \pdfrac{\stiffmat}{\absvar_i}}_{2} &
  \leq \\
  \tolsa \norm{ \stiffmat \tfrac{ \itn{\fic{\structdisp}}  }{\absvar_i}  -  \pdfrac{TODO}{\absvar_i} - \pdfrac{\itn{\sload}}{\absvar_i}  +  \pdfrac{\stiffmat}{\absvar_i}}_{2} &
\end{split} &
\\
\begin{split}
  \norm{\jactwo \tfrac{\itss{\fstate}}{\absvar_i} + \pdfrac{\tensor{F}_2}{\absvar_i} + \pdfrac{\tensor{F}_2}{\mms} \itss{\tfrac{\mms}{\absvar_i} }   }_{2} &
  \leq \\
  \tolsa \norm{\jactwo \tfrac{\itn{\fstate}}{\absvar_i} + \pdfrac{\tensor{F}_2}{\absvar_i} + \pdfrac{\tensor{F}_2}{\mms} \itn{\tfrac{\mms}{\absvar_i} }   }_{2} &
\end{split} &
\end{align}


\subsubsection{Adjoint \acl{SA} for the Euler equations}\label{sec:adjoint_sa}
The adjoint SA follows the same scheme as the direct one


Equation \eqref{eq:firststep_adjoint} can be written as:
\def\AonetwoALEadjoint{\begin{bmatrix}
                \pdfrac{\sload}{\mms_\Omega}
                \pdfrac{\sload}{\mms_\Gamma}
                \end{bmatrix}             }

\def\AthreetwoALEadjoint{\begin{bmatrix}
                  \pdfrac{\dmat{F}_2}{\mms_\Omega}
                  \pdfrac{\dmat{F}_2}{\mms_\Gamma}
                  \end{bmatrix}                   }
                  
\def\PoptcritsBYstructdisp{\pdfrac{\optcrits}{\structdisp}}

\begin{align}\label{eq:adjoint_equation}
\T{\begin{bmatrix}
\AoneoneALE    &  \AonetwoALEadjoint    &  \AonethreeALE  \\[0.5em]
\AtwooneALE    &  \AtwotwoALE    &  \AtwothreeALE  \\[0.5em]
\AthreeoneALE  &  \AthreetwoALEadjoint  &  \AthreethreeALE
\end{bmatrix}}
  \begin{bmatrix}
  \vec{a}_{\structdisp} \\
    \begin{bmatrix}
    \vec{a}_{\mms_\Omega} \\
    \vec{a}_{\mms_\Gamma} \\
    \end{bmatrix}       \\
  \vec{a}_{\fstate}     \\
  \end{bmatrix}
  =
    \begin{bmatrix}
    \PoptcritJBYstructdisp \\
      \begin{bmatrix}
      \pdfrac{\optcrits}{\mms_\Omega} \\
      \pdfrac{\optcrits}{\mms_\Gamma} \\
      \end{bmatrix}        \\
    \PoptcritJBYfstate     \\
    \end{bmatrix}
\end{align}
\textbf{TODO} check if the index j is really required here!
A stated earlier, the matrices $\pdfrac{\optcrits}{\mms}$ and $\pdfrac{\optcrits}{\fstate}$ can be computed analytically. As for $\AthreethreeALE$, we follow the methodology outlined in \cite{Lesoinne2001} for evaluating ans storing it efficiently as the product of flux operators. Again the staggered procedure for solving the adjoint state problem shares the same computational kernels with the partitioned aeroelastic scheme described in~\cite{Farhat1995}\\

\paragraph{1) Update the adjoint structure displacement to the new time step}
\begin{align}
\itss{\structdispad}=(1-\theta)\its{\structdispad}+\theta \itss{\fic{\structdispad}} \\
\end{align}
where $\itss{\fic{\structdispad}}$ is obtained from
\begin{align}
\stiffmat \itss{\fic{\structdispad}} = \PoptcritsBYstructdisp - \stiffmat_{\Omega\Gamma} \ifaceprojStoF \its{{\mmsad}_{\Omega}} + \ifaceprojStoF \its{{\mmsad}_{\Gamma}}
\end{align}\textbf{TODO derive this equation}

\paragraph{2) Compute the adjoint fluid state by solving}
\begin{align}
\T{\jactwo} \itss{\fstatead} = \pdfrac{\mms}{\fstate}+ \pdfrac{\sload}{\T{\fstate}} \itss{\structdispad}
\end{align}\textbf{TODO derive this equation}
Again, $\pdfrac{\optcrits}{\fstate}$ is computed analytically from the relations defined in the design and aeroelastic model.


\paragraph{3) Compute adjoint mesh motion in domain and on the interface}
\def\PoptcritsBYfstate{\pdfrac{\optcrits}{\fstate}}
\begin{align}
\fic{\stiffmat}_{\Omega\Omega} \itss{{\mmsad}_{\Omega}} =
\pdfrac{\optcrits}{\mms_\Omega} -
\pdfrac{\sload}{\mms_\Omega} \itss{\structdispad} -
\pdfrac{\T{\dmat{F}_1}}{\mms_\Omega} \itss{\mmsad}\text{ in }\Omega
\end{align}\textbf{TODO check equations}
And the adjoint mesh motion on the interface is computed as $\itss{{\mmsad}_{\Gamma}}$
\begin{align}
\itss{{\mmsad}_{\Gamma}} =
\pdfrac{\optcrits}{\mms_{Gamma}} +
\pdfrac{\sload}{\mms_\Gamma} \itss{\structdispad}    -
\T{\pdfrac{\dmat{F}_2}{\mms)_\Gamma}}\text{ on }\Gamma
\end{align}
where $\PoptcritsBYfstate$ is computed analytically.
~\\
~\\
The convergence of the staggered adjoint optimization algorithm can be monitored via
\begin{align}
\norm{\vec{R}_{\optcrits}-
\T{\tensor{A}} \itss{\vec{a}} }
\leq
\tolsa\norm{\vec{R}_{\optcrits}}
\end{align}



\end{document}
