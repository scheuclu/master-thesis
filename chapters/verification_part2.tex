\documentclass[../main.tex]{subfiles}
%\usepackage{algorithm}
%\usepackage{algorithmic}


\everymath{\displaystyle}
\def\arraystretch{2.0}
\begin{document}
\setlength{\delimitershortfall}{0pt}




\FloatBarrier


\subsection{Work in progress}


\foreach \eqtype in {Euler, Laminar, RANS}{
	\begin{figure}
	\textbf{Lift and Drag sensitivity: body-fitted \eqtype ~equations}
	  \centering
	  % This file was created by matplotlib2tikz v0.6.7.
\begin{tikzpicture}[scale=0.75]

\definecolor{color0}{rgb}{0.12156862745098,0.466666666666667,0.705882352941177}
\definecolor{color1}{rgb}{1,0.498039215686275,0.0549019607843137}
\definecolor{color2}{rgb}{0.172549019607843,0.627450980392157,0.172549019607843}

\begin{groupplot}[group style={group size=2 by 1}]
\nextgroupplot[
title={${\frac{\partial L_x}{\partial Ma}}_{analytic}$ - $\frac{\partial L_x}{\partial Ma}_{FD}$},
xlabel={$\epsilon$},
xmin=7.07945784384137e-06, xmax=0.0141253754462276,
ymin=1.0997870763893e-05, ymax=0.00410495705563733,
xmode=log,
ymode=log,
tick align=outside,
tick pos=left,
x grid style={lightgray!92.026143790849673!black},
y grid style={lightgray!92.026143790849673!black},
legend entries={{$\alpha_o$=0.0},{$\alpha_o$=3.0},{$\alpha_o$=6.0}},
legend cell align={left},
legend style={at={(0.03,0.97)}, anchor=north west, draw=white!80.0!black}
]
\addplot [semithick, color0, mark=*, mark size=3, mark options={solid}]
table {%
0.01 0.00216151720000113
0.00457 0.000455843200001027
0.00214 0.000106659200000081
0.001 3.39452000002183e-05
0.000457 1.88342000004837e-05
0.000214 1.53542000003171e-05
0.0001 1.45852000006386e-05
4.57e-05 1.44192000011145e-05
2.14e-05 1.43952000009051e-05
1e-05 1.44051999999562e-05
};
\addplot [semithick, color1, mark=*, mark size=3, mark options={solid}]
table {%
0.01 0.00238611229999997
0.00457 0.00051455329999861
0.00214 0.000132214299998878
0.001 5.50442999998069e-05
0.000457 2.11932999985009e-05
0.000214 1.75152999997152e-05
0.0001 1.66812999999877e-05
4.57e-05 1.64942999987261e-05
2.14e-05 1.64562999991347e-05
1e-05 1.644729999839e-05
};
\addplot [semithick, color2, mark=*, mark size=3, mark options={solid}]
table {%
0.01 0.0031361694999994
0.00457 0.000669670500002439
0.00214 0.000163671500001072
0.001 5.34985000015809e-05
0.000457 2.99185000010027e-05
0.000214 2.47125000001347e-05
0.0001 2.36015000005807e-05
4.57e-05 2.33555000015429e-05
2.14e-05 2.33055000009585e-05
1e-05 2.32945000000484e-05
};
\nextgroupplot[
title={${\frac{\partial L_y}{\partial Ma}}_{analytic}$ - $\frac{\partial L_y}{\partial Ma}_{FD}$},
xlabel={$\epsilon$},
xmin=7.07945784384137e-06, xmax=0.0141253754462276,
ymin=1.61645869634857e-06, ymax=0.000946955598502839,
xmode=log,
ymode=log,
tick align=outside,
xtick pos=left,
ytick pos=right,
x grid style={lightgray!92.026143790849673!black},
y grid style={lightgray!92.026143790849673!black},
legend entries={{$\alpha_o$=0.0},{$\alpha_o$=3.0},{$\alpha_o$=6.0}},
legend cell align={left},
legend style={at={(0.03,0.97)}, anchor=north west, draw=white!80.0!black}
]
\addplot [semithick, color0, mark=*, mark size=3, mark options={solid}]
table {%
0.01 0.000604994063999997
0.00457 0.000464485104000045
0.00214 0.000384464194000012
0.001 0.000270318774000022
0.000457 3.06226139999977e-05
0.000214 2.20361400005142e-06
0.0001 2.1657540000275e-06
4.57e-05 2.15959400001742e-06
2.14e-05 2.16478400000275e-06
1e-05 2.17661400003788e-06
};
\addplot [semithick, color1, mark=*, mark size=3, mark options={solid}]
table {%
0.01 5.60427000024788e-05
0.00457 0.000102894299999434
0.00214 8.44722999957526e-05
0.001 0.000108613300000115
0.000457 4.76792999961617e-05
0.000214 4.7981299999833e-05
0.0001 4.86042999980896e-05
4.57e-05 4.86382999937973e-05
2.14e-05 4.86342999934664e-05
1e-05 4.86332999969363e-05
};
\addplot [semithick, color2, mark=*, mark size=3, mark options={solid}]
table {%
0.01 0.00070879739999441
0.00457 0.000206129399998645
0.00214 0.000120153399990386
0.001 0.00010602339999366
0.000457 0.000103380399991693
0.000214 0.000104129400000375
0.0001 0.000103993399989122
4.57e-05 0.000103961400000685
2.14e-05 0.000103956399996719
1e-05 0.000103954399989448
};
\end{groupplot}

\end{tikzpicture}
	  \caption[Validation of the Lift and Drag results for mach-sensitivity: body-fitted \eqtype equations]{The graph shows the relative error between the Lift and Drag Mach-sensitivities obtained by \ac{FD} of two steady states, compared to the mach-sensitivity obtained by analytic direct evaluation. $L_x$ denotes the Drag, $L_y$ denotes the lift. Please note that a \ac{FD} Jacobian has been used here, which of course introduces an approximation error. Furthermore, the finite precision of the obtained steady states is another error source. This explains why the error levels off. For all practical optimization purposes, the obtained accuracy is more than enough.}
	  \label{fig:dLdMa_\eqtype_ale}
	\end{figure}
}

%/home/lukas/Desktop/project/independence/project/thesis/fig/studies/force_convergence/Emb/machsens/ALE_Euler/tikzfiles/Ma0.3/liftdrag.tex

\foreach \eqtype in {Euler,Laminar,RANS}{
	\begin{figure}
	\textbf{Lift and Drag sensitivity: embedded \eqtype ~equations}
	  \centering
	  % This file was created by matplotlib2tikz v0.6.7.
\begin{tikzpicture}[scale=0.75]

\definecolor{color0}{rgb}{0.12156862745098,0.466666666666667,0.705882352941177}
\definecolor{color1}{rgb}{1,0.498039215686275,0.0549019607843137}
\definecolor{color2}{rgb}{0.172549019607843,0.627450980392157,0.172549019607843}

\begin{groupplot}[group style={group size=2 by 1}]
\nextgroupplot[
title={${\frac{\partial L_x}{\partial Ma}}_{analytic}$ - $\frac{\partial L_x}{\partial Ma}_{FD}$},
xlabel={$\epsilon$},
xmin=7.07945784384137e-06, xmax=0.0141253754462276,
ymin=1.0997870763893e-05, ymax=0.00410495705563733,
xmode=log,
ymode=log,
tick align=outside,
tick pos=left,
x grid style={lightgray!92.026143790849673!black},
y grid style={lightgray!92.026143790849673!black},
legend entries={{$\alpha_o$=0.0},{$\alpha_o$=3.0},{$\alpha_o$=6.0}},
legend cell align={left},
legend style={at={(0.03,0.97)}, anchor=north west, draw=white!80.0!black}
]
\addplot [semithick, color0, mark=*, mark size=3, mark options={solid}]
table {%
0.01 0.00216151720000113
0.00457 0.000455843200001027
0.00214 0.000106659200000081
0.001 3.39452000002183e-05
0.000457 1.88342000004837e-05
0.000214 1.53542000003171e-05
0.0001 1.45852000006386e-05
4.57e-05 1.44192000011145e-05
2.14e-05 1.43952000009051e-05
1e-05 1.44051999999562e-05
};
\addplot [semithick, color1, mark=*, mark size=3, mark options={solid}]
table {%
0.01 0.00238611229999997
0.00457 0.00051455329999861
0.00214 0.000132214299998878
0.001 5.50442999998069e-05
0.000457 2.11932999985009e-05
0.000214 1.75152999997152e-05
0.0001 1.66812999999877e-05
4.57e-05 1.64942999987261e-05
2.14e-05 1.64562999991347e-05
1e-05 1.644729999839e-05
};
\addplot [semithick, color2, mark=*, mark size=3, mark options={solid}]
table {%
0.01 0.0031361694999994
0.00457 0.000669670500002439
0.00214 0.000163671500001072
0.001 5.34985000015809e-05
0.000457 2.99185000010027e-05
0.000214 2.47125000001347e-05
0.0001 2.36015000005807e-05
4.57e-05 2.33555000015429e-05
2.14e-05 2.33055000009585e-05
1e-05 2.32945000000484e-05
};
\nextgroupplot[
title={${\frac{\partial L_y}{\partial Ma}}_{analytic}$ - $\frac{\partial L_y}{\partial Ma}_{FD}$},
xlabel={$\epsilon$},
xmin=7.07945784384137e-06, xmax=0.0141253754462276,
ymin=1.61645869634857e-06, ymax=0.000946955598502839,
xmode=log,
ymode=log,
tick align=outside,
xtick pos=left,
ytick pos=right,
x grid style={lightgray!92.026143790849673!black},
y grid style={lightgray!92.026143790849673!black},
legend entries={{$\alpha_o$=0.0},{$\alpha_o$=3.0},{$\alpha_o$=6.0}},
legend cell align={left},
legend style={at={(0.03,0.97)}, anchor=north west, draw=white!80.0!black}
]
\addplot [semithick, color0, mark=*, mark size=3, mark options={solid}]
table {%
0.01 0.000604994063999997
0.00457 0.000464485104000045
0.00214 0.000384464194000012
0.001 0.000270318774000022
0.000457 3.06226139999977e-05
0.000214 2.20361400005142e-06
0.0001 2.1657540000275e-06
4.57e-05 2.15959400001742e-06
2.14e-05 2.16478400000275e-06
1e-05 2.17661400003788e-06
};
\addplot [semithick, color1, mark=*, mark size=3, mark options={solid}]
table {%
0.01 5.60427000024788e-05
0.00457 0.000102894299999434
0.00214 8.44722999957526e-05
0.001 0.000108613300000115
0.000457 4.76792999961617e-05
0.000214 4.7981299999833e-05
0.0001 4.86042999980896e-05
4.57e-05 4.86382999937973e-05
2.14e-05 4.86342999934664e-05
1e-05 4.86332999969363e-05
};
\addplot [semithick, color2, mark=*, mark size=3, mark options={solid}]
table {%
0.01 0.00070879739999441
0.00457 0.000206129399998645
0.00214 0.000120153399990386
0.001 0.00010602339999366
0.000457 0.000103380399991693
0.000214 0.000104129400000375
0.0001 0.000103993399989122
4.57e-05 0.000103961400000685
2.14e-05 0.000103956399996719
1e-05 0.000103954399989448
};
\end{groupplot}

\end{tikzpicture}
	  \caption[Validation of the Lift and Drag results for mach-sensitivity: body-fitted \eqtype equations]{The graph shows the relative error between the Lift and Drag Mach-sensitivities obtained by \ac{FD} of two steady states, compared to the mach-sensitivity obtained by analytic direct evaluation. $L_x$ denotes the Drag, $L_y$ denotes the lift. Please note that a \ac{FD} Jacobian has been used here, which of course introduces an approximation error. Furthermore, the finite precision of the obtained steady states is another error source. This explains why the error levels off. For all practical optimization purposes, the obtained accuracy is more than enough.}
	  \label{fig:dLdMa_\eqtype_emb}
	\end{figure}
}



%\foreach \eqtype in {Laminar}{
%	\begin{figure}.
%	  \textbf{Validation of the Lift and Drag results for mach-sensitivity: body-fitted \eqtype  equations}
%	  \centering
%	  % This file was created by matplotlib2tikz v0.6.7.
\begin{tikzpicture}[scale=0.75]

\definecolor{color0}{rgb}{0.12156862745098,0.466666666666667,0.705882352941177}
\definecolor{color1}{rgb}{1,0.498039215686275,0.0549019607843137}
\definecolor{color2}{rgb}{0.172549019607843,0.627450980392157,0.172549019607843}

\begin{groupplot}[group style={group size=2 by 1}]
\nextgroupplot[
title={${\frac{\partial L_x}{\partial Ma}}_{analytic}$ - $\frac{\partial L_x}{\partial Ma}_{FD}$},
xlabel={$\epsilon$},
xmin=7.07945784384137e-06, xmax=0.0141253754462276,
ymin=1.0997870763893e-05, ymax=0.00410495705563733,
xmode=log,
ymode=log,
tick align=outside,
tick pos=left,
x grid style={lightgray!92.026143790849673!black},
y grid style={lightgray!92.026143790849673!black},
legend entries={{$\alpha_o$=0.0},{$\alpha_o$=3.0},{$\alpha_o$=6.0}},
legend cell align={left},
legend style={at={(0.03,0.97)}, anchor=north west, draw=white!80.0!black}
]
\addplot [semithick, color0, mark=*, mark size=3, mark options={solid}]
table {%
0.01 0.00216151720000113
0.00457 0.000455843200001027
0.00214 0.000106659200000081
0.001 3.39452000002183e-05
0.000457 1.88342000004837e-05
0.000214 1.53542000003171e-05
0.0001 1.45852000006386e-05
4.57e-05 1.44192000011145e-05
2.14e-05 1.43952000009051e-05
1e-05 1.44051999999562e-05
};
\addplot [semithick, color1, mark=*, mark size=3, mark options={solid}]
table {%
0.01 0.00238611229999997
0.00457 0.00051455329999861
0.00214 0.000132214299998878
0.001 5.50442999998069e-05
0.000457 2.11932999985009e-05
0.000214 1.75152999997152e-05
0.0001 1.66812999999877e-05
4.57e-05 1.64942999987261e-05
2.14e-05 1.64562999991347e-05
1e-05 1.644729999839e-05
};
\addplot [semithick, color2, mark=*, mark size=3, mark options={solid}]
table {%
0.01 0.0031361694999994
0.00457 0.000669670500002439
0.00214 0.000163671500001072
0.001 5.34985000015809e-05
0.000457 2.99185000010027e-05
0.000214 2.47125000001347e-05
0.0001 2.36015000005807e-05
4.57e-05 2.33555000015429e-05
2.14e-05 2.33055000009585e-05
1e-05 2.32945000000484e-05
};
\nextgroupplot[
title={${\frac{\partial L_y}{\partial Ma}}_{analytic}$ - $\frac{\partial L_y}{\partial Ma}_{FD}$},
xlabel={$\epsilon$},
xmin=7.07945784384137e-06, xmax=0.0141253754462276,
ymin=1.61645869634857e-06, ymax=0.000946955598502839,
xmode=log,
ymode=log,
tick align=outside,
xtick pos=left,
ytick pos=right,
x grid style={lightgray!92.026143790849673!black},
y grid style={lightgray!92.026143790849673!black},
legend entries={{$\alpha_o$=0.0},{$\alpha_o$=3.0},{$\alpha_o$=6.0}},
legend cell align={left},
legend style={at={(0.03,0.97)}, anchor=north west, draw=white!80.0!black}
]
\addplot [semithick, color0, mark=*, mark size=3, mark options={solid}]
table {%
0.01 0.000604994063999997
0.00457 0.000464485104000045
0.00214 0.000384464194000012
0.001 0.000270318774000022
0.000457 3.06226139999977e-05
0.000214 2.20361400005142e-06
0.0001 2.1657540000275e-06
4.57e-05 2.15959400001742e-06
2.14e-05 2.16478400000275e-06
1e-05 2.17661400003788e-06
};
\addplot [semithick, color1, mark=*, mark size=3, mark options={solid}]
table {%
0.01 5.60427000024788e-05
0.00457 0.000102894299999434
0.00214 8.44722999957526e-05
0.001 0.000108613300000115
0.000457 4.76792999961617e-05
0.000214 4.7981299999833e-05
0.0001 4.86042999980896e-05
4.57e-05 4.86382999937973e-05
2.14e-05 4.86342999934664e-05
1e-05 4.86332999969363e-05
};
\addplot [semithick, color2, mark=*, mark size=3, mark options={solid}]
table {%
0.01 0.00070879739999441
0.00457 0.000206129399998645
0.00214 0.000120153399990386
0.001 0.00010602339999366
0.000457 0.000103380399991693
0.000214 0.000104129400000375
0.0001 0.000103993399989122
4.57e-05 0.000103961400000685
2.14e-05 0.000103956399996719
1e-05 0.000103954399989448
};
\end{groupplot}

\end{tikzpicture}
%	  \caption[Validation of the Lift and Drag results for mach-sensitivity: body-fitted \eqtype equations]{The graph shows the relative error between the Lift and Drag Mach-sensitivities obtained by \ac{FD} of two steady states, compared to the mach-sensitivity obtained by analytic direct evaluation. $L_x$ denotes the Drag, $L_y$ denotes the lift. Please note that a \ac{FD} Jacobian has been used here, which of course introduces an approximation error. Furthermore, the finite precision of the obtained steady states is another error source. This explains why the error levels off. For all practical optimization purposes, the obtained accuracy is more than enough.}
%	  \label{fig:dLdMA_laminar_ale}
%	\end{figure}
%}

%

%---------------------------------------------------------------------------------------------------------------------%
% Force convergence for shape sensitivity                                                                             %
%---------------------------------------------------------------------------------------------------------------------%
%\foreach \eqtype in {Euler,Laminar,RANS}{
%	\begin{figure}
%	  \centering
%	  \textbf{Validation of the Lift and Drag results for shape-sensitivity: body-fitted \eqtype equations}\par\medskip  
%	  % This file was created by matplotlib2tikz v0.6.7.
\begin{tikzpicture}

\definecolor{color1}{rgb}{0.301960784313725,0.309803921568627,0.325490196078431}
\definecolor{color0}{rgb}{0.549019607843137,0.0823529411764706,0.0823529411764706}
\definecolor{color2}{rgb}{0.701960784313725,0.6,0.364705882352941}

\begin{groupplot}[group style={group size=2 by 1}]
\nextgroupplot[
title={${\frac{\partial F_x}{\partial Ma}}_{analytic}$ - $\frac{\partial F_x}{\partial Ma}_{FD}$},
xlabel={$\epsilon$},
xmin=7.07945784384137e-06, xmax=0.0141253754462276,
ymin=2.12330574687735e-05, ymax=0.00761369436377756,
xmode=log,
ymode=log,
tick align=outside,
tick pos=left,
x grid style={white!69.019607843137251!black},
y grid style={white!69.019607843137251!black},
legend style={at={(0.03,0.97)}, anchor=north west, draw=white!80.0!black},
legend entries={{$\alpha_o$=0.0},{$\alpha_o$=3.0},{$\alpha_o$=6.0}},
legend cell align={left}
]
\addplot [semithick, color0, mark=*, mark size=3, mark options={solid}]
table {%
0.01 0.0042630664999983
0.00457 0.000814685499999968
0.00214 2.77414999985126e-05
0.001 0.000137136500001134
0.000457 0.000167667500001301
0.000214 0.000175603500000676
0.0001 0.000177345499999149
4.57e-05 0.000177731499999112
2.14e-05 0.000177811500002178
1e-05 0.00017782749999995
};
\addplot [semithick, color1, mark=*, mark size=3, mark options={solid}]
table {%
0.01 0.00512230860000074
0.00457 0.000954174600000357
0.00214 8.75556000003996e-05
0.001 0.000103155399997945
0.000457 0.000145890399998905
0.000214 0.000156877399998478
0.0001 0.000158766399998456
4.57e-05 0.000159181399997266
2.14e-05 0.000159266399997193
1e-05 0.000159285399998765
};
\addplot [semithick, color2, mark=*, mark size=3, mark options={solid}]
table {%
0.01 0.0058274429999976
0.00457 0.00117730399999871
0.00214 0.000145728999999761
0.001 8.01760000008755e-05
0.000457 0.000130020000000286
0.000214 0.000140351000002426
0.0001 0.000142615000001456
4.57e-05 0.000143116000000276
2.14e-05 0.000143219000001693
1e-05 0.000143242000000043
};
\nextgroupplot[
title={${\frac{\partial F_y}{\partial Ma}}_{analytic}$ - $\frac{\partial F_y}{\partial Ma}_{FD}$},
xlabel={$\epsilon$},
xmin=7.07945784384137e-06, xmax=0.0141253754462276,
ymin=2.95895548606447e-06, ymax=0.0143023861305332,
xmode=log,
ymode=log,
tick align=outside,
xtick pos=left,
ytick pos=right,
x grid style={white!69.019607843137251!black},
y grid style={white!69.019607843137251!black},
legend style={at={(0.03,0.97)}, anchor=north west, draw=white!80.0!black},
legend entries={{$\alpha_o$=0.0},{$\alpha_o$=3.0},{$\alpha_o$=6.0}},
legend cell align={left}
]
\addplot [semithick, color0, mark=*, mark size=3, mark options={solid}]
table {%
0.01 0.000369655272000002
0.00457 8.21740620000067e-05
0.00214 2.16043119999898e-05
0.001 4.35116199998253e-06
0.000457 2.90071319999918e-05
0.000214 2.85016219999934e-05
0.0001 2.83905919999938e-05
4.57e-05 2.83666719999853e-05
2.14e-05 2.83599019999803e-05
1e-05 2.83596319999857e-05
};
\addplot [semithick, color1, mark=*, mark size=3, mark options={solid}]
table {%
0.01 0.00367593489999507
0.00457 0.00112775889999739
0.00214 0.00058711690000024
0.001 0.00046708889999536
0.000457 0.000436678900001652
0.000214 0.00042127089999866
0.0001 0.000420002900000327
4.57e-05 0.000419747899996992
2.14e-05 0.000419696899996325
1e-05 0.000419683899998802
};
\addplot [semithick, color2, mark=*, mark size=3, mark options={solid}]
table {%
0.01 0.0097261660000072
0.00457 0.0021997160000069
0.00214 0.00097066600000062
0.001 0.000700706000003493
0.000457 0.00064134600000898
0.000214 0.000629136000000585
0.0001 0.000626426000010838
4.57e-05 0.000625826000003826
2.14e-05 0.000625706000008108
1e-05 0.000625676000012731
};
\end{groupplot}

\end{tikzpicture}
%	  \caption[Validation of the Lift and Drag results for shape-sensitivity: body-fitted \eqtype equations]{The graph shows the relative error between the Lift and Drag shape-sensitivities obtained by \ac{FD} of two steady states, compared to the sensitivity obtained by analytic direct evaluation. Please not that a \ac{FD} Jacobian has been used here, which introduces an approximation error, Furthermore, the finite precision of the obtained steady states is another error source.This explains why the error levels off. For all practical optimization purposes, the obtained accuracy is more than enough.}
%	  \label{fig:validation_liftdrag_shapesens_bodyfitted_\eqtype}
%	\end{figure}
%}



%\begin{figure}
%  \textbf{Force convergence machsens ALE Euler body-fitted}
%  % This file was created by matplotlib2tikz v0.6.7.
\begin{tikzpicture}[scale=0.75]

\definecolor{color0}{rgb}{0.12156862745098,0.466666666666667,0.705882352941177}
\definecolor{color1}{rgb}{1,0.498039215686275,0.0549019607843137}
\definecolor{color2}{rgb}{0.172549019607843,0.627450980392157,0.172549019607843}

\begin{groupplot}[group style={group size=2 by 1}]
\nextgroupplot[
title={${\frac{\partial L_x}{\partial Ma}}_{analytic}$ - $\frac{\partial L_x}{\partial Ma}_{FD}$},
xlabel={$\epsilon$},
xmin=7.07945784384137e-06, xmax=0.0141253754462276,
ymin=1.0997870763893e-05, ymax=0.00410495705563733,
xmode=log,
ymode=log,
tick align=outside,
tick pos=left,
x grid style={lightgray!92.026143790849673!black},
y grid style={lightgray!92.026143790849673!black},
legend entries={{$\alpha_o$=0.0},{$\alpha_o$=3.0},{$\alpha_o$=6.0}},
legend cell align={left},
legend style={at={(0.03,0.97)}, anchor=north west, draw=white!80.0!black}
]
\addplot [semithick, color0, mark=*, mark size=3, mark options={solid}]
table {%
0.01 0.00216151720000113
0.00457 0.000455843200001027
0.00214 0.000106659200000081
0.001 3.39452000002183e-05
0.000457 1.88342000004837e-05
0.000214 1.53542000003171e-05
0.0001 1.45852000006386e-05
4.57e-05 1.44192000011145e-05
2.14e-05 1.43952000009051e-05
1e-05 1.44051999999562e-05
};
\addplot [semithick, color1, mark=*, mark size=3, mark options={solid}]
table {%
0.01 0.00238611229999997
0.00457 0.00051455329999861
0.00214 0.000132214299998878
0.001 5.50442999998069e-05
0.000457 2.11932999985009e-05
0.000214 1.75152999997152e-05
0.0001 1.66812999999877e-05
4.57e-05 1.64942999987261e-05
2.14e-05 1.64562999991347e-05
1e-05 1.644729999839e-05
};
\addplot [semithick, color2, mark=*, mark size=3, mark options={solid}]
table {%
0.01 0.0031361694999994
0.00457 0.000669670500002439
0.00214 0.000163671500001072
0.001 5.34985000015809e-05
0.000457 2.99185000010027e-05
0.000214 2.47125000001347e-05
0.0001 2.36015000005807e-05
4.57e-05 2.33555000015429e-05
2.14e-05 2.33055000009585e-05
1e-05 2.32945000000484e-05
};
\nextgroupplot[
title={${\frac{\partial L_y}{\partial Ma}}_{analytic}$ - $\frac{\partial L_y}{\partial Ma}_{FD}$},
xlabel={$\epsilon$},
xmin=7.07945784384137e-06, xmax=0.0141253754462276,
ymin=1.61645869634857e-06, ymax=0.000946955598502839,
xmode=log,
ymode=log,
tick align=outside,
xtick pos=left,
ytick pos=right,
x grid style={lightgray!92.026143790849673!black},
y grid style={lightgray!92.026143790849673!black},
legend entries={{$\alpha_o$=0.0},{$\alpha_o$=3.0},{$\alpha_o$=6.0}},
legend cell align={left},
legend style={at={(0.03,0.97)}, anchor=north west, draw=white!80.0!black}
]
\addplot [semithick, color0, mark=*, mark size=3, mark options={solid}]
table {%
0.01 0.000604994063999997
0.00457 0.000464485104000045
0.00214 0.000384464194000012
0.001 0.000270318774000022
0.000457 3.06226139999977e-05
0.000214 2.20361400005142e-06
0.0001 2.1657540000275e-06
4.57e-05 2.15959400001742e-06
2.14e-05 2.16478400000275e-06
1e-05 2.17661400003788e-06
};
\addplot [semithick, color1, mark=*, mark size=3, mark options={solid}]
table {%
0.01 5.60427000024788e-05
0.00457 0.000102894299999434
0.00214 8.44722999957526e-05
0.001 0.000108613300000115
0.000457 4.76792999961617e-05
0.000214 4.7981299999833e-05
0.0001 4.86042999980896e-05
4.57e-05 4.86382999937973e-05
2.14e-05 4.86342999934664e-05
1e-05 4.86332999969363e-05
};
\addplot [semithick, color2, mark=*, mark size=3, mark options={solid}]
table {%
0.01 0.00070879739999441
0.00457 0.000206129399998645
0.00214 0.000120153399990386
0.001 0.00010602339999366
0.000457 0.000103380399991693
0.000214 0.000104129400000375
0.0001 0.000103993399989122
4.57e-05 0.000103961400000685
2.14e-05 0.000103956399996719
1e-05 0.000103954399989448
};
\end{groupplot}

\end{tikzpicture}
%\end{figure}
%
%
%\begin{figure}
%  \textbf{Force convergence machsens ALE Laminar body-fitted}
%  % This file was created by matplotlib2tikz v0.6.7.
\begin{tikzpicture}[scale=0.75]

\definecolor{color0}{rgb}{0.12156862745098,0.466666666666667,0.705882352941177}
\definecolor{color1}{rgb}{1,0.498039215686275,0.0549019607843137}
\definecolor{color2}{rgb}{0.172549019607843,0.627450980392157,0.172549019607843}

\begin{groupplot}[group style={group size=2 by 1}]
\nextgroupplot[
title={${\frac{\partial L_x}{\partial Ma}}_{analytic}$ - $\frac{\partial L_x}{\partial Ma}_{FD}$},
xlabel={$\epsilon$},
xmin=7.07945784384137e-06, xmax=0.0141253754462276,
ymin=1.0997870763893e-05, ymax=0.00410495705563733,
xmode=log,
ymode=log,
tick align=outside,
tick pos=left,
x grid style={lightgray!92.026143790849673!black},
y grid style={lightgray!92.026143790849673!black},
legend entries={{$\alpha_o$=0.0},{$\alpha_o$=3.0},{$\alpha_o$=6.0}},
legend cell align={left},
legend style={at={(0.03,0.97)}, anchor=north west, draw=white!80.0!black}
]
\addplot [semithick, color0, mark=*, mark size=3, mark options={solid}]
table {%
0.01 0.00216151720000113
0.00457 0.000455843200001027
0.00214 0.000106659200000081
0.001 3.39452000002183e-05
0.000457 1.88342000004837e-05
0.000214 1.53542000003171e-05
0.0001 1.45852000006386e-05
4.57e-05 1.44192000011145e-05
2.14e-05 1.43952000009051e-05
1e-05 1.44051999999562e-05
};
\addplot [semithick, color1, mark=*, mark size=3, mark options={solid}]
table {%
0.01 0.00238611229999997
0.00457 0.00051455329999861
0.00214 0.000132214299998878
0.001 5.50442999998069e-05
0.000457 2.11932999985009e-05
0.000214 1.75152999997152e-05
0.0001 1.66812999999877e-05
4.57e-05 1.64942999987261e-05
2.14e-05 1.64562999991347e-05
1e-05 1.644729999839e-05
};
\addplot [semithick, color2, mark=*, mark size=3, mark options={solid}]
table {%
0.01 0.0031361694999994
0.00457 0.000669670500002439
0.00214 0.000163671500001072
0.001 5.34985000015809e-05
0.000457 2.99185000010027e-05
0.000214 2.47125000001347e-05
0.0001 2.36015000005807e-05
4.57e-05 2.33555000015429e-05
2.14e-05 2.33055000009585e-05
1e-05 2.32945000000484e-05
};
\nextgroupplot[
title={${\frac{\partial L_y}{\partial Ma}}_{analytic}$ - $\frac{\partial L_y}{\partial Ma}_{FD}$},
xlabel={$\epsilon$},
xmin=7.07945784384137e-06, xmax=0.0141253754462276,
ymin=1.61645869634857e-06, ymax=0.000946955598502839,
xmode=log,
ymode=log,
tick align=outside,
xtick pos=left,
ytick pos=right,
x grid style={lightgray!92.026143790849673!black},
y grid style={lightgray!92.026143790849673!black},
legend entries={{$\alpha_o$=0.0},{$\alpha_o$=3.0},{$\alpha_o$=6.0}},
legend cell align={left},
legend style={at={(0.03,0.97)}, anchor=north west, draw=white!80.0!black}
]
\addplot [semithick, color0, mark=*, mark size=3, mark options={solid}]
table {%
0.01 0.000604994063999997
0.00457 0.000464485104000045
0.00214 0.000384464194000012
0.001 0.000270318774000022
0.000457 3.06226139999977e-05
0.000214 2.20361400005142e-06
0.0001 2.1657540000275e-06
4.57e-05 2.15959400001742e-06
2.14e-05 2.16478400000275e-06
1e-05 2.17661400003788e-06
};
\addplot [semithick, color1, mark=*, mark size=3, mark options={solid}]
table {%
0.01 5.60427000024788e-05
0.00457 0.000102894299999434
0.00214 8.44722999957526e-05
0.001 0.000108613300000115
0.000457 4.76792999961617e-05
0.000214 4.7981299999833e-05
0.0001 4.86042999980896e-05
4.57e-05 4.86382999937973e-05
2.14e-05 4.86342999934664e-05
1e-05 4.86332999969363e-05
};
\addplot [semithick, color2, mark=*, mark size=3, mark options={solid}]
table {%
0.01 0.00070879739999441
0.00457 0.000206129399998645
0.00214 0.000120153399990386
0.001 0.00010602339999366
0.000457 0.000103380399991693
0.000214 0.000104129400000375
0.0001 0.000103993399989122
4.57e-05 0.000103961400000685
2.14e-05 0.000103956399996719
1e-05 0.000103954399989448
};
\end{groupplot}

\end{tikzpicture}
%\end{figure}
% 
%\begin{figure}
%  \textbf{Force convergence machsens ALE RANS body-fitted}
%  % This file was created by matplotlib2tikz v0.6.7.
\begin{tikzpicture}[scale=0.75]

\definecolor{color0}{rgb}{0.12156862745098,0.466666666666667,0.705882352941177}
\definecolor{color1}{rgb}{1,0.498039215686275,0.0549019607843137}
\definecolor{color2}{rgb}{0.172549019607843,0.627450980392157,0.172549019607843}

\begin{groupplot}[group style={group size=2 by 1}]
\nextgroupplot[
title={${\frac{\partial L_x}{\partial Ma}}_{analytic}$ - $\frac{\partial L_x}{\partial Ma}_{FD}$},
xlabel={$\epsilon$},
xmin=7.07945784384137e-06, xmax=0.0141253754462276,
ymin=1.0997870763893e-05, ymax=0.00410495705563733,
xmode=log,
ymode=log,
tick align=outside,
tick pos=left,
x grid style={lightgray!92.026143790849673!black},
y grid style={lightgray!92.026143790849673!black},
legend entries={{$\alpha_o$=0.0},{$\alpha_o$=3.0},{$\alpha_o$=6.0}},
legend cell align={left},
legend style={at={(0.03,0.97)}, anchor=north west, draw=white!80.0!black}
]
\addplot [semithick, color0, mark=*, mark size=3, mark options={solid}]
table {%
0.01 0.00216151720000113
0.00457 0.000455843200001027
0.00214 0.000106659200000081
0.001 3.39452000002183e-05
0.000457 1.88342000004837e-05
0.000214 1.53542000003171e-05
0.0001 1.45852000006386e-05
4.57e-05 1.44192000011145e-05
2.14e-05 1.43952000009051e-05
1e-05 1.44051999999562e-05
};
\addplot [semithick, color1, mark=*, mark size=3, mark options={solid}]
table {%
0.01 0.00238611229999997
0.00457 0.00051455329999861
0.00214 0.000132214299998878
0.001 5.50442999998069e-05
0.000457 2.11932999985009e-05
0.000214 1.75152999997152e-05
0.0001 1.66812999999877e-05
4.57e-05 1.64942999987261e-05
2.14e-05 1.64562999991347e-05
1e-05 1.644729999839e-05
};
\addplot [semithick, color2, mark=*, mark size=3, mark options={solid}]
table {%
0.01 0.0031361694999994
0.00457 0.000669670500002439
0.00214 0.000163671500001072
0.001 5.34985000015809e-05
0.000457 2.99185000010027e-05
0.000214 2.47125000001347e-05
0.0001 2.36015000005807e-05
4.57e-05 2.33555000015429e-05
2.14e-05 2.33055000009585e-05
1e-05 2.32945000000484e-05
};
\nextgroupplot[
title={${\frac{\partial L_y}{\partial Ma}}_{analytic}$ - $\frac{\partial L_y}{\partial Ma}_{FD}$},
xlabel={$\epsilon$},
xmin=7.07945784384137e-06, xmax=0.0141253754462276,
ymin=1.61645869634857e-06, ymax=0.000946955598502839,
xmode=log,
ymode=log,
tick align=outside,
xtick pos=left,
ytick pos=right,
x grid style={lightgray!92.026143790849673!black},
y grid style={lightgray!92.026143790849673!black},
legend entries={{$\alpha_o$=0.0},{$\alpha_o$=3.0},{$\alpha_o$=6.0}},
legend cell align={left},
legend style={at={(0.03,0.97)}, anchor=north west, draw=white!80.0!black}
]
\addplot [semithick, color0, mark=*, mark size=3, mark options={solid}]
table {%
0.01 0.000604994063999997
0.00457 0.000464485104000045
0.00214 0.000384464194000012
0.001 0.000270318774000022
0.000457 3.06226139999977e-05
0.000214 2.20361400005142e-06
0.0001 2.1657540000275e-06
4.57e-05 2.15959400001742e-06
2.14e-05 2.16478400000275e-06
1e-05 2.17661400003788e-06
};
\addplot [semithick, color1, mark=*, mark size=3, mark options={solid}]
table {%
0.01 5.60427000024788e-05
0.00457 0.000102894299999434
0.00214 8.44722999957526e-05
0.001 0.000108613300000115
0.000457 4.76792999961617e-05
0.000214 4.7981299999833e-05
0.0001 4.86042999980896e-05
4.57e-05 4.86382999937973e-05
2.14e-05 4.86342999934664e-05
1e-05 4.86332999969363e-05
};
\addplot [semithick, color2, mark=*, mark size=3, mark options={solid}]
table {%
0.01 0.00070879739999441
0.00457 0.000206129399998645
0.00214 0.000120153399990386
0.001 0.00010602339999366
0.000457 0.000103380399991693
0.000214 0.000104129400000375
0.0001 0.000103993399989122
4.57e-05 0.000103961400000685
2.14e-05 0.000103956399996719
1e-05 0.000103954399989448
};
\end{groupplot}

\end{tikzpicture}
%\end{figure}


%\begin{figure}
%  \textbf{Bla bla bla}
%  % This file was created by matplotlib2tikz v0.6.7.
\begin{tikzpicture}

\definecolor{color1}{rgb}{0.301960784313725,0.309803921568627,0.325490196078431}
\definecolor{color0}{rgb}{0.549019607843137,0.0823529411764706,0.0823529411764706}
\definecolor{color2}{rgb}{0.701960784313725,0.6,0.364705882352941}

\begin{groupplot}[group style={group size=2 by 1}]
\nextgroupplot[
title={${\frac{\partial F_x}{\partial Ma}}_{analytic}$ - $\frac{\partial F_x}{\partial Ma}_{FD}$},
xlabel={$\epsilon$},
xmin=7.07945784384137e-06, xmax=0.0141253754462276,
ymin=2.12330574687735e-05, ymax=0.00761369436377756,
xmode=log,
ymode=log,
tick align=outside,
tick pos=left,
x grid style={white!69.019607843137251!black},
y grid style={white!69.019607843137251!black},
legend style={at={(0.03,0.97)}, anchor=north west, draw=white!80.0!black},
legend entries={{$\alpha_o$=0.0},{$\alpha_o$=3.0},{$\alpha_o$=6.0}},
legend cell align={left}
]
\addplot [semithick, color0, mark=*, mark size=3, mark options={solid}]
table {%
0.01 0.0042630664999983
0.00457 0.000814685499999968
0.00214 2.77414999985126e-05
0.001 0.000137136500001134
0.000457 0.000167667500001301
0.000214 0.000175603500000676
0.0001 0.000177345499999149
4.57e-05 0.000177731499999112
2.14e-05 0.000177811500002178
1e-05 0.00017782749999995
};
\addplot [semithick, color1, mark=*, mark size=3, mark options={solid}]
table {%
0.01 0.00512230860000074
0.00457 0.000954174600000357
0.00214 8.75556000003996e-05
0.001 0.000103155399997945
0.000457 0.000145890399998905
0.000214 0.000156877399998478
0.0001 0.000158766399998456
4.57e-05 0.000159181399997266
2.14e-05 0.000159266399997193
1e-05 0.000159285399998765
};
\addplot [semithick, color2, mark=*, mark size=3, mark options={solid}]
table {%
0.01 0.0058274429999976
0.00457 0.00117730399999871
0.00214 0.000145728999999761
0.001 8.01760000008755e-05
0.000457 0.000130020000000286
0.000214 0.000140351000002426
0.0001 0.000142615000001456
4.57e-05 0.000143116000000276
2.14e-05 0.000143219000001693
1e-05 0.000143242000000043
};
\nextgroupplot[
title={${\frac{\partial F_y}{\partial Ma}}_{analytic}$ - $\frac{\partial F_y}{\partial Ma}_{FD}$},
xlabel={$\epsilon$},
xmin=7.07945784384137e-06, xmax=0.0141253754462276,
ymin=2.95895548606447e-06, ymax=0.0143023861305332,
xmode=log,
ymode=log,
tick align=outside,
xtick pos=left,
ytick pos=right,
x grid style={white!69.019607843137251!black},
y grid style={white!69.019607843137251!black},
legend style={at={(0.03,0.97)}, anchor=north west, draw=white!80.0!black},
legend entries={{$\alpha_o$=0.0},{$\alpha_o$=3.0},{$\alpha_o$=6.0}},
legend cell align={left}
]
\addplot [semithick, color0, mark=*, mark size=3, mark options={solid}]
table {%
0.01 0.000369655272000002
0.00457 8.21740620000067e-05
0.00214 2.16043119999898e-05
0.001 4.35116199998253e-06
0.000457 2.90071319999918e-05
0.000214 2.85016219999934e-05
0.0001 2.83905919999938e-05
4.57e-05 2.83666719999853e-05
2.14e-05 2.83599019999803e-05
1e-05 2.83596319999857e-05
};
\addplot [semithick, color1, mark=*, mark size=3, mark options={solid}]
table {%
0.01 0.00367593489999507
0.00457 0.00112775889999739
0.00214 0.00058711690000024
0.001 0.00046708889999536
0.000457 0.000436678900001652
0.000214 0.00042127089999866
0.0001 0.000420002900000327
4.57e-05 0.000419747899996992
2.14e-05 0.000419696899996325
1e-05 0.000419683899998802
};
\addplot [semithick, color2, mark=*, mark size=3, mark options={solid}]
table {%
0.01 0.0097261660000072
0.00457 0.0021997160000069
0.00214 0.00097066600000062
0.001 0.000700706000003493
0.000457 0.00064134600000898
0.000214 0.000629136000000585
0.0001 0.000626426000010838
4.57e-05 0.000625826000003826
2.14e-05 0.000625706000008108
1e-05 0.000625676000012731
};
\end{groupplot}

\end{tikzpicture}
%\end{figure}


\end{document}





