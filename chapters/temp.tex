\documentclass[../main.tex]{subfiles}
%\usepackage{algorithm}
%\usepackage{algorithmic}
\everymath{\displaystyle}
\def\arraystretch{2.0}
\begin{document}
\setlength{\delimitershortfall}{0pt}

\section{Derivative of the Numerical flux with respect to the fluid state variables}
This section is denoted to a thorough derivation of the derivative $\pdfrac{\fluxesnum}{\fstate_i}$.

First, we recall the definition of the numerical flux as
\begin{align}
\fluxesnum(\fstate_i,\fstate_j,\normal_{ij}) =
\frac{1}{2}\left[\fluxesconv(\fstate_i)\cdot\normal_{ij} +
                 \fluxesconv(\fstate_i)\cdot\normal_{ij}   \right] -
\frac{1}{2}\abs{\roeavgmat(\fstate_i,\fstate_j,\normal_{ij})} (\fstate_j-\fstate_i)
\end{align}

Similar to the MUSCL procedure \textbf{TODO} we trransform the fluid state vector to primitive form
\begin{align}
\fstate=\begin{pmatrix}
        \dens \\ \dens\fluidvel \\ \energytot
        \end{pmatrix}
\overset{\mathsf{U}}{\mapsto}
\prim{\fstate}=\begin{pmatrix}
               \dens \\ \fluidvel \\ \pres
               \end{pmatrix}
\end{align}
Of course, this transformation involves the Equation of state of the fluid. We will use $\prim{(\cdot)}$ from now on top mark any quantity as based on/formulated in primitive variables. The primitive state vector is also much more appealing when it comes to the implementation of boundary conditions, since the solution variables are seperated there.
 \\
Next we recall the conservative form of the Euler equations, which was presented in REF as
\begin{align}
\pdfrac{\fstate}{t}+\nabla\cdot\fluxesconv{\fstate}=\vec{0}
\end{align}
Where the convective flux matrix $\fluxesconv$ is provided in equation \REF, and can be writtes as
\begin{align}
\fluxesconv(\fstate)=\begin{pmatrix}
                      \fluxesconv_x(\fstate)&\fluxesconv_x(\fstate)&\fluxesconv_x(\fstate)
                      \end{pmatrix}
\end{align}
Therefore, applying the Nabla operator in Equation \REF gives
\begin{align}
\pdfrac{\fstate}{t}+
\underbrace{\pdfrac{\fluxesconv_x(\fstate)}{\fstate}}_{\tensor{A}} \pdfrac{\fstate}{x} +
\underbrace{\pdfrac{\fluxesconv_y(\fstate)}{\fstate}}_{\tensor{B}} \pdfrac{\fstate}{y} +
\underbrace{\pdfrac{\fluxesconv_z(\fstate)}{\fstate}}_{\tensor{C}} \pdfrac{\fstate}{z}
=\vec{0}
\end{align}
where the matrices $\tensor{A}$,$\tensor{B}$ and $\tensor{C}$ are called the \expression{flux Jacobians}.
\end{document}

\textbf{Notation error when it comes to marking qiantities as continous, discrete, etc}
In Section \REF, we have derived that for any sensitivity calculation, the derivative of the flux matrix with respect to the fluid state vector $\pdfrac{\fluxmatconv}{\fstate}$is required.\\
We have derived in Section \REF. that at an interior vertex point of the fluid mesh, we have
\begin{align}
\left[\fluxmatconv(\st{\fstate},\st{\mms},\st{\dot{\mms}})\right]_i=
\sum_{j \in \vertexset(i)} mes(\partial C_{ij})\fluxesnum(\st{\fstate}_i,\st{\fstate}_j,\normal_{ij})
\end{align}
Since an equilibrium point is considered, $\st{\dot{\mms}}=\vec{0}$ and is neglected in the following derivations.

In section \REF we have introduced $\fluxesnum$ as the numerical flux function. In this thesis, we consider the popular Roe flux, which is defined as given in eaution \REF.

Let us remind, that $\fstate_i$ is the flux vector at vertex $i$, $\vertexset(i)$ is the set of vertices connected to vertex $i$ by an edge, $C_i$ is the control volume of the dual cell centered at vertex $i$ and $\partial C_ij$ is the segment of the boundary that intersects edge $(ij)$ and $\normal_{ij}$ is the outward facing normal to $\partial C_{ij}$
 \\
From Equation \REF it is obvious that when looking for the derivative $\pdfrac{\fluxmatconv}{\fstate}$, the derivative of the numerical flux with respect to the fluid state vector $\pdfrac{\fluxesnum}{\fstate}$ is required.

To make things easier we will in the following derive the primitive version of that quantity $\pdfrac{\prim{\fluxesnum}}{\prim{\fstate}}$
 \\
 \\
The primitive version of the numerical Roe flux can be expressed as
\begin{align}
\prim{\fluxesconv}(\prim{\fstate}_i,\prim{\fstate}_j,\normal_{ij}) =
\frac{1}{2} \left(\prim{\fluxesconv}(\prim{\fstate}_i)\cdot\normal_{ij} +
                  \prim{\fluxesconv}(\prim{\fstate}_i)\cdot\normal_{ij}
            \right) -
\frac{1}{2} \inv{\jaceigvecs}(\sim)\jaceigvals(\sim)\jaceigvecs(\sim)(\prim{\fstate}_j-\prim{\fstate}_i) \\
\end{align}
where, for ease of notation
\begin{align}
\inv{\jaceigvecs}(\sim) &= \inv{\jaceigvecs}(\roeavgfunc(\sim),\normal_{ij}) \\
\roeavgfunc(\sim)       &= \roeavgfunc(\prim{\fstate}_i,\prim{\fstate}_j)    \\
\jaceigvals(\sim)       &= \jaceigvals(\roeavgfunc(\sim),\normal_{ij})        \\
\jaceigvecs(\sim)       &= \jaceigvecs(\roeavgfunc(\sim),\normal_{ij})        \\
\end{align}


The Jacobi matrix in prmitive form can be written as
\def\vn{\fluidvel\cdot\normal}
\def\enthalp{\mathsf{H}}
\def\jfo{H~(\vn)-\dens\vn \frac{\gamma\pres}{\dens(\gamma)}}
\def\jft{\dens\vn v_1+\dens H n_1}
\def\jftt{\dens\vn v_2+\dens H n_2}
\def\jff{\dens\vn v_3+\dens H n_3}
\def\jfff{\dens\vn\frac{\gamma}{\dens(\gamma-1)}}

\textbf{TODO notation}
\begin{align}
\prim{\fluxjac}&=\pdfrac{(\prim{\fluxesconv}\cdot\normal)}{\prim{\fstate}} \\
&=
\begin{bmatrix}
\vn      &  \dens n_x                &  \dens n_y                &  \dens n_z                 &  0   \\
v_1(\vn) &  \dens(\vn)+\dens v_1 n_1 &  \dens v_1 n_2            &  \dens v_1 n_3             &  n_1 \\
v_2(\vn) &  \dens v_2 n_1            &  \dens(\vn)+\dens v_2 n_2 &  \dens v_2 n_3             &  n_2 \\
v_3(\vn) &  \dens v_3 n_1            &  \dens v_3 n_2            &  \dens(\vn)+\dens v_3 n_3  &  n_2 \\
A_{51}   &  A_{52}                   &  A_{53}                   &  A_{54}                    & A_{55}
\end{bmatrix} \notag\\
\fluxjac_{51}&=\jfo  \notag\\
\fluxjac_{52}&=\jft  \notag\\
\fluxjac_{53}&=\jftt \notag\\
\fluxjac_{54}&=\jff  \notag\\
\fluxjac_{55}&=\jfff \notag\\
\end{align}


For this Jacobian, one can now symbolically derive an eigenvalue decomposition, such that
\textbf{TODO notation}
\begin{align}
\prim{\jac}=\inv{\jaceigvecs}\jaceigvals\jaceigvecs
\end{align}


The averaging function associated with the roe flux can be written as
\begin{align}
\roeavgfunc(\prim{\fstate}_i,\prim{\fstate}_j)=
\frac{1}{\sqrt{\rho_i}+\sqrt{\rho_j}}
\left(
\sqrt{\dens_i}
\begin{bmatrix}
\dens_i\\ \fluidvel_i \\ \enthalp_i
\end{bmatrix}
+
\sqrt{\dens_j}
\begin{bmatrix}
\dens_j \\ \fluidvel_j \\ \enthalp_j
\end{bmatrix}
\right)
\end{align}

The total specific enthalpy for a perfect gas is given as
\begin{align}
\enthalp=\frac{\specheatratio\pres}{\dens(\specheatratio-1)}+\frac{1}{2}\T{\fluidvel}\fluidvel
\end{align}
where we note the following derivatives
\begin{align}
\pdfrac{\enthalp}{\dens}=-\frac{\specheatratio\pres}{\dens^2(\specheatratio-1)}~~~~~~
\pdfrac{\enthalp}{\fluidvelcomp_i}=\fluidvelcomp_i~~~~~~
\pdfrac{\enthalp}{\pres}=\frac{\specheatratio}{\dens(\specheatratio-1)}
\end{align}

Therefore, the derivative of the averaging function with respect to the primitive fluid state vector can be calculated as
\def\sqdi{{\sqrt{\dens_{i}}}}
\def\sqdj{{\sqrt{\dens_{j} }}}
\def\Mwoneone{3\dens_i-\frac{\sqdi\dens_i+\sqdj\dens_j}{\sqdi+\sqdj}}
\def\Mwonetwo  {\fluidvelx_i-\frac{\sqdi\fluidvelx_i+\sqdj\fluidvelx_j}{\sqdi+\sqdj}}
\def\Mwonethree{\fluidvely_i-\frac{\sqdi\fluidvely_i+\sqdj\fluidvely_j}{\sqdi+\sqdj}}
\def\Mwonefour {\fluidvelz_i-\frac{\sqdi\fluidvelz_i+\sqdj\fluidvelz_j}{\sqdi+\sqdj}}
\def\Mwonefive{\frac{\T{\fluidvel}\fluidvel}{2}-\frac{\specheatratio\pres}{\dens(\specheatratio-1)}-\frac{\sqdi\enthalp_i+\sqdj\enthalp_j}{\sqdi+\sqdj}  }
\begin{align}
\pdfrac{\roeavgfunc(\sim)}{\dens_i}&=
\frac{1}{2\sqrt{\dens_i}(\sqrt{\dens_i}+\sqrt{\dens_j})}
\left(
  -\roeavgfunc(\sim)+
  \T{\begin{bmatrix}
    3\dens_i & 2\sqrt{\dens_i}\fluidvel_i & 2\enthalp_i-2\dens_i\frac{\specheatratio\pres}{\dens^2(\specheatratio-1)}
  \end{bmatrix}}
\right)                                                                                                                      \notag\\
\pdfrac{\roeavgfunc(\sim)}{\fluidvelcomp_i}&=
\frac{1}{(\sqrt{\dens_i}+\sqrt{\dens_j})}\sqrt{\dens_i}
\T{\begin{bmatrix}
  0 & \T{\vec{e}_i} & \fluidvelcomp_i
\end{bmatrix}}                                                                                                               \notag\\
\pdfrac{\roeavgfunc(\sim)}{\pres_i} &=
\frac{1}{(\sqrt{\dens_i}+\sqrt{\dens_j})}\sqrt{\dens_i}
\T{\begin{bmatrix}
  0 & \T{\vec{0}} & \frac{\specheatratio}{\dens(\specheatratio-1)}
\end{bmatrix}}                                                                                                               \notag\\
\pdfrac{\roeavgfunc(\sim)}{\prim{\fstate}_i}&=
\begin{bmatrix}
  \Mwoneone  &  0                    &  0                    &  0                    & 0                                         \\
  \Mwonetwo   &  2\dens_i             &  0                    &  0                    & 0                                        \\
  \Mwonethree &  0                    &  2\dens_i             &  0                    & 0                                        \\
  \Mwonefour  &  0                    &  0                    &  2\dens_i             & 0                                        \\
  \Mwonefive  &  2\dens_i\fluidvelx_i &  2\dens_i\fluidvelx_i &  2\dens_i\fluidvelx_i & \frac{2\specheatratio}{\specheatratio-1} \\
\end{bmatrix}                                                                                                                      
\end{align}


The matrix $\jaceigvecs$ can be written as
\def\gmo{\specheatratio-1}
\def\tone{\normal-\frac{(\gmo)\T{\fluidvel}\fluidvel }{2 c^2}\normal+\normal\times\fluidvel}
\begin{align}
\resizebox{.9\hsize}{!}{$\jaceigvecs=
\begin{bmatrix}
\left(\tone\right)\cdot\vec{e}_1 &  \normalx(\gmo)\frac{\fluidvelx}{c^2} &  \normalx(\gmo)\frac{\fluidvely}{c^2}+\normalz &  \normalx(\gmo)\frac{\fluidvelz}{c^2}-\normaly  &  -\normalx\frac{\gmo}{c^2} \\
\left(\tone\right)\cdot\vec{e}_2 &  \normaly(\gmo)\frac{\fluidvelx}{c^2}-\normalz &  \normaly(\gmo)\frac{\fluidvely}{c^2} &  \normaly(\gmo)\frac{\fluidvelz}{c^2}+\normalx  &  -\normaly\frac{\gmo}{c^2} \\
\left(\tone\right)\cdot\vec{e}_3 &  \normalz(\gmo)\frac{\fluidvelx}{c^2}+\normaly &  \normalz(\gmo)\frac{\fluidvely}{c^2}-\normalx &  \normalz(\gmo)\frac{\fluidvelz}{c^2}  &  -\normaly\frac{\gmo}{c^2} \\
(\gmo)\frac{\T{\fluidvel}\fluidvel}{c^2}-c\fluidvel\cdot\normal & -(\gmo)\fluidvelx+c\normalx & -(\gmo)\fluidvely+c\normaly & -(\gmo)\fluidvelz+c\normalz & \gmo \\
(\gmo)\frac{\T{\fluidvel}\fluidvel}{c^2}-c\fluidvel\cdot\normal & -(\gmo)\fluidvelx-c\normalx & -(\gmo)\fluidvely-c\normaly & -(\gmo)\fluidvelz-c\normalz & \gmo \\
\end{bmatrix}$}
\end{align}
where $c$ is the speed of sound given by $c=\sqrt{\frac{\specheatratio\pres}{\dens}}$

The matrix $\jaceigvals$ derives to
\begin{align}
\jaceigvals=
diag(\left[\fluidvel\cdot\normal,\fluidvel\cdot\normal,\fluidvel\cdot\normal,\fluidvel\cdot\normal+c,\fluidvel\cdot\normal-c)\right])
\end{align}



\begin{align}
\pdfrac{\inv{\jaceigvecs}}{\fstateprim}(\fstateprim,\normal)\cdot b =
\left[
\pdfrac{r_1}{\fstateprim}(\fstateprim,\normal)\cdot b~~~~\pdfrac{r_2}{\fstateprim}(\fstateprim,\normal)\cdot b~~~~\pdfrac{r_3}{\fstateprim}(\fstateprim,\normal)\cdot b~~~~\pdfrac{r_4}{\fstateprim}(\fstateprim,\normal)\cdot b~~~~\pdfrac{r_5}{\fstateprim}(\fstateprim,\normal)\cdot b
\right]
\end{align}

\begin{align}
\pdfrac{r_1}{\fstateprim}(\fstateprim,\normal)=
\begin{bmatrix}
0    &  0                    &  0                           &  0                           &  0 \\
0    &  \normal_x            &  0                           &  0                           &  0 \\
0    &  0                    &  \normal_x                   &  0                           &  0 \\
0    &  0                    &  0                           &  \normal_x                   &  0 \\
0    &  \fluidvelx\normal_x  &  \fluidvely\normalx+\normalz &  \fluidvelz\normalx-\normaly &  0
\end{bmatrix}
\end{align}
\begin{align}
\pdfrac{r_2}{\fstateprim}(\fstateprim,\normal)=
\begin{bmatrix}
0    &  0                             &  0                  &  0                           &  0 \\
0    &  \normal_y                     &  0                  &  0                           &  0 \\
0    &  0                             &  \normal_y          &  0                           &  0 \\
0    &  0                             &  0                  &  \normal_y                   &  0 \\
0    &  \fluidvelx\normaly-\normalz   &  \fluidvely\normaly &  \fluidvelz\normaly-\normalx &  0
\end{bmatrix}
\end{align}
\begin{align}
\pdfrac{r_3}{\fstateprim}(\fstateprim,\normal)=
\begin{bmatrix}
0    &  0                             &  0                           &  0                  &  0 \\
0    &  \normal_z                     &  0                           &  0                  &  0 \\
0    &  0                             &  \normal_z                   &  0                  &  0 \\
0    &  0                             &  0                           &  \normal_z          &  0 \\
0    &  \fluidvelx\normalz+\normaly   &  \fluidvely\normalz-\normalx &  \fluidvelz\normalz &  0
\end{bmatrix}
\end{align}

\begin{align}
\pdfrac{r_4}{\fstateprim}(\fstateprim,\normal)=
\begin{bmatrix}
\frac{1}{2\specheatratio\pres}                            &  0                   &  0                   &  0              & -\frac{\dens}{2\gamma\pres^2}                             \\
\frac{\fluidvelx}{2\dens c^2} + \frac{\normalx}{4\dens c} &  \frac{1}{2 c^2}     &  0                   &  0              &  -\frac{\fluidvelx}{2\pres c^2}-\frac{\normalx}{4\pres c} \\
\frac{\fluidvely}{2\dens c^2} + \frac{\normaly}{4\dens c} &  0                   &  \frac{1}{2 c^2}     &  0              &  -\frac{\fluidvely}{2\pres c^2}-\frac{\normaly}{4\pres c} \\
\frac{\fluidvelz}{2\dens c^2} + \frac{\normalz}{4\dens c} &  0                   &  0                   & \frac{1}{2 c^2} &  -\frac{\fluidvelz}{2\pres c^2}-\frac{\normalz}{4\pres c} \\
\frac{1}{2\specheatratio}+\frac{\fluidvel\cdot\normal}{4\dens c} & \frac{\normalx}{2 c} & \frac{\normaly}{2 c} & \frac{\normalz}{2 c} & -\frac{\fluidvel\cdot\normal}{4\pres c}
\end{bmatrix}
\end{align}

\begin{align}
\pdfrac{r_5}{\fstateprim}(\fstateprim,\normal)=
\begin{bmatrix}
\frac{1}{2\specheatratio\pres}                            &  0                   &  0                   &  0              & -\frac{\dens}{2\gamma\pres^2}                             \\
\frac{\fluidvelx}{2\dens c^2} - \frac{\normalx}{4\dens c} &  \frac{1}{2 c^2}     &  0                   &  0              &  -\frac{\fluidvelx}{2\pres c^2}+\frac{\normalx}{4\pres c} \\
\frac{\fluidvely}{2\dens c^2} - \frac{\normaly}{4\dens c} &  0                   &  \frac{1}{2 c^2}     &  0              &  -\frac{\fluidvely}{2\pres c^2}+\frac{\normaly}{4\pres c} \\
\frac{\fluidvelz}{2\dens c^2} - \frac{\normalz}{4\dens c} &  0                   &  0                   & \frac{1}{2 c^2} &  -\frac{\fluidvelz}{2\pres c^2}+\frac{\normalz}{4\pres c} \\
\frac{1}{2\specheatratio}-\frac{\fluidvel\cdot\normal}{4\dens c} & -\frac{\normalx}{2 c} & -\frac{\normaly}{2 c} & -\frac{\normalz}{2 c} & \frac{\fluidvel\cdot\normal}{4\pres c}
\end{bmatrix}
\end{align}

Due to its shape, the derivation of $\jaceigvals$ can be simplified to a diffrentiation of the eigenvalues $\eigval_i$.
\begin{align}
\pdfrac{\eigval_1}{\fstateprim}(\fstateprim,\normal)&=\pdfrac{\eigval_2}{\fstateprim}(\fstateprim,\normal)=\pdfrac{\eigval_3}{\fstateprim}(\fstateprim,\normal)=
\T{\left[
0~~~~\T{\normal}~~~~0
\right] } \\
\pdfrac{\eigval_4}{\fstateprim}(\fstateprim,\normal)&=
\T{\left[
-\frac{c}{2\dens}~~~~\T{\normal}~~~~\frac{c}{2 \pres}
\right] } \\
\pdfrac{\eigval_5}{\fstateprim}(\fstateprim,\normal)&=
\T{\left[
\frac{c}{2\dens}~~~~\T{\normal}~~~~-\frac{c}{2 \pres}
\right] } \\
\end{align}



\begin{align}
\pdfrac{\jaceigvecs}{\fstateprim}(\fstateprim,\normal)\cdot b=
\begin{bmatrix}
\pdfrac{\T{l_1}}{\fstateprim}(\fstateprim,\normal)\cdot b \\
\pdfrac{\T{l_2}}{\fstateprim}(\fstateprim,\normal)\cdot b \\
\pdfrac{\T{l_3}}{\fstateprim}(\fstateprim,\normal)\cdot b \\
\pdfrac{\T{l_4}}{\fstateprim}(\fstateprim,\normal)\cdot b \\
\pdfrac{\T{l_5}}{\fstateprim}(\fstateprim,\normal)\cdot b \\
\end{bmatrix}
\end{align}

\begin{align}
\resizebox{.9\hsize}{!}{$
\pdfrac{l_1}{\fstateprim}(\fstateprim,\normal)=
\begin{bmatrix}
-\frac{\normalx(\gmo)\norm{\fluidvel}^2}{2\dens\sspeed^2} & -\frac{\normalx(\gmo)\fluidvelx}{\sspeed^2}        & -\frac{\normalx(\gmo)\fluidvely}{\sspeed^2}-\normalz & -\frac{\normalx(\gmo)\fluidvelz}{\sspeed^2}+\normaly & \frac{\normalx(\gmo)\norm{\fluidvel}^2}{2\dens\sspeed^2} \\
\frac{\normalx(\gmo)\fluidvelx}{\specheatratio\pres} & \frac{\normalx\dens(\gmo)}{\specheatratio\pres} & 0 & 0 & -\frac{\normalx\fluidvelx\dens(\gmo)}{\specheatratio\pres^2} \\
\frac{\normalx(\gmo)\fluidvely}{\specheatratio\pres} & 0 & \frac{\normalx\dens(\gmo)}{\specheatratio\pres} & 0 & -\frac{\normalx\fluidvely\dens(\gmo)}{\specheatratio\pres^2} \\
\frac{\normalx(\gmo)\fluidvelz}{\specheatratio\pres} & 0 & 0 &\frac{\normalx\dens(\gmo)}{\specheatratio\pres}  & -\frac{\normalx\fluidvelz\dens(\gmo)}{\specheatratio\pres^2} \\
-\frac{\normalx\gmo}{\dens\sspeed^2}                 & 0 & 0 & 0                                               &  \frac{\normalx\gmo}{\pres\sspeed^2}
\end{bmatrix}
$}
\end{align}

\begin{align}
\resizebox{.9\hsize}{!}{$
\pdfrac{l_2}{\fstateprim}(\fstateprim,\normal)=
\begin{bmatrix}
-\frac{\normaly(\gmo)\norm{\fluidvel}^2}{2\dens\sspeed^2} & -\frac{\normaly(\gmo)\fluidvelx}{\sspeed^2}+\normalz        & -\frac{\normaly(\gmo)\fluidvely}{\sspeed^2} & -\frac{\normaly(\gmo)\fluidvelz}{\sspeed^2}-\normalx & \frac{\normaly(\gmo)\norm{\fluidvel}^2}{2\dens\sspeed^2} \\
\frac{\normaly(\gmo)\fluidvelx}{\specheatratio\pres} & \frac{\normaly\dens(\gmo)}{\specheatratio\pres} & 0 & 0 & -\frac{\normaly\fluidvelx\dens(\gmo)}{\specheatratio\pres^2} \\
\frac{\normaly(\gmo)\fluidvely}{\specheatratio\pres} & 0 & \frac{\normaly\dens(\gmo)}{\specheatratio\pres} & 0 & -\frac{\normaly\fluidvely\dens(\gmo)}{\specheatratio\pres^2} \\
\frac{\normaly(\gmo)\fluidvelz}{\specheatratio\pres} & 0 & 0 &\frac{\normaly\dens(\gmo)}{\specheatratio\pres}  & -\frac{\normaly\fluidvelz\dens(\gmo)}{\specheatratio\pres^2} \\
-\frac{\normaly\gmo}{\dens\sspeed^2}                 & 0 & 0 & 0                                               &  \frac{\normaly\gmo}{\pres\sspeed^2}
\end{bmatrix}
$}
\end{align}

\begin{align}
\resizebox{.9\hsize}{!}{$
\pdfrac{l_3}{\fstateprim}(\fstateprim,\normal)=
\begin{bmatrix}
-\frac{\normalz(\gmo)\norm{\fluidvel}^2}{2\dens\sspeed^2} & -\frac{\normalz(\gmo)\fluidvelx}{\sspeed^2}-\normaly        & -\frac{\normalz(\gmo)\fluidvely}{\sspeed^2}+\normalx & -\frac{\normalz(\gmo)\fluidvelz}{\sspeed^2} & \frac{\normalz(\gmo)\norm{\fluidvel}^2}{2\dens\sspeed^2} \\
\frac{\normalz(\gmo)\fluidvelx}{\specheatratio\pres} & \frac{\normalz\dens(\gmo)}{\specheatratio\pres} & 0 & 0 & -\frac{\normalz\fluidvelx\dens(\gmo)}{\specheatratio\pres^2} \\
\frac{\normalz(\gmo)\fluidvely}{\specheatratio\pres} & 0 & \frac{\normaly\dens(\gmo)}{\specheatratio\pres} & 0 & -\frac{\normalz\fluidvely\dens(\gmo)}{\specheatratio\pres^2} \\
\frac{\normalz(\gmo)\fluidvelz}{\specheatratio\pres} & 0 & 0 &\frac{\normaly\dens(\gmo)}{\specheatratio\pres}  & -\frac{\normalz\fluidvelz\dens(\gmo)}{\specheatratio\pres^2} \\
-\frac{\normalz\gmo}{\dens\sspeed^2}                 & 0 & 0 & 0                                               &  \frac{\normalz\gmo}{\pres\sspeed^2}
\end{bmatrix}
$}
\end{align}

\begin{align}
\pdfrac{l_4}{\fstateprim}(\fstateprim,\normal)=
\begin{bmatrix}
\frac{\sspeed}{2\dens}\fluidvel\cdot\normal & (\gmo)\fluidvelx-\sspeed\normalx & (\gmo)\fluidvely-\sspeed\normaly & (\gmo)\fluidvelz-\sspeed\normalz & -\frac{\sspeed}{2\pres}\fluidvel\cdot\normal \\
-\frac{\sspeed\normalx}{2\dens} &  -(\gmo) &  0       &  0       &  \frac{\sspeed\normalx}{2\pres} \\
-\frac{\sspeed\normaly}{2\dens} &  0       &  -(\gmo) &  0       &  \frac{\sspeed\normaly}{2\pres} \\
-\frac{\sspeed\normalz}{2\dens} &  0       &  0       &  -(\gmo) &  \frac{\sspeed\normalz}{2\pres} \\
0                               &  0       &  0       &  0       &  0                              \\
\end{bmatrix}
\end{align}


\begin{align}
\pdfrac{l_5}{\fstateprim}(\fstateprim,\normal)=
\begin{bmatrix}
\frac{\sspeed}{2\dens}\fluidvel\cdot\normal & (\gmo)\fluidvelx-\sspeed\normalx & (\gmo)\fluidvely-\sspeed\normaly & (\gmo)\fluidvelz-\sspeed\normalz & -\frac{\sspeed}{2\pres}\fluidvel\cdot\normal \\
\frac{\sspeed\normalx}{2\dens} &  -(\gmo) &  0       &  0       &  -\frac{\sspeed\normalx}{2\pres} \\
\frac{\sspeed\normaly}{2\dens} &  0       &  -(\gmo) &  0       &  -\frac{\sspeed\normaly}{2\pres} \\
\frac{\sspeed\normalz}{2\dens} &  0       &  0       &  -(\gmo) &  -\frac{\sspeed\normalz}{2\pres} \\
0                              &  0       &  0       &  0       &  0                               \\
\end{bmatrix}
\end{align}



