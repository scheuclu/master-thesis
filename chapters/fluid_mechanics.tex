\documentclass[../main.tex]{subfiles}
%\usepackage{import}
\begin{document}
\setlength{\delimitershortfall}{0pt}

\chapter{Introduction to fluid mechanics}\label{sec:fluid_mechanics}
%\addcontentsline{lof}{chapter}{Basic equations of fluid mechanics}
\minitoc
%\secttoc
%\sectlof
%\sectlot

\section{Eulerian and Lagrangian view}\label{sec:eulerian_lagrangian}
In time dependent analytical physics there are two basic concepts of how to view a system of interest: the Eulerian and the Lagrangian approach. It shall be stressed here, that Eulerian view and Euler equations are two unrelated concepts.\\
The issue of Lagrangian vs. Eulerian perspective can be explained in many ways, in the end it boils down to the interpretation of the time derivative. Since we will be using the different derivatives frequently in this thesis, a short introduction shall be provided.
\subsection{Material derivative}
The material derivative of a property, is a derivative(rate of change), that follows a specific particle 'p'. That means that at every time instance the material derivative gives as the current value of the property of a specific particle. Since in general, a particle will move its position in time quite a lot, some mean is required to track the particles motion. The material derivative is also known as the Lagrangian concept, and is commonly used in solid mechanics, since the motion of the solid particles is typically little, or at least nearly uniform over the body, which makes position tracking easy.

\subsection{Eulerian derivative}
The Eulerian derivative takes a different approach. It refers to a fixed frame of reference in the domain, and regards the rate of change of a quantity of interest at that specific position. Due to the particle movement, it typically refers to different particles at different times.
\\
Material and Eulerian derivatives can be linked via the so-called convective rate of change as:
\begin{align}
\underbrace{\mfrac{\vec{G}}{t}}_{\text{Lagrangian rate of change}} = \underbrace{\pdfrac{\vec{G}}{t}}_{\text{Eulerian rate of change}} + \underbrace{\fluidvel\cdot\nabla\vec{G}}_{\text{Convective rate of change}}
\end{align}

In fluid mechanics the Eulerian concept is typically preferred.\\
However, when looking at fluid-structure interaction problems, the interface is typically moving. To account for that motion in the fluid, one either has to come up with a cell-intersection approach, or the fluid-mesh has to be deformed too.\\
An appealing and straightforward way to do so, is the so called \acf{ALE} approach. As the name suggest, it is a hybrid between Eulerian and Lagrangian point of view. The \ac{ALE} approach allows to utilize the best of both approaches. The mesh can either be fixed, moved with the continuum in an Eulerian manner, or be moved in some arbitrarily specified manner in between. The \ac{ALE} method can handle larger distortions than a pure Eulerian would, while still allowing to keep interface continuity between the structure and the fluid, such that no intersection of fluid cells is necessary.\\
In this thesis, both Eulerian and \ac{ALE} methods will be covered. In general, no recommendation on to prefer one or the other method can be given. Instead, one should always be aware of the pros and cons of either method and choose the appropriate one for a specific problem accordingly.\\
For a more in depth-analysis, the interested reader is referred to~\cite{LeVeque1992}


\section{Basic equations of fluid mechanics}\label{sec:fluid_equations_basic}

The physics of fluid motion are governed by the so-called \acf{NSE}. These equations were derived independently of one another by Claude-Louis Navier and George Gabriel Stokes as a generalization of the Euler-equations by including viscosity effects.
In general one distinguishes the incompressible and the compressible \ac{NSE}. The validity of their application depends on the problem setup. As a rule of thumb, flows with Mach-numbers lower than $0.3$ can usually be safely approximated by the incompressible \ac{NSE}.\\
The Navier stokes equation consist of the \acf{NSME}, the \ac{NSCE} and, if the compressible equations are solved, the \ac{NSEE}. In the literature, the term \ac{NSE} is often synonymously used for the \ac{NSCE}, in this thesis however, with \ac{NSE} we will always refer to the full set of equations and will explicitly name the momentum, continuity or energy equation otherwise.

\subsection{Incompressible \acl{NSE}}
The incompressible \acf{NSE} are derived by putting a few assumptions on the Cauchy stress tensor:
\begin{itemize}
\item Galilean invariance of the fluid stress tensor
\item Isotropy of the fluid
\item The stokes stress constitutive equation applies: $\fluidstress=2 \viscosdyn \fluidstrain$ with $\fluidstrain=\frac{1}{2}(\nabla \fluidvel+\T{(\nabla \fluidvel)})$
\end{itemize}

In convective form, the incompressible \ac{NSME} can be written as
\begin{align}\label{eq:nsg_incompressible}
\overbrace{
           \underbrace{\pdfrac{\fluidvel}{t}}_{\text{Variation}}  +
          \underbrace{(\fluidvel\cdot\nabla)\fluidvel}_{\text{Convection}}
          }^{\text{Inertia per volume}}
 -
\overbrace{
           \underbrace{\viscoskin\nabla^2\fluidvel}_{\text{Diffusion}} =
          -\underbrace{\nabla\specificwork}_{\text{Internal source}}
          }^{\text{Divergence of stress}}
+
\underbrace{g}_{\text{External source}}
\end{align}
where $\specificwork$ is the specific thermodynamic work, which satisfies
\begin{align}
\nabla\specificwork =
\frac{1}{\rho}\nabla\pres
\end{align}
And the incompressible \ac{NSCE} can be written as
\begin{align}
\nabla\cdot\fluidvel=0
\end{align}

It shall be noted, that in the incompressible \ac{NSE}, the pressure can only be solved up to a certain constant. The pressure at a certain point therefore has to be fixed, or the system will be singular.


\subsection{Compressible \acl{NSE}}\label{sec:nsg_compressible}
Compared to the incompressible \acf{NSE}, the compressible \ac{NSE} introduces an additional unknown, since the density can now vary. To close the system, one additional equation is required: the energy equation.\\
The compressible \ac{NSE} are derived by making the following assumptions:
\begin{itemize}
\item Galilean invariance of the stress tensor
\item Linearity of the stress in the velocity gradient: $\fluidstress(\nabla\fluidvel)=\tensor{C}:(\nabla\fluidvel)$, where $\tensor{C}$ is called the viscosity or elasticity tensor.
\item Isotopy of the fluid
\end{itemize}

The compressible \ac{NSME} can be written as
\begin{align}\label{eq:nsg_compressible}
\pdfrac{\fluidvel}{t} +
\fluidvel\cdot\nabla\fluidvel =
-\pdfrac{1}{\rho}\nabla\pres +
\viscoskin\nabla^2\fluidvel +
\frac{1}{3} \viscoskin\nabla(\nabla\cdot\fluidvel) +
g
\end{align}
where it shall be noted that for the special case of an incompressible flow, the volume of the fluid elements is constant, resulting in a solenoidal velocity field. Thus $\nabla\cdot\fluidvel=\vec{0}$ and $\nabla\pres=0$, which gives equation \ref{eq:nsg_incompressible}.
\\
The \ac{NSCE} now takes into accounts for the variable density
\begin{align}
\pdfrac{\dens}{t}=\nabla\cdot(\rho\fluidvel)=0
\end{align}

And finally the \ac{NSEE} writes
\begin{align}
\pdfrac{s}{t}=-\fluidvel\cdot s + \frac{Q}{T}
\end{align}
where $s$ is the entropy per unit mass, $Q$ is the heat transferred, and $T$ is the temperature.
F

\subsection{Euler equations}\label{sec:euler_convective}
The Euler equations can be regarded as a special case of the Navies stokes equations where the diffusive(viscous) contributions are neglected. Dimension analysis \cite{Doering1995} reveals that this is usually justified for high Mach numbers and distant from walls. In fact, boundary Layer theory suggests that a lot of problems can be well approximated by solving the full, diffusive equations only near the wall boundary and using the Euler equations everywhere else.
Although the Euler equations can be derived for both incompressible and compressible flows, the statement above suggests that only the compressible ones give actual, physically correct results. We therefore restrict our considerations to the compressible Euler equations.\\
Summing up, these equations can be written in convective form as
\begin{align}
\begin{cases}
\pdfrac{\dens}{t}+\fluidvel\cdot\nabla\dens+\rho\nabla\cdot\fluidvel = 0           &\text{ Continuity Equation} \\[0.5em]
\pdfrac{\fluidvel}{t}+\fluidvel\cdot\nabla\fluidvel+\frac{\pres}{\dens} = \vec{g}  &\text{ Momentum Equation} \\[0.5em]
\pdfrac{e}{t}+\fluidvel\cdot\nabla e + \frac{\pres}{\dens}\nabla\cdot\fluidvel = 0 &\text{ Energy Equation} \\[0.5em]
\end{cases}
\end{align}







\subsection{Conservative form of a \acl{PDE}}\label{sec:conservative_form_nsg}
For the purpose of \ac{FV} discretization one usually brings the equations into the so-called conservative form, which can be written generally as
\begin{align}\label{eq:conservative_form_general}
\tfrac{\vec{\xi}}{t}+\nabla\cdot f(\vec{\xi}) = \vec{0}
\end{align}
where $\vec{\xi}$ is the conserved quantity of interest and $f(\vec{\xi})$ is denoted as "flux function".\\
This equation can then be transformed into an integral form using the divergence theorem
\begin{align}\label{eq:conservation_form_integral}
\tfrac{~}{t}\int_{V} \vec{\xi} dV + \int_{\partial V} f(\vec{\xi})\cdot\normal~dS
\end{align}
This equations states that the rate of change of the integral of the quantity $\vec{\xi}$ over an arbitrary control volume $V$ is equal to the negative of the "flux" through the boundary of the control volume $\pd V$.\\
A simple choice for $f$ would be $f(\xi)=\xi\fluidvel$, which means that the quantity $\vec{\xi}$ follows the fluid field and gives the so-called "advection equation".\\
Although one form is derived from the other, these two are not equivalent. Particularly, it is possible to find solutions to the integral equations that are non-differentiable and therefore not a solutions to the differential form. This leads to so called "weak solutions" and is the cause for many numerical difficulties in the finite volume simulation of such problems.
 \\
 \\
This thesis focuses on the compressible \ac{NSE} only. They can be brought to conservative form as:

\begin{align}\label{eq:nsg_conservative_form}
\pdfrac{\fstate}{t} + \nabla \cdot \fluxesconv(\fstate) + \nabla \cdot \fluxesdiff(\fstate) = \vec{0}
\end{align}
where, comparing to the general conservative form, one can see that $\vec{\xi}=\fstate$ and $f(\vec{\xi})=\fluxesconv(\fstate) +\fluxesdiff(\fstate)$
 \\
In the conservative form of the \ac{NSE} the so-called "fluid state vector" $\fstate$ is defined as
\begin{align}\label{eq:fstate_definition}
\fstate=
\begin{bmatrix}
  \dens            \\
  \dens \fluidvel  \\
  \energytot          \\
\end{bmatrix}
\end{align}
with
\begin{align}\label{eq:fluidvel_definition}
\fluidvel=
\begin{bmatrix}
  \fluidvelx  \\
  \fluidvely  \\
  \fluidvelz  \\
\end{bmatrix}
\end{align}
denoting the fluid velocity vector.
 \\
 \\
The total energy $\energytot$ can be computed as a sum of the so called internal energy and the kinematic energy due to fluid velocity as
\begin{align}\label{eq:energytot}
\underbrace{\energytot}_{\text{total energy}} = \underbrace{\rho \energyint}_{\text{internal energy}} + \underbrace{\frac{1}{2} \T{\fluidvel} \fluidvel}_{\text{kinematic energy}}
\end{align}


Bringing the convective form of the compressible \ac{NSE}~\eqref{eq:nsg_compressible} into the conservative form~\eqref{eq:nsg_conservative_form}, one can derive the convective fluxes $\fluxesconv$ as
\begin{align}\label{eq:fluxesconv}
\fluxesconv
%&=
%(\frac{1}{\dens} \fstate \T{\fstate} + \pres + \tensor{R}_3 \T{\fstate})
%\begin{bmatrix}
%\vec{0} \\ \eye_{3} \\ \vec{0}
%\end{bmatrix} \\ %TODO check if this one is correct
&=
\fstate \T{\fluidvel} + \pres
\begin{bmatrix}
0\\ \eye \\ \T{\fluidvel}
\end{bmatrix} \\
  &=
	\begin{bmatrix}
	  \dens \T{\fluidvel} \\
	  \dens(\fluidvel \T{\fluidvel} ) + \pres \eye \\
	  (\energytot+\pres) \T{\fluidvel}
	\end{bmatrix} \\
	  &=
	  \begin{bmatrix}
	  \dens \fluidvelx             &  \dens\fluidvely              &  \dens\fluidvelz             \\
	  \pres+\dens\fluidvelx^{2}    &  \dens\fluidvelx\fluidvely    &  \dens\fluidvelx\fluidvelz   \\
	  \dens\fluidvely\fluidvelx    &  \pres+\dens\fluidvely^{2}    &  \dens\fluidvely\fluidvelz   \\
	  \dens\fluidvelz\fluidvelx    &  \dens\fluidvelz\fluidvely    &  \pres+\dens\fluidvelz^{2}   \\
	  \fluidvelx(\energytot+\pres) &  \fluidvelx(\energytot+\pres) & \fluidvelx(\energytot+\pres) \\
	  \end{bmatrix}
\end{align}
depending on the algorithm, one or another form can be advantageous.


The diffusive fluxes can be written as
\begin{align}\label{eq:fluxes_diff}
\fluxesdiff&=
\begin{bmatrix}
\vec{0}  \\
\fluidstress \\
\fluidstress \fluidvel + \heatflux
\end{bmatrix} \\
  &=
  \T{\begin{bmatrix}
  0                    &  0                 &  0          \\
  -\fluidstresscomp_{xx}  &  -\fluidstresscomp_{yx}  &  -\fluidstresscomp_{zx} \\
  -\fluidstresscomp_{xy}  &  -\fluidstresscomp_{yy}  &  -\fluidstresscomp_{zy} \\
  -\fluidstresscomp_{xz}  &  -\fluidstresscomp_{yz}  &  -\fluidstresscomp_{zz}  \\
  -\heatfluxcomp_x-\fluidvelx\fluidstresscomp_{xx}-\fluidvelx\fluidstresscomp_{yx}-\fluidvelx\fluidstresscomp_{zx} &
  -\heatfluxcomp_y-\fluidvelx\fluidstresscomp_{xy}-\fluidvelx\fluidstresscomp_{yy}-\fluidvelx\fluidstresscomp_{zy} &
  -\heatfluxcomp_z-\fluidvelx\fluidstresscomp_{xz}-\fluidvelx\fluidstresscomp_{yz}-\fluidvelx\fluidstresscomp_{zz} \\
  \end{bmatrix}}
\end{align}

The definition of the stress tensor depends of the fluid model used. For the simple case of a Newtonian fluid, it can be written as
\begin{align}\label{eq:fluidstress}
\fluidstress = \viscosdyn(\nabla\fluidvel+\T{(\nabla\fluidvel)} )+\lambda(\nabla\cdot\fluidvel)\eye
\end{align}

For the relation between Temperature and heat flux an assumption has to be made. Most often, a simple Fourier's law is assumed
\begin{align}\label{eq:heatflux}
\heatflux=-\thermcond\nabla\temp
\end{align}

\subsection{Conservative form of the Euler equations}
The Euler equations are a simplified form of the Navier-Stokes Equations~\eqref{eq:nsg_incompressible}\eqref{eq:nsg_compressible}, where the viscous effects are neglected by setting $\fluxesdiff=\vec{0}$.
\begin{align}\label{eq:euler}
\pdfrac{\fstate}{t} + \nabla \cdot \fluxesconv(\fstate) = \vec{0}
\end{align}
The Euler equations are appropriate for a wide range of applications. A typical indicator is the Reynolds number, which describes the ratio between inertial forces and viscous forces:
\begin{align}\label{eq:reynolds}
\reynolds=\frac{\dens\fluidvel L}{\viscosdyn}=\frac{\fluidvel L}{\viscoskin}
\end{align}
where $L$ describes a characteristic length.

A high Reynolds number thus indicates that the flow is dominated by inertial forces, thus the Euler Equations should give satisfying results. However, an Euler flow lacks the ability to represent stick wall boundary conditions, thus it is unable to represent boundary layers.

\subsection{Equations of state}
Looking at the above Equations, one might notice that the number of unknowns is greater than the number of equations. Particularly, the pressure only appears in Equation~\eqref{eq:fluxesconv} and is not linked to the other formulas. This problem is solved by introducing an \acf{EOS} that relates pressure, internal energy and density. The \ac{EOS} depends on the fluid model, some well-known ones are: \ac{PG}, \ac{SG} or \ac{JWL}.\\
For the simplest one, PG, the \ac{EOS} can be written as
\begin{align}\label{eq:eos_pg}
\pres=(\specheatratio-1)\dens\energyint
\end{align}


\subsection{\acf{RANS} Equations}\label{sec:rans}
The \acf{RANS} Equations are time-averaged equations of motion for the fluid.
\begin{align}\label{eq:rans_averaging}
\fstate \rightarrow \av{\fstate} = \lim_{\temp \to \infty} \frac{1}{T} \int_{t^0}^{t^0+T} \fstate dt
\end{align}

The main idea of the approach is to decompose an instantaneous quantity into time-averaged and fluctuating components
\begin{align}\label{eq:rans_decomposition}
\fstate=\underbrace{\av{\fstate}}_{time-average}+\underbrace{\fluc{\fstate}}_{fluctuation}
\end{align}

When substituting this decomposition back into the \ac{NSE}(and injecting several other approximations), a closure problem induced by the arising non-linear Reynolds stress term ${\reynolds}_{ij}=-\av{\fluc{v_i}\fluc{v_j}}$ arises. Additional modeling is therefore required to close the \ac{RANS} equations, which has led to many different turbulence models.\\
\\
Whatever turbulence model is chosen, the fluid-state vector is augmented by the $m$ parameters of the turbulence model
\begin{align}
\fstaterans \leftarrow
\begin{bmatrix}
\fstate            \\
\turbulenceparam_1 \\
\vdots             \\
\turbulenceparam_m
\end{bmatrix} =
	\begin{bmatrix}
	\dens              \\
	\dens\T{\fluidvel} \\
	\energytot         \\
	\turbulenceparam_1 \\
	\vdots             \\
	\turbulenceparam_m
	\end{bmatrix}
\end{align}

The \ac{RANS} equations can then be written as
\begin{align}\label{eq:rans_eqautions}
\pdfrac{\av{\fstate}}{t} + \nabla\cdot\fluxesconv(\av{\fstate}) +  \nabla\cdot\fluxesdiff(\av{\fstate}) =
\turbulencesource(\av{\fstate},\turbulenceparam_1,\cdots,\turbulenceparam_m)
\end{align}

\paragraph{Spalart Allmaras turbulence model}

As explained in the previous chapter, the \ac{RANS} equations necessitate additional modeling to close the system of equations. This is typically performed by applying the Boussinesq assumption and assuming that the fluid stress tensor can be written as
\begin{align}
\fluidstress_{ij}&=2 \nut \left( \bar{S}_{ij}-\frac{1}{3} \pdfrac{\bar{v}_k}{x_k} \delta_{ij} \right) - \frac{2}{3}\bar{\dens}k\delta_{ij} \label{eq:rans_turbulent_stress}\\
&\bar{S}_{ij}=\frac{1}{2}\left( \pdfrac{\bar{v}_i}{x_j}+\pdfrac{\bar{v}_j}{x_i} \right) \\
&k=\frac{1}{2} \left( \fluc{v_i}\fluc{v_i} \right)
\end{align}

where $\nut$ denotes the turbulent viscosity and $k$ is the mean turbulent kinetic energy. Both $\nut$ and $k$ need to be modeled, thus an additional two equations are required to close the system. A variety of different turbulence models have been proposed in the literature, this thesis however, focuses on the very particular case of the Spalart-Allmaras model, introduced in \cite{Spalart1994}.\\
Herein, the last term of equation~\eqref{eq:rans_turbulent_stress} is being neglected, thus a single additional equation is sufficient to close the system.
Specifically, Spalart-Allmaras introduces the single, scalar, viscosity-like turbulent variable
\begin{align}
\nu^{\star}=\frac{\mu_t}{\bar{\dens} f_{v1}}
\end{align}
and the corresponding transport equation
\begin{align}
\pdfrac{~}{t}\left( \bar{\dens} \nu^{\star}\right) + \pdfrac{~}{x_j}\left( \bar{\dens}\bar{v}_j \nu^{\star} \right) = 
&\frac{1}{\sigma}\left[ \pdfrac{~}{x_j} \left( \bar{\dens}(\nu+\nu^{\star})\pdfrac{\nu^{\star}}{x_j} \right) + c_{b2} \bar{\dens}(\pdfrac{\nu^{\star}}{x_j}\pdfrac{\nu^{\star}}{x_j})\right]+\nonumber\\
&c_{b1}(1-f_{t2}) S^{\star} \bar{\dens}\nu^{\star}-\bar{\dens}(c_{w1}f_{w}-\frac{c_{b1}}{\kappa^2} f_{t2})
\left(\frac{\nu^{\star}}{d}\right)^{2}
\end{align}
with
\begin{align}
f_{v1}=\frac{\turbulenceparam^3}{\turbulenceparam^3+c_{v1}^3},~~~\turbulenceparam=\frac{\nu^{\star}}{\nu},~~~S^{\star}=\omega+\frac{\nu^{\star}}{k^2d^2}f_{v2}
\end{align}
where $\omega$ is the vorticity magnitude and $d$ is the nearest distance to the wall.


\end{document}