\documentclass[../main.tex]{subfiles}
%\usepackage{algorithm}
%\usepackage{algorithmic}
%\everymath{\displaystyle}
\def\arraystretch{2.0}
\begin{document}
\setlength{\delimitershortfall}{0pt}

\def\incrabsvars{\Delta \absvars}
\def\incrlagmultsneq{\Delta \lagmultsneq}
\def\incrlagmultseq{\Delta \lagmultseq}

\def\PPLagfuncBYabsvars{\ppdfrac{\Lagfunc}{\absvars}}
\def\PLagfuncBYabsvars{\pdfrac{\Lagfunc}{\absvars}}
\def\PneqctrBYabsvars{\pdfrac{\neqctr}{\absvar}}
\def\PeqctrBYabsvars{\pdfrac{\eqctr}{\absvar}}

\def\DoptcritJBYabsvarI{\frac{d \optcrit_{j}}{d \absvar_{i}}}
\def\PoptcritJBYabsvarI{\pdfrac{\optcrit_{j}}{\absvar_{i}}}

\def\PoptcritJBYstructdisp{\pdfrac{\optcrit_{j}}{\structdisp}}
\def\PoptcritJBYmms       {\pdfrac{\optcrit_{j}}{\mms}}
\def\PoptcritJBYfstate    {\pdfrac{\optcrit_{j}}{\fstate}}

\def\DoptcritJBYstructdisp{\tfrac{\optcrit_{j}}{\structdisp}}
\def\DoptcritJBYmms       {\tfrac{\optcrit_{j}}{\mms}}
\def\DoptcritJBYfstate    {\tfrac{\optcrit_{j}}{\fstate}}

\def\DstructdispBYabsvarI{\frac{d \structdisp}{d \absvar_{i}} }
\def\DmmsBYabsvarI       {\frac{d \mms       }{d \absvar_{i}} }
\def\DfstateBYabsvarI    {\frac{d \fstate    }{d \absvar_{i}} }

\def\PstructdispBYabsvarI{\pdfrac{d \structdisp}{d \absvar_{i}} }
\def\PmmsBYabsvarI       {\pdfrac{d \mms       }{d \absvar_{i}} }
\def\PfstateBYabsvarI    {\pdfrac{d \fstate    }{d \absvar_{i}} }

\def\PEOSstructBYabsvarI{\pdfrac{\EOSstruct} {\absvar_i}}
\def\PEOSmeshBYabsvarI  {\pdfrac{\EOSmesh}  {\absvar_i}}
   %\def\PEOSfluidBYabsvarI{\pdfrac{\EOSfluid}{\absvar_i}}

\def\DEOSstructBYabsvarI{\tfrac{\EOSstruct} {\absvar_i}}
\def\DEOSmeshBYabsvarI  {\tfrac{\EOSmesh}  {\absvar_i}}
  %\def\DEOSfluidBYabsvarI{\tfrac{\EOSfluid}{\absvar_i}}

\def\PEOSstructBYstructdisp{\pdfrac{\EOSstruct} {\structdisp}}
\def\PEOSstructBYmms{\pdfrac{\EOSstruct}{\mms}}
\def\PEOSstructBYfstate    {\pdfrac{\EOSstruct}{\fstate}}

\def\PEOSmeshBYstructdisp{\pdfrac{\EOSmesh}{\structdisp}}
\def\PEOSmeshBYmms       {\pdfrac{\EOSmesh} {\mms}}
\def\PEOSmeshBYfstate    {\vec{0}}

\def\PEOSfluidBYstructdisp{\vec{0}}
   %\def\PEOSfluidBYmms       {\pdfrac{\EOSfluid} {\mms}}
   %\def\PEOSfluidBYfstate    {\pdfrac{\EOSfluid} {\fstate}}

\def\PstructdispBYabsvarI{\pdfrac{\structdisp} {\absvar_i}}
\def\PmmsBYabsvarI       {\pdfrac{\mms}        {\absvar_i}}
\def\PfstateBYabsvarI    {\pdfrac{\fstate}     {\absvar_i}}



\chapter{Fluid \acf{SA}}\label{sec:SA}
\addcontentsline{lof}{chapter}{Fluid \acf{SA}}
\minitoc
%\sectlof
%\sectlot

As explained in chapter \ref{sec:optimization}, optimization is based on the calculations of gradients, also denoted as sensitivities.
The derivation and implementation of these gradients constituted the lion-share of this thesis. We will therefore denote this chapter to the general derivation of fluid-sensitivity equations, before a detailed look into the most important terms within these equations is provided in chapter~\ref{sec:dresidual_derivative}.

\section{General concepts}
When it comes to calculating the gradients of a solution of a \ac{PDE}, there are two fundamentally different approaches one could take.\\
Firstly, one could think of first deriving the continuous equations and then applying the discretization scheme on them. On the other hand its equally legitimate to first apply the discretization and compute the gradients of this approximated solution. This thesis focuses on the latter for very particular reasons, see figure~\ref{fig:discretization_differentiation_sequence}.
If one would take the first approach, getting a steady state solution for the subsequent \acf{SA} would entail the solution of the derived system of equations. This can not be done with standard CFD software. In addition to the \ac{SA} routines, one would therefor have to implement an additional steady equation solver. This is not only uneconomical, it increases the risk of creating implementation bugs and also defeats the purpose of decades of research in stable and efficient \ac{NSE} solvers.


 \begin{figure}[h!]
	\begin{center}
        \includegraphics[scale=0.85]{\mainpath/fig/tikz/build/discretization_differentiation_sequence.pdf}
        \caption[Sensitivity Analysis approaches]{Two different approaches of \ac{SA}. The question here is whether to discretize the system of equations first and then compute the derivatives of the approximate solution, or whether the continuous system of equation is first derived and discretized afterwards. Both approaches are valid, the final result however will generally not be the same. We have solemnly focused on the case of derivation after discretization in this thesis(thick lines). There are several reason to this. Firstly, if one chose to first derive and then compute a discretized solution, one would solve a completely different equation. If on the other hand the discretization(e.g. Finite Volumes) is done first, standard solvers can be utilized. What is more, approximating the derivative by a Finite difference becomes much easier, since the finite difference evaluation involves only a evaluation of $\dmat{F}$, which a standard fluid solver is perfectly capable of, whereas the \ac{FD} approximation in the other case would involve evaluations of the continuous function and the discretization of the obtained expression which would be much more cumbersome.}
		\label{fig:discretization_differentiation_sequence}
    \end{center}
\end{figure}





\section{Sensitivity derivation}\label{sec:sensitivity_derivation}
When it comes to \ac{SA} within the context of \ac{CFD}, one also has to distinguish between \ac{SA} of the non-coupled fluid problem and the \ac{SA} in the fully coupled aeroelastic case. The latter case is much more involved, since it requires additional terms from the Structure and mesh motion equations as well as for the coupling itself. In the following we will therefore begin with the \ac{SA} of a regular fluid problem. A generalization to a coupled FSI problem will follow.
 \\


As mentioned in Section~\ref{sec:optimization}, we typically deal with an optimization criteria $\optcrit_j$, that is in this case dependent on the fluid state variables $\fstate$ that themselves may depend on abstract variables $\absvar_i$
\begin{align}
\optcrit_j=\optcrit_j(\fstate(\absvar_i))
\end{align}



\begin{align}\label{eq:sensitivity_startingpoint}
\DoptcritJBYabsvarI\bigg\rvert_{\fstate_0}=\setulcolor{primary_0}\setul{}{2pt}
\underbrace{\PoptcritJBYabsvarI\bigg\rvert_{\fstate_0}}_{
                                                        \substack{
                                                                 \text{\ul{directly derived from}} \\
                                                                 \text{\ul{the definition of} $\optcrit$}
                                                                 } 
                                                        }  +
\underbrace{\PoptcritJBYfstate\bigg\rvert_{\fstate_0}}_ {\setulcolor{neutral_1}
                                                        \substack{
                                                                 \text{\ul{derived analytically or~~}}\\
                                                                 \text{\ul{by} \ac{FD}}
                                                                 }
                                                        }
\underbrace{\DfstateBYabsvarI\bigg\rvert_{\fstate_0}}_  {\setulcolor{neutral_2}
                                                        \substack{
                                                                 \text{\ul{derived from dynamic}}\\
                                                                 \text{\ul{fluid equilibrium}}
                                                                 }
                                                        }
\end{align}
\subsection{Calculation of $\PoptcritJBYabsvarI\bigg\rvert_{\fstate_0}$}
For the optimization criterion, we restrict ourselves to Lift($L$), Drag($D$) and combinations of  these quantities( e.g. Lift-Drag ration $\frac{L}{D}$ ). As abstract variables we only consider the free stream Mach-number $\machnum_{\infty}$, the free-stream angle of attack $\aoa$ and an airfoil shape parameter as explained in chapter \ref{sec:design_model}.
The formula for the lift over an airfoil can be written as


If $\Gamma_{FS}$ denotes the fluid structure interface, which in the \ac{ALE} context coincides with the airfoil surface, one can formulate the lift and drag of an airfoil in a steady state as:
\begin{align}\label{eq:lift_and_drag_integrals}
\lift=\int_{\Gamma_{FS}} p(\fstate,\cvec{x}) \normal(\cvec{x})\cdot \cvec{e}_L dS \nonumber\\
\drag=\int_{\Gamma_{FS}} p(\fstate,\cvec{x}) \normal(\cvec{x})\cdot \cvec{e}_D dS \\
\end{align}
where $\cvec{e}_D$ is the unit vector pointing in the direction of the free stream, and $\cvec{e}_L$ is the unit vector perpendicular to that.
We therefor note, that $\pdfrac{\lift}{\absvar_i}$ and $\pdfrac{\drag}{\absvar_i}$ are zero for $\absvar_{i}=\machnum_{\infty}$ and $\absvar_{i}=\aoa$ and non-zero if $\absvar_{i}$ is a shape parameter.\\
Also, having determined the derivatives $\pdfrac{\lift}{\absvar_i}$ and $\pdfrac{\drag}{\absvar_i}$, the derivative of the Lift-Drag ratio follows simply as
\begin{align}\label{eq:lifttoddrag_by_absvar}
\mathunderline[primary_0]{-\pdfrac{(\frac{\lift}{\drag})}{\absvar_i}\rvert_{\fstate_0}}=
-\frac{\drag\rvert_{\fstate_0} \pdfrac{\lift}{\absvar_i}\rvert_{\fstate_0} - \lift\rvert_{\fstate_0} \pdfrac{\drag}{\absvar_i}\rvert_{\fstate_0} }{{\drag\rvert_{\fstate_0}}^2}
\end{align}


\subsection{Calculation of $\PoptcritJBYfstate\bigg\rvert_{\fstate_0}$}
The derivative of the optimization criteria with respect to the fluid state vector can again be splitted into Lift and Drag part.
For an aeroelastic simulation, the biggest difficulty here is the dependence of the fluid state by the fluid-structure interface itself.
We are therefore faced with a derivative of an integral quantity, where the integration area itself is dependent on the quantity of interest.\\
For this purpose, we recall the Leibniz rule from calculus:
\begin{align}\label{eq:leibnitz_rule}
\frac{d}{dx}\Big(\int_{a(x)}^{b(x)} f(x,t) dt \Big) =
f(x,b(x))\cdot \frac{d}{dx}b(x)-
f(x,a(x))\cdot \frac{d}{dx}a(x)+
\int_{a(x)}^{b(x)} \frac{d}{dx} f(x,t) dt
\end{align}

Alternatively, we can directly switch to the discrete form, where the calculation of Lift and drag as outlined in Equation~\eqref{eq:lift_and_drag_integrals} can be performed as
\begin{align}\label{eq:discrete_lift_and_drag_formulas}
L=\sum_{e \in \sum_{\Gamma_{FS}}} \sum_{i=1}^{n_g} g_i \dpres(\dfstate_i,\dmpos_i) \normal_e \cdot e_L
D=\sum_{e \in \sum_{\Gamma_{FS}}} \sum_{i=1}^{n_g} g_i \dpres(\dfstate_i,\dmpos_i) \normal_e \cdot e_D
\end{align}
where $\sum_{\Gamma_{FS}}$ is the set of surface elements on the fluid-structure interface, $n_g$ is the number of gauss-points and $g_i$ are the gauss weights. The dependency of the surface shape, on the fluid state vector is now reflected in non-zero derivatives of the gauss point locations.

Finally, the derivative of the lift-to-drag ratio can be obtained analogously to equation~\eqref{eq:lifttoddrag_by_absvar} as
\begin{align}\label{eq:lifttodrag_by_dfstate}
\mathunderline[neutral_1]{\pdfrac{(\frac{\mathsf{L}}{\mathsf{D}})}{\dfstate}\rvert_{\fstate_0}}=
\frac{\mathsf{D}\rvert_{\fstate_0} \pdfrac{\mathsf{L}}{\dfstate}\rvert_{\fstate_0} - \mathsf{L}\rvert_{\fstate_0} \pdfrac{\drag}{\dfstate}\rvert_{\dfstate_0} }{{\mathsf{D}\rvert_{\fstate_0}}^2}
\end{align}


\subsection{Calculation of $\DfstateBYabsvarI\bigg\rvert_{\fstate_0}$}

Also, we keep in mind, that the state equation of the fluid can be expressed as
\begin{align}
\EOSfluid(\fstate(\absvar_i),\mpos(\absvar_i),\absvar_i)=
\pdfrac{\fstate(\absvar_i)}{t}+\nabla\cdot\fluxesconv(\fstate(\absvar_i))+\nabla\cdot\fluxesdiff(\fstate(\absvar_i))+S=
\vec{0}
\end{align}

After discretization, we have derived our discrete governing equation as 
\begin{align}\label{eq:nse_final_discretized_2}
\pdfrac{\bar{\dfstate_i}}{t} +
\sum_{j \in \vertexset(i)} \fluxesnum_{ij}(\dfstate_{ij},\dfstate_{ji},\wnormal_{ij}) -
\sum_{T_i \in \elementset(i)} \int_{T_j} \difftensor \nabla \dfstate \nabla \phi_j dx =
\vec{0}
\end{align}
For the purpose of Sensitivity analysis around a given steady state, the fluid-state vector does not change in time, and can thus be omitted, also we summarize the convective and the diffusive part as $\dresidual$ 
\begin{align}\label{eq:nse_final_discretized_notationchange}
\cancelto{0}{\pdfrac{\bar{\dfstate_i}}{t}} +
\underbrace{
  \underbrace{\sum_{j \in \vertexset(i)} \fluxesnum_{ij}(\dfstate_{ij},\dfstate_{ji},\wnormal_{ij})}_{\dresidual^i} -
  \underbrace{\sum_{T_i \in \elementset(i)} \int_{T_j} \difftensor \nabla \dfstate \nabla \phi_j dx}_{\dresidual^v}      }_{\dresidual} =
\vec{0}
\end{align}
where $\dresidual^i$ denotes the inviscid residual and $\dresidual^v$ the viscous contribution.
We now can express the steady state equation in very compact form as:
\begin{align}\label{eq:discrete_steady_state}
\dresidual=\dvec{0}
\end{align}


In general, this residual depends on the fluid state solution $\dfstate$ as well as on a potential mesh movement $\dmpos$, both of which can be dependent on the abstract design variable. Additionally, a direct dependence on $\absvar$ might be encountered too.
\begin{align}
\dresidual(\dfstate(\absvar),\dmpos(\absvar),\absvar)=\dvec{0}
\end{align}

The total derivative of the residual with respect to the design variable can thus be expanded via the chain rule to
%\begin{align}
%\DEOSfluidBYabsvarI=\vec{0}=
%\PEOSfluidBYabsvarI+
%\PEOSfluidBYfstate\DfstateBYabsvarI+
%\PEOSfluidBYmms\DmmsBYabsvarI
%\end{align}

\def\DdresidualBYabsvarI{ \tfrac{\dresidual}{\absvar_i} }
\def\PdresidualBYabsvarI{ \pdfrac{\dresidual}{\absvar_i} }
\def\PdresidualBYdfstate{ \pdfrac{\dresidual}{\dfstate}  }
\def\DdfstateBYabsvarI  { \tfrac{\dfstate}{\absvar_i}   }
\def\PdresidualBYdmms   { \pdfrac{\dresidual}{\dmpos}     }
\def\DdmmsBYabsvarI     { \tfrac{\dmpos}{\absvar_i}      }
\begin{align}
\DdresidualBYabsvarI=\vec{0}=
\PdresidualBYabsvarI                      +
\PdresidualBYdfstate    \DdfstateBYabsvarI +
\PdresidualBYdmms       \DdmmsBYabsvarI
\end{align}



Therefore the total derivative of the fluid state with respect to the shape variable, needed in equation \eqref{eq:sensitivity_startingpoint}, can be obtained by solving

\begin{align}\label{eq:dfstate_by_absvarI}
\PdresidualBYdfstate    \mathunderline[neutral_2]{\displaystyle\DdfstateBYabsvarI}=
-\DdresidualBYabsvarI 
-\PdresidualBYdmms       \DdmmsBYabsvarI
\end{align}


In this equation, $\PdresidualBYabsvarI$ can be computed analytically or by \ac{FD}.
This part is the most cumbersome part of the whole sensitivity analysis, which is why we denoted a full chapter(\ref{sec:dresidual_derivative}) to it.

The derivative of the mesh motion with respect to the abstract variables is often denoted as "shape gradient". It can be divided into two components:
\begin{itemize}
\item The interface component $\tfrac{\mms_\Gamma}{\absvar_i}$, which is associated with the grid points lying on the fluid boundary
\item The interior component $\tfrac{\mms_\Omega}{\absvar_i}$, which is associated with the grid points located in the interior $\Omega$ of the computational domain.
\end{itemize}
The interface component is determined by the structure. Having obtained this one, the interior component can be computed by solving an auxiliary, fictitious Dirichlet problem:
\begin{align}
\tfrac{\mms_\Omega}{\absvar_i}=-\left[\inv{\fic{\stiffmat}_{\Omega\Omega}} \fic{\stiffmat}_{\Omega\Gamma}\right] \tfrac{\mpos_\Gamma}{\absvar_i}
\end{align}
where $\fic{\stiffmat}$ is a pseudo stiffness matrix that can be obtained by a simple spring analogy or similar approaches.
In general, the mesh-motion algorithm should be chosen consistently with the one used in the calculation of the steady state solution.
For the later introduced Embedded framework, $\tfrac{\mpos_\Gamma}{\absvar_i}$ is the sensitivity of the embedded discrete surface.
 \\
 
\section{Full sensitivity equation}\label{sec:full_sensitivity_equation}
After inserting \eqref{eq:lifttoddrag_by_absvar}, \eqref{eq:lifttodrag_by_dfstate} and \eqref{eq:dfstate_by_absvarI} into \eqref{eq:sensitivity_startingpoint}, one can derive the final sensitivity equations for the special case of a rigid or non-existing structure as
\begin{align}\label{eq:full_sa_nostruct}
\tfrac{\optcrit_j}{\absvar_i}\bigg\rvert_{\dfstate_0} &=
-\tfrac{\optcrit_j}{\fstate}\bigg\rvert_{\dfstate_0}
\inv{\left[\PdresidualBYdfstate\bigg\rvert_{\fstate_0}\right]}
\left(
  \PdresidualBYabsvarI\bigg\rvert_{\fstate_0} +
  \begin{bmatrix}
    \alpha\pdfrac{\dresidual}{\dmms_\Omega}\bigg\rvert_{\dfstate_0}~
    \pdfrac{\dresidual}{\dmms_\Gamma}\bigg\rvert_{\dfstate_0}
  \end{bmatrix}
  \begin{bmatrix}
    \alpha \inv{\fic{\stiffmat}_{\Omega\Omega}} \fic{\stiffmat}_{\Omega\Gamma} \nonumber\\
    \eye
  \end{bmatrix}
  \tfrac{\dmms_\Gamma}{\absvar_i}
\right)\\
\alpha&=
\begin{cases}
  1\text{  in \ac{ALE} framework}\\
  0\text{  in Embedded framework}
\end{cases}
\end{align}
































\section{Direct and adjoint approach}
Looking at the above equation one can identify a matrix-matrix-matrix product:

\begin{align}
\inv{\pdfrac{\dresidual}{\absvar}} \pdfrac{\dresidual}{\dfstate}  \pdfrac{\dfstate}{\absvar}
\end{align}

 Matrix-matrix products are not commutative, one can, however still solve this product in different ways by performin the following transformation
\begin{align}\label{eq:adjoint_concept}
\dmat{A}\dmat{B}\dmat{C}                    &= \dmat{A}(\dmat{B}\dmat{C}) \nonumber\\
\left[ \dmat{A}(\dmat{B}\dmat{C}) \right]^T &= (\dmat{B}\dmat{C})^T\dmat{A}^T=\dmat{C}^T\dmat{B}^T\dmat{A}^T=\dmat{C}^T(\dmat{A}\dmat{B})^T \nonumber\\
\rightarrow \dmat{A}\dmat{B}\dmat{C}        &= \left[\dmat{C}^T(\dmat{A}\dmat{B})\right]^T
\end{align}
Depending on the sizes of the involved matrices one or the other form might be advantageous. Specifically, the straight forward way of computing the triple product is known as the \textit{direct approach} and the transposed matrix way is called the \textit{adjoint approach}.


By standard algebra theory, the total complexity of the direct computation can be approximated as $\order{n_{eq}^2n_{\absvars}+n_{\optcrits}n_{eq}n_{\absvars}}$, whereas the complexity of the adjoint computation is approximated as $\order{n_{eq}^2n_{\optcrits}+n_{\optcrits}n_{eq}n_{\absvars}}$.\\


If one or the other approach is to be preferred depends on the optimization setup, particularly the number of optimization criteria and the number of optimization variables. Looking at the orders above, one can conclude that if the number of abstract parameters $n_{\absvars}$ is smaller than the number of optimization criteria, the direct approach is more efficient, otherwise the adjoint approach is to be preferred. Additionally one can argue, that the relevant term in the orders above is the one with $n_{eq}^2$  since it dominates the sum. Therefore, depending on whether $n_s$ or $n_q$ is bigger, the direct or the adjoint method is to be preferred.



\end{document}
