\documentclass[../main.tex]{subfiles}
%\usepackage{import}


\begin{document}

\setlength{\delimitershortfall}{0pt}
\section{Fiver}\label{sec:fiver}

It was already outlined in Section \ref{sec:fluid_mechanics}, that using an Eulerian approach int the context of an aeroelastic simulation leads to a so-called embedded formulation, where the interface of the structure mesh no longer coincides with the fluid mesh. This was appealing, since it allowed for the usage of fixed meshes and an Eulerian formulation, on the other hand as  explained in Section \ref{sec:fiver_setup}, an error of the order $\order{\frac{h}{2}}$ is introduced. A possible solution to recover second order convergence rate is quickly described in section~\textbf{TODO}.


\subsection{Setup}
In this thesis, we consider embedded structure interfaces only. Whether the structure is deformable or rigid does not really make a difference for the subsequent considerations.\\
The basic setup is depicted in Figure~\ref{fig:FIVER_intersection}.\\
Several issues have to be addresses in an embedded framework. Firstly, the interface may move during the simulation. In fact, large deformations of the structure have been on on the primary aspects for the development of an embedded framework in the first place. The question thus becomes how to track the interface during the simulation. This is addressed in section~\ref{sec:interface_tracking}.\\
Secondly, the evaluation of the inviscid \eqref{eq:fiver_inviscid_split} and viscous \eqref{eq:fiver_viscous_split} term becomes cumbersome for cells that are being intersected. We address this issue separate for the inviscid and the viscous term in sections \ref{sec:fiver_inviscid_term} and \ref{sec:fiver_viscous_term}.\\
Finally, in an \ac{FSI} simulation we are typically interested in integral quantities over the structure surface, e.g. the lift and drag values of an airfoil. We now have several possibilities to define the structure surface to perform integration on in an embedded simulation. This issues will is discussed in section~\ref{sec:fiver_force_evaluation}.

\begin{figure}[h!]
	\begin{center}
        \includegraphics{\mainpath/fig/tikz/build/fiver_surrogate_interface.pdf}
        \caption[FIVER setup]{Sketch of an embedded simulation setup. The primal grid is intersected by a material interface. In \ac{FIVER}, the material interface is replaced by a surrogate interface that is created by connecting the dual-grid interfaces closest to the embedded surface. By connecting the intersection points of the embedded surface with the primal grid, we can also create a so-called, reconstructed surface that is particularly useful for the calculation of forces on the surface.}
		\label{fig:FIVER_intersection}
    \end{center}
\end{figure}


\subsection{Level-set method}\label{sec:interface_tracking}

The issue of interface tracking is extensively discussed in \textbf{CITEWangGretarsson2012} and TODO.\\

Since the interface in an embedded stimulation typically moves, an appropriate interface tracking approach is required.\\
For \ac{FIVER}, the popular level set method \cite{Sethian1999} was chosen.

The level set approach is characterized by the following equation:
\begin{align}\label{eq:level_set}
\frac{\phi}{t}+\fluidvel\cdot\nabla\phi=0
\end{align}

where $\phi$ is a function designed to track the material interface. Particularly, $\phi$ is initialized such that that the interface is characterized by $\phi=0$. The interface can then be tracked via a simple time integration of equation~\eqref{eq:level_set}.
Details of this approach, which was developed for computer visualization purposes, can be found in \cite{Sethian1999}.

\textbf{TODO}




\subsection{Evaluation of the inviscid term at the interface}\label{sec:fiver_inviscid_term}
For this section as well as for section \ref{sec:fiver_viscous_term} we consider Figure~\ref{fig:FIVER_intersection}.\\
A question, that arrises when lookin at equation~\eqref{eq:nse_final_discretized} is how to evaluate those terms for node $i$, where some of the vertices in $\vertexset(i)$ are inactive ghost nodes.\\
We will answer this question for the inviscid term in this subsections and look at the viscous term in the subsequent one.
\vskip 0.5cm
First, we notice that we can split the summation as follows
\begin{align}\label{eq:fiver_inviscid_split}
\sum_{j \in \vertexset(i)} \fluxesnum_{ij}(\fstate_{i},\fstate_{j},\wnormal_{ij}) =
\sum_{j \in \vertexset(i)^a} \fluxesnum_{ij}(\fstate_{i},\fstate_{j},\wnormal_{ij}) +
\sum_{j \in \vertexset(i)\setminus\vertexset(i)^a} \fluxesnum_{ij}(\fstate_{i},\fstate^{*},\wnormal_{ij})
\end{align}
%\begin{align}\label{eq:fiver_inviscid_split}
%\fluxesnum_{ij}
%\end{align}

where $\fstate_{*}$ is the fluid state at the auxiliary intersection between edge $ij$ and the auxiliary interface.
To obtain $\fstate_{8}$ a one-sided Riemann problem is defined
\begin{align}
\pdfrac{\fstateprim}{t}=\pdfrac{\fluxesconv}{s}(\fstateprim) = \vec{0}
\end{align}
where $s$ is the local abscissa along the direction $ij$, that has its origin in $M_{ij}$.\\
The one sided Riemann problem can then be initialized with
\begin{align}
\tilde{\fstate}_{L} =
\begin{bmatrix}
\dens_i &
\fluidvel_i &
\pres_i
\end{bmatrix}
\end{align}

The exact solution of the one-sided, one-dimensional Riemann problem contains a constatn (in time) state at the fluid-structure interface which is denoted here by
\begin{align}
\fstate_{*} =
\begin{bmatrix}
\dens_* &
\fluidvel_* &
\pres_*
\end{bmatrix}
\end{align}
which can then be used in equation~\eqref{eq:fiver_inviscid_split}





\subsection{Evaluation of the viscous terms at the interface}\label{sec:fiver_viscous_term}

If one wants to keep the \ac{FE}-like evaluation of the second term,~\eqref{eq:nse_final_discretized} can be splitted as
\begin{align}\label{eq:fiver_viscous_split}
\sum_{T_i \in \elementset(i)} \int_{T_i} \difftensor \nabla \fstate \nabla \phi_i dx=
\sum_{T_i \in \left(\elementset(i^a)\right)} \int_{T_i} \difftensor \nabla \fstate \nabla \phi_i dx+
\sum_{T_i \in \left(\elementset(i)\setminus\elementset(i^a)\right)} \int_{T_i} \difftensor \nabla \fstate^R \nabla \phi_i dx=
\end{align}

where $\elementset(i^a)$ is the set of triangles that can be build from the active nodes around node $i$ and $\elementset(i)\setminus\elementset(i^a)$ are all the triangles where at least one node is inactive(ghost).\\
\\
The only difficulty now becomes the evaluation of the last term, meaning the integration over cutted elements, where at least on node is inactive, and thus does not have a fluid state solution.\\
We did not have that problem for the inviscid term of equation \eqref{eq:nse_final_discretized}, since a \ac{FV} approximation was chosen, and therefore we could construct a flux at the interface thanks to the piston problem. No inactive node had to be considered.\\
For the viscous part, however, we want to keep the \ac{FE} like formulations described in \ref{sec:mixed_FV_FE_formulation} to avoid re-writing large portions of the code. If a triangle is cut by the interface, one or two nodes will therefor be labeled inactive, and be denoted as ghost-nodes.
Clearly, the fluid state vector at this ghost nodes is not defined. The question thus becomes how to evaluate this terms (last part in equation~\eqref{eq:fiver_viscous_split} ).\\
\\
The approach we take, that is also outlined in \textbf{TODO ask for reference} is to reconstruct a pseudo fluid-state at the ghost points, such that the boundary conditions at the wall are full-filled.

\subsubsection{Reconstruction of the fluid-state at ghost-nodes}\label{sec:ghost_node_reconstruction}
\textbf{TODO ask Farhat if this is really how it is done}
To explain the reconstruction at the ghost nodes, it again helps to consider Figure~\ref{fig:FIVER_intersection}.\\
We also notice that the viscous flux, as defined in Equation \ref{eq:fluxes_diff}, only depends on the fluid velocity and the temperature.
Reconstruction of the velocity at the ghost node is straight forward. Assuming a linear interpolation inside the elements, the velocity at node $i$ is reconstructed such, that the stick condition is fulfilled at the interface.\\
As far as the reconstruction of the temperature is concerned, two different cases have to be considered: adiabatic walls and isothermal walls.\\
For an adiabatic wall, the temperature gradient at the wall is zeros, which can be achieved by setting the temperature of the ghost note equal to that of node $i$. An isothermal wall boundary condition enforces a certain, constant temperature at the wall. Similar to the velocity boundary condition, this can be achieved by finding an appropriate ghost point value such that the condition is enforces.\\
Figure~\ref{fig:FIVER_intersection} also reveals that there is no unique solution for the ghost point value. Multiple active node connect to the ghost node and the above described relations can be formulated for every one of them. \ac{FIVER} therefore solves a least square system to chose an appropriate value.\\
The elegance of this approach is that once the ghost point state have been found, the evaluation of the viscous contribution can be done with the standard code routines. No adaption is required to make them work in the embedded framework.


\subsection{Evaluation of forces}\label{sec:fiver_force_evaluation}
Another issue that arises for embedded simulations is the appropraite evaluation of the forces on an embedded interface. This is especially important if the embedded framework is used in an \ac{FSI} setup to get an approriate resultant on the structure.\\
Sensitivity Analysis is another application, where approriate evaluation of Lift and Drag integrals are crucial.\\
This problem as been analyzed in \textbf{REF REF}
MOst importantly, the qeustion becomes whether to evaluate the pressure integral \textbf{REF} over the ausivilary control volume interface \textbf{Ref} or over the so called reconstructed interface, which can be obtained by creating ausiliary surfaces inside the intersected elements through the knowledge ablout intersection points at the edges.

\end{document}
