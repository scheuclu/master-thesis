\section{The actual structure of your report} %Your new development/method/finding
\label{sec:developments}

This section is added to show what the complete structure of your thesis
might look like. Coincidentally, this is also the typical structure of a scientific report, and hence important to know anyway. 
The typical sections are discussed below, but some specifics might vary according to the type of research project. Please discuss the actual structure of your thesis with your supervisor.

\subsection{Abstract}
\label{sec:Abstract}
This is the most often read section. It will determine if someone actually bothers to read more
of what you've written. Only write this section after you have completed your
report. The abstract consist of only one paragraph, mostly limited
to 200-300 words. Literally, a summary of your work. Write a sentence
or two about each of the main sections of your report, as discussed in this section. When
summarizing results, make the reader aware of the most important results (including numbers when applicable)
and important conclusions or questions that follow from these.


\subsection{Introduction}
\label{sec:Introduction}
The objective of writing this section is to introduce the reader to
the problem and also show that you understand the problem statement
in the context of current knowledge in the field.

Any problem in science can be introduced by simply including the following
sections:
\begin{enumerate}
\item Give background about the problem you are investigating. 
\item Describe what has been done up to now to address this problem (and
also refer to the literature in which this work has been done). 
\item State the objective for taking on the current study or the hypothesis
that will be tested in the current study. 
\item Provide a single paragraph of the contents of the rest of the article
(focusing on that which is reported in \textquotedblleft{}Materials and Methods\textquotedblright{},
\textquotedblleft{}Results\textquotedblright{}, \textquotedblleft{}Discussion\textquotedblright{}
and \textquotedblleft{}Conclusions\textquotedblright{}).
\end{enumerate}

\subsection{Materials and Methods}
\label{sec:MaterialsAndMethods}

Describe the relevant materials, methods, tools, equipment, software,
hardware, experimental setup, experimental conditions, general procedures,
etc. so that someone else can repeat your study or judge the scientific
merit thereof. Do not describe everything you did, or who did what
(unless specified or relevant). This section is not a set of instructions! It rather describes the complete methodology used, so that it can be reproduced by someone of similar skill.
Keep it as concise as possible. Also provide the reader the name of
the company that produced relevant equipment/software
used and the country where the company is located.

\subsection{Results}
\label{sec:Results}

Here you must simply report on the results that you achieved. When
reporting the results, provide a context to the reader, such as to
describe the question that was addressed by reporting a specific result.
Do not interpret any results here \textendash{} leave this for the
discussion! Do not show the same results in a figure and a table \textendash{}
choose the one which can best communicate your results. The most important
aspect(s) of the results reported in a table or figure should be reported
in the text while also referring to the table or figure. In other
words, the text should complement the tables and figures. Results
reported here can be raw data (obtained from instruments), converted
data (obtained after converting raw data as described in the \textquotedblleft{}Materials
and Methods\textquotedblright{} section) or applicable numerical examples.


\subsection{Discussion}
\label{sec:Discussion}

Here you must provide an interpretation of your results and support
for all your conclusions using what you have at your disposal \textendash{}
your results and generally accepted knowledge, such as published literature
(if needed). Also describe the significance of your findings clearly.
Commenting on the methods that you employed might also be relevant. 

If your results agree with what you expected, describe the theory
or previously described observations that your results delivered.
If your results do not agree with what expected, explain why this
might have happened. A lot can be learned from what
did not work! You may also deduce alternative explanations if reasonable
ones exist. Understanding and interpreting what the limitations of
your study was or what went wrong is crucial for the general increase of scientific
knowledge -- and of course your own knowledge. Do not neglect this aspect of your discussion.
Do not just dismiss your results as useless or inconclusive if it does
not clearly align with your initially stated objective/hypothesis.
Since you were doing very good scientific research, appropriately
report on the limitations of your study, so that these limitations do not come across
as shortcomings, thereby deeming your work of lower quality. Therefore, attempt
to end your discussion section by motivating the significance of your
results, even if your results have been shown to not be statistically significant/aligned with the initial objectives.


\subsection{Conclusions}
\label{sec:Conclusions}

Concluding remarks should naturally come from the discussion described
above. This is the \textit{Grand Finale} of your report. This section consists of 2 parts at most. In the first part, repeat your most important finding. The last part should 
suggest existing challenges, potential future directions or recommendations
for further studies. Keep it concise!


\subsection{References}
\label{sec:References}

In all of the sections prior to this section, a citation is made in the text which refers to a reference -- i.e. a source where this information comes from. The full details of these references are then listed in the this section -- the references.

The purpose of a citation can best be described as,
\begin{quote}
\textquotedblleft{}A prime purpose of a citation is intellectual honesty: to attribute prior or unoriginal work and ideas to the correct sources, and to allow the reader to determine independently whether the referenced material supports the author's argument in the claimed way.\textquotedblright{}\cite{citation}
\end{quote}

Different styles of citation and accompanying referencing exist. The style refers to the order in which the author names are listed, information of the cited reference to be listed both in the text and also in the references section, formatting, punctuation etc., which should be consistently used throughout the report. The style to be used depends on the type of publication, e.g. book, journal, report, newspaper article, website%
\footnote{For your submitted work, avoid websites - for submission of the great and significant scientific articles that you will one day publish, only refer to websites if
information can be verified!%
}.

For your report to be submitted to LNM, you must use the numerical system when you use
any references in your work. This system of referencing basically
consist of placing a number in-text (starting at \textquotedblleft{}1\textquotedblright{},
of course) next to where you want to indicate a reference.  This reference
is then fully described in the \textquotedblleft{}References\textquotedblright{}
section. 