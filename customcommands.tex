\newcommand{\abbrevation}[2]{\newcommand{#1}{#2}}
\abbrevation{\SQP}{SQP}      %Sequential Quatdratic Programming

%%==============================%
% OPERATORS
% Representation of scalars, tensors and other quantities
%==============================%
\newcommand{\s}[1]{#1}                             %Scalar quantity with{a}
\renewcommand{\v}[1]{\boldsymbol{#1}}              %Vector quantity with{a}
\newcommand{\m}[1]{\boldsymbol{#1}}                %Matrix quantity with{A}
\renewcommand{\c}[1]{\mathnormal{#1}}              %continous quantity with{A,a}
\renewcommand{\d}[1]{\mathrm{#1}}                  %discrete quantity  with{A,a}
\renewcommand{\it}[1]{{#1}^{(k)}}                  %Iteration index in the optimization loop            with{a}
\newcommand{\ito}[1]{{#1}^{(k)}}                   %Iteration index in the optimization loop            with{a}
\newcommand{\its}[1]{{#1}^{(n)}}                   %Iteration index in thestaggered algorithm           with{a}
\newcommand{\fic}[1]{\bar{#1}}                     %Fictious entity                                     with{A}
\newcommand{\order}[1]{\mathcal{O}(#1)}            %Order of magnitude                                  with{A}


\newcommand{\itn}[1]{{#1}^{(0)}}                   %Iteration index 0                                   with{A}
\newcommand{\itss}[1]{{#1}^{(n+1)}}                %Iteration index in thestaggered algorithm           with{A}
\newcommand{\pd}{\mathrm{\partial}}                %TODO check whether I need the partial or not        with{A}
\renewcommand{\vec}[1]{\boldsymbol{#1}}            %Vector                                                                       with{A}
\newcommand{\abs}[1]{\lvert#1\rvert}               %Norm of a vector                                                             with{A}
\newcommand{\expression}[1]{\textit{#1}}           %To mark a new term beein introduced
\newcommand{\prim}[1]{\tilde{#1}}                  %To mark a variabla as being expressen in primitive variables
\newcommand{\dvec}[1]{\boldsymbol{ \mathsf{#1} } } %discrete vector represenation of a scalar field     with{A}
\newcommand{\dmat}[1]{\boldsymbol{\mathsf{#1}}}    %discrete vector represenation of a tensor field     with{A}

\newcommand{\cvec}[1]{\c{\v{#1}}}    %discrete vector represenation of a scalar field     with{A}
\newcommand{\cmat}[1]{\c{\m{#1}}}    %discrete vector represenation of a tensor field     with{A}

\newcommand{\nd}[1]{\bar{#1}} %asdasd
\newcommand{\reference}[1]{ {#1}_{ref} }


\newcommand{\conv}[1]{{#1}^{c}} %convective form of a quantity (quantity in ALE-description}

%%==============================%
% OPERATORS
% Algebraic operations
%==============================%
\newcommand{\T}[1]{{#1}^{T}}                      %Transpose of a tensor                with{A}
\newcommand{\inv}[1]{ {#1}^{-1}}                  %Inverse of a tensor                  with{A}
\newcommand{\av}[1]{\bar{#1}}                     %Average component of a quantity      with{a}
\newcommand{\fluc}[1]{{#1}'}                      %Fluctuating component of a qunatity  with{a}
\newcommand{\norm}[1]{\left\lVert#1\right\rVert}  %Norm of a quatntity                  with{a}


%%==============================%
% OPERATORS
% Analytic operations
%==============================%
\newcommand{\dderivtime}[1]{\ddot{#1}}                        %Second time derivative at a fixed refrence position                   with{A}
\newcommand{\derivtime}[1]{\dot{#1}}                          %First time derivative at a fixed reference position                   with{A}
\newcommand{\pdfrac}[2]{\frac{\pd #1}{\pd #2}}                %partial derivative of one argument with respect to the other          with{A&B}
\newcommand{\ppdfrac}[2]{\frac{\pd^{2} #1}{\pd {#2}^{2}}}     %Second partial derivative of one argument with respect to the other   with{A&B}
\newcommand{\mfrac}[2]{\frac{D #1}{D #2}}                     %Material time derivative                                              with{A&B}
\renewcommand{\tfrac}[2]{\frac{\mathrm{d} #1}{\mathrm{d} #2}} %First total derivative of one argument with respect to the other      with{A&B}




\newcommand{\stdy}[1]{{#1}^{0}}                  %Value at steady state
\newcommand{\0}{\textcolor{lightgrey}{0}}
\newcommand{\tensor}[1]{\mathcal{\boldsymbol{#1}}} %Vector quantity with{a}
\newcommand{\scalar}[1]{\s #1}                        %Scalar quantity with{a}


%==============================
% SYMBOLS
% 3-field formulations
%==============================
\newcommand{\EOSstruct}{\mathcal{S}_{gov}}     %State equation of the structure
\newcommand{\EOSmesh}  {\mathcal{D}_{gov}}     %State equation of the mesh
\newcommand{\EOSfluid} {\mathcal{F}_{gov}}     %State equation of the fluid
\newcommand{\mpos}       {\vec{x}}        %Fluid mesh motion
\newcommand{\dmpos}       {\dvec{x}}        %Fluid mesh motion
\newcommand{\mms}       {\dot{\vec{x}}}        %Fluid mesh motion
\newcommand{\dmms}       {\dot{\dvec{x}}}        %Fluid mesh motion
\newcommand{\mmsad}       {\vec{a}_{\vec{x}}}        %Adjoint fluid mesh motion
\newcommand{\fstate}    {\vec{w}}        %Fluid state vector
\newcommand{\fstatead}{\vec{a}_{\vec{w}}}  %Adjoint fluid state vector
\newcommand{\fstateprim}{\prim{\fstate}}        %Primitive ffluid state vector
\newcommand{\dfstateprim}{\prim{\dfstate}}        %Primitive ffluid state vector
\newcommand{\dfstate}    {\dvec{w}}        %Discrete fsluid state vector
\newcommand{\dresidual}    {\dvec{R}}        %Discrete fsluid state vector
\newcommand{\structdisp}{\vec{u}}        %Structure displacement
\newcommand{\structdispad}{\vec{a}_{\vec{u}}}        %Adjoint structure displacement
\newcommand{\load}{\vec{P} }          %Fluid load TODO
\newcommand{\fload}{\vec{P}_F}           %Fluid load
\newcommand{\sload}{\vec{P}_T}           %Structure load
\newcommand{\jactwo}{\tensor{H}_2}       %Second order Jacobian
\newcommand{\jacale}{\tensor{J}}       %Second order Jacobian
\newcommand{\ifaceprojFtoS}{\dmat{T}_p}  %Interface projection matrix from fluid to structure mesh
\newcommand{\ifaceprojStoF}{\dmat{T}_u}  %Interface projection matrix from structure to fluid mesh
\newcommand{\fstaterans}{\fstate_{RANS}}   %Augumentes fluid state vector in the \ac{RANS} formulation
\newcommand{\turbulenceparam}{\chi}        %Additional fluid state variable introduced by the turbulence model
\newcommand{\turbparamvec}{\dvec{\chi}}        %Additional fluid state variable introduced by the turbulence model
\newcommand{\turbulencesource}{\mathsf{S}} %Source term in the \ac{RANS} equations
\newcommand{\fluxmatconv}{\dmat{F}}       %Convective part of the flux matrix
\newcommand{\fluxmatdiff}{\dmat{G}}       %diffusive part of the flux matrix
\newcommand{\turbmat}{\dmat{S}}           %Turbulence term
\newcommand{\cellvolmat}{\dmat{A}}        %Diagonal matrix with vell volumes

\newcommand{\specificwork}{\vec{w}}       %specific work
\newcommand{\aoa}{\alpha_{\infty}}  %angle of attack



%==============================%
% SYMBOLS
% Optimization
%==============================%
\newcommand{\costfunc}{z}               %Target cost function
\newcommand{\eqctr}{\vec{h}}            %Equality constraints
\newcommand{\numeqctr}{n_{\vec{h}}}     %Number of equality constraints
\newcommand{\neqctr}{\vec{g}}           %Non-equality constraints
\newcommand{\numneqctr}{n_{\vec{g}}}    %Number of non-equality constraints
\newcommand{\absvar}{s}                 %Abstract optimazation variable
\newcommand{\absvars}{\vec{s}}          %Vector of abstract optimization variables
\newcommand{\optcrit}{q}                %Optimization criterium
\newcommand{\optcrits}{\vec{\optcrit}}  %Vector of optimization criteria
\newcommand{\physvars}{\vec{d}}         %Physical design parameters
\newcommand{\Lagfunc}{L}                %Lagrangian function of the optimization problem
\newcommand{\lagmultseq}{\vec{\eta}}    %Lagrange multipliers of the equality constraints
\newcommand{\lagmultsneq}{\vec{\gamma}} %Lagrange multipliers of the inequality constraints
\newcommand{\adjoints}{\vec{a}}         %Adjoint solutions
\newcommand{\tolsa}{\epsilon^{SA}}      %Specified tolerance in the Sensitivity analysis
\newcommand{\lift}{L}                   %Lift of an airfoil
\newcommand{\drag}{D}                   %Drag of an airfoil


%==============================%
% SYMBOLS
% Fluid Mechanics
%==============================%
\newcommand{\fluxesconv}{\mathcal{F}} %Convective fluxes
\newcommand{\fluxesdiff}{\mathcal{G}} %Diffusive fluxes
\newcommand{\dens}{\rho}              %Density
\newcommand{\pres}{p}                 %Pressure
\newcommand{\dpres}{\mathsf{p}}                 %Pressure
\newcommand{\totpres}{p_{tot}}                 %Pressure
\newcommand{\fluidvel}{\vec{v}}       %Fluid velocity vector
\newcommand{\fluidvelcomp}{v}         %Fluid velocity vector
\newcommand{\fluidvelx}{{v_1}}          %Fluid velocity in x-direction
\newcommand{\fluidvely}{{v_2}}          %Fluid velocity in y-direction
\newcommand{\fluidvelz}{{v_3}}          %Fluid velocity in z-direction
\newcommand{\energytot}{E}            %Total energies
\newcommand{\energyint}{e}            %Internal energy
\newcommand{\eye}{\dmat{I}}           %Identity matrix
\newcommand{\fluidstrain}{\tensor{\epsilon}}    %Deviatoric fluid stress tensor
\newcommand{\fluidstress}{\tensor{\tau}}    %Deviatoric fluid stress tensor
\newcommand{\fluidstresscomp}{\tau}   %Deviatoric fluid stress tensor
\newcommand{\thermcond}{k}            %Thermal conductivity of the fluid
\newcommand{\temp}{T}                 %Fluid temperature
\newcommand{\heatfluxcomp}{q}         %Heat flux comopnenent
\newcommand{\heatflux}{\vec{\heatfluxcomp}} %Heat flux vector
\newcommand{\viscosdyn}{\mu}          %Dynamic viscosity
\newcommand{\viscoskin}{\nu}          %Kinematic viscosity
\newcommand{\reynolds}{R_{e}}            %Reynolds number
\newcommand{\specheatratio}{\gamma}   %Specific heat ratio
\newcommand{\shr}{\gamma}   %Specific heat ratio
\newcommand{\jac}{\mathcal{H}}        %Jacoian matrix
\newcommand{\fluxjac}{\mathcal{A}}    %Flux Jacobian
\newcommand{\jaceigvecs}{\tensor{P}}  %Matrix that contains the eigenvectors of the jacobian matrix of $\fluxmatconv$
\newcommand{\jaceigvals}{\tensor{\Lambda}}  %Diagonal matrix that contains the eigenvalues of the jacobian matrix of $\fluxmatconv$
\newcommand{\roeavgfunc}{\tensor{M}}  %Averaging function associated with the Roe flux
\newcommand{\machnum}{M}              %Mach number
\newcommand{\M}{\machnum}              %Mach number



\newcommand{\dualcell}{\mathcal{C}}
\newcommand{\difftensor}{\mathbb{K}}
\newcommand{\primmesh}{\mathcal{P}_{h}}
\newcommand{\dualmesh}{\mathcal{D}_{h}}
\newcommand{\normal}{\vec{n}}        %Symbol for dyadic product
\newcommand{\wnormal}{\vec{\nu}}        %Weighted normal
\newcommand{\normalx}{n_1}        %Symbol for dyadic product
\newcommand{\normaly}{n_2}        %Symbol for dyadic product
\newcommand{\normalz}{n_3}        %Symbol for dyadic product
\newcommand{\testfunc}{\phi}         %Variational test function
\newcommand{\vertex}{\boldsymbol{X}}  %Mesh vertex
\newcommand{\vertexset}{\mathsf{\kappa}} %Set of vertices
\newcommand{\elementset}{\mathsf{\lambda}} %Set of elements related to vertex i
\newcommand{\fluxesnum}{\boldsymbol{\mathsf{{\phi}}}} %numerical flux
\newcommand{\roeavgmat}{\tensor{A}_{\text{Roe}}}  %Averaging matrix of the Roe flux
\newcommand{\eigval}{\lambda}         %Eigen value
\newcommand{\sspeed}{c} %Speed of sound

\newcommand{\dyadic}{\otimes}        %Symbol for dyadic product
\newcommand{\fot}{\frac{1}{2}}
\newcommand{\REF}{\textbf{REF}}
%%%==============================%
%%% SYMBOLS
%%% Fluid Analysis
%%%==============================%
%%\newcommand{\sate}{w}
%%\newcommand{•}{•}
%%\newcommand{•}{•}
%%\newcommand{•}{•}
%%\newcommand{\statevec}{\vec{\state}}


%==============================%
% SYMBOLS
% Sturctural Analysis
%==============================%
\newcommand{\stiffmat}{\dmat{K}}               %\ac{FE} stiffness matrix
\newcommand{\disp}{u}                          %Displacement vector
\newcommand{\dispvec}{\dvec{\disp}}            %Discrete displacement vector
\newcommand{\motion}{\vec{x}}                  %Mesh motion
\newcommand{\ifacedisp}{d}                     %Interface displacement
\newcommand{\ifacedispvec}{\dvec{\ifacedisp}}  %Interface displacement 


%==========================================
% SYMBOLS
% Fluid Structure Interaction
%==========================================
\newcommand{\strucstateq}{\mathcal{P}}         %State equation of the structure
\newcommand{\fluidstateq}{\mathcal{F}}         %State equation of the fluid
\newcommand{\mmsstateeq}{\mathcal{D}}          %State equation of the mesh motion


%=============================================================================%
% Nassi-Schneidermann auxiliary commands                                      %
%=============================================================================%

%commands for the text part of the dioagrams
%the bold command is typcally used for TODO
%there is now automatic line wrapping in strtuktex, so the  second command is
% used if the text becomes too long
\newcommand{\nassitext}[2]{\makebox[#1\textwidth]{\textit{\textcolor{nassitextcolor}{\small #2}}\hfill}}
\newcommand{\nassitextbold}[2]{\makebox[#1\textwidth]{\textbf{\textit{\textcolor{nassitextcolor}{\small #2}}}\hfill}}
\newcommand{\nassitexttwolines}[3]{\nassitext{#1}{#2}\hspace*{\hsize}\linebreak \nassitext{\df2}{#3}}

\newcommand{\tablesix}[6]{
\begin{tabular}{|c|l|l|}
\hline
  \parbox[t]{2mm}{\multirow{3}{*}{\rotatebox[origin=c]{90}{Eulerian}}}  & Euler      & #1 \\
  \hhline{~--}                                                          & \ac{NSE}   & #2 \\
  \hhline{~--}                                                          & \ac{RANS}  & #3 \\
\hline
  \parbox[t]{2mm}{\multirow{3}{*}{\rotatebox[origin=c]{90}{ALE}}}       & Euler      & #4 \\
  \hhline{~--}                                                          & \ac{NSE}   & #5 \\
  \hhline{~--}                                                          & \ac{RANS}  & #6 \\
\hline
\end{tabular}
}


\newcommand{\tablestatederivs}[6]{
\begin{center}
\begin{tabular}{ m{0.45\textwidth} | m{0.45\textwidth} }\hline
\rowcolor{black!20}\Centering Dimensional & \Centering Non-Dimensional\\ \hline
#1 & #2 \\
#3 & #4 \\
#5 & #6 \\
\end{tabular}
\end{center}
}


%\newenvironment{salign}
%    {
%    \begingroup\makeatletter\def\f@size{8}\check@mathfonts
%    \def\maketag@@@#1{\hbox{\m@th\large\normalfont#1}}
%    \begin{align}
%    }
%    { 
%    \end{align}
%    \endgroup
%    }

\newcommand\question[1]{\textcolor{red}{\textbf{#1}}}

