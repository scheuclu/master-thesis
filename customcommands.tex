%%==============================%
% OPERATORS
% Representation of scalars, tensors and other quantities
%==============================%
\newcommand{\scalar}[1]{#1}                        %Scalar quantity with{A}
\newcommand{\tensor}[1]{\mathcal{\boldsymbol{#1}}} %Tensor with{A}
\newcommand{\derivtime}[1]{\dot{#1}}               %First time derivative at a fixed reference position with{A}
\newcommand{\dderivtime}[1]{\ddot{#1}}             %Second time derivative at a fixed refrence position                       with{A} %partial derivative of a quantity
\newcommand{\dvec}[1]{\boldsymbol{ \mathsf{#1} } } %discrete vector represenation of a scalar field with{A}
\newcommand{\dmat}[1]{\boldsymbol{\mathsf{#1}}}    %discrete vector represenation of a tensor field with{A}
\renewcommand{\it}[1]{{#1}^{(k)}}    %Iteration index in the optimization loop{A}
\newcommand{\ito}[1]{{#1}^{(k)}}    %Iteration index in the optimization loop{A}
\newcommand{\its}[1]{{#1}^{(n)}}    %Iteration index in thestaggered algorithm{A}
\newcommand{\fic}[1]{\bar{#1}}      %Fictious entity
\newcommand{\av}[1]{\bar{#1}}       %Time-averaged quantity
\newcommand{\fluc}[1]{{#1}'}       %Time-averaged quantity
\newcommand{\fstaterans}{\fstate_{RANS}} %Augumentes fluid state vector in the RANS formulation
\newcommand{\turbulenceparam}{\chi} %Additional fluid state variable introduced by the turbulence model
\newcommand{\turbulencesource}{\mathsf{S}} %Source term in the RANS equations

\newcommand{\itn}[1]{{#1}^{(0)}}    %Iteration index 0{A}
\newcommand{\itss}[1]{{#1}^{(n+1)}}    %Iteration index in thestaggered algorithm{A}
\newcommand{\pd}{\mathrm{\partial}} %TODO check whether I need the partial or not
\newcommand{\pdfrac}[2]{\frac{\pd #1}{\pd #2}} %partial derivative of one argument with respect to the other
\newcommand{\ppdfrac}[2]{\frac{\pd^{2} #1}{\pd {#2}^{2}}} %Second partial derivative of one argument with respect to the other
\renewcommand{\tfrac}[2]{\frac{\mathrm{d} #1}{\mathrm{d} #2}} %First total derivative of one argument with respect to the other
\renewcommand{\vec}[1]{\boldsymbol{#1}}            %Vector
\newcommand{\norm}[1]{\left\lVert#1\right\rVert}

\newcommand{\0}{\textcolor{lightgrey}{0}}

%==============================%
% OPERATORS
% Operators and Symbols
%==============================%
\newcommand{\T}[1]{{#1}^{T}}       %Transpose of a tensor with{A}
\newcommand{\inv}[1]{ {#1}^{-1}}   %Inverse of a tensor   with{A}


%==============================
% SYMBOLS
% 3-field formulations
%==============================
\newcommand{\EOSstruct}{\mathcal{S}}      %State equation of the structure
\newcommand{\EOSmesh}  {\mathcal{D}}      %State equation of the mesh
\newcommand{\EOSfluid} {\mathcal{F}}      %State equation of the fluid
\newcommand{\mms}       {\vec{x}}        %Fluid mesh motion
\newcommand{\fstate}    {\vec{w}}        %Fluid state vector
\newcommand{\structdisp}{\vec{u}}        %Structure displacement
\newcommand{\fload}{\vec{P}_F}           %Fluid load TODO
\newcommand{\sload}{\vec{P}_T}           %Structure load TODO
\newcommand{\jactwo}{\tensor{H}_2}       %Second order Jacobian  TODO
\newcommand{\ifaceprojFtoS}{\dmat{T}_p}  %Interface projection matrix from fluid to structure mesh
\newcommand{\ifaceprojStoF}{\dmat{T}_u}  %Interface projection matrix from structure to fluid mesh



%==============================%
% SYMBOLS
% Optimization
%==============================%
\newcommand{\costfunc}{z}        %Target cost function
\newcommand{\eqctr}{\vec{h}}     %Equality constraints
\newcommand{\numeqctr}{n_{\vec{h}}}     %Number of equality constraints
\newcommand{\neqctr}{\vec{g}}    %Non-equality constraints
\newcommand{\numneqctr}{n_{\vec{g}}} %Number of non-equality constraints
\newcommand{\absvar}{s}          %abstract optimazation variable
\newcommand{\absvars}{\vec{s}}   %Vector of abstract optimization variables
\newcommand{\optcrit}{q}  %Optimization criterium
\newcommand{\optcrits}{\vec{\optcrit}}  %Vector of optimization criteria
\newcommand{\phydsgpar}{\vec{d}} %physical design parameters
\newcommand{\Lagfunc}{L} %Lagrangian function of the optimization problem
\newcommand{\lagmultseq}{\vec{\eta}} %Lagrange multipliers of the equality constraints
\newcommand{\lagmultsneq}{\vec{\gamma}} %Lagrange multipliers of the inequality constraints
\newcommand{\adjoints}{\vec{a}} %Adjoint solutions
\newcommand{\tolsa}{\epsilon^{SA}} %Specified tolerance in the Sensitivity analysis


%==============================%
% SYMBOLS
% Fluid Mechanics
%==============================%
\newcommand{\fluxesconv}{\mathcal{F}} %Convective fluxed
\newcommand{\fluxesdiff}{\mathcal{G}} %Diffusive fluxes
\newcommand{\dens}{\rho}              %Density
\newcommand{\pres}{p}              %Pressure
\newcommand{\fluidvel}{\vec{v}}     %Fluid velocity vector
\newcommand{\fluidvelx}{v_x}     %Fluid velocity in x-direction
\newcommand{\fluidvely}{v_y}     %Fluid velocity in y-direction
\newcommand{\fluidvelz}{v_z}     %Fluid velocity in z-direction
\newcommand{\energytot}{E}         %Total energys
\newcommand{\energyint}{e}        %Internal energy
\newcommand{\eye}{\dmat{I}}        %Identity matrix
\newcommand{\fluidstress}{\tensor{\tau}}      %Deviatoric fluid stress tensor
\newcommand{\fluidstresscomp}{\tau}      %Deviatoric fluid stress tensor
\newcommand{\thermcond}{k}         %Thermal conductivity of the fluid
\newcommand{\temp}{T}         %Fluid temperature
\newcommand{\heatfluxcomp}{q} %Heat flux comopnenent
\newcommand{\heatflux}{\vec{\heatfluxcomp}} %Heat flux vector
\newcommand{\viscosdyn}{\mu} %Dynamic viscosity
\newcommand{\viscoskin}{\nu} %Kinematic viscosity
\newcommand{\reynolds}{RE} %Reynolds number
\newcommand{\specheatratio}{\gamma} %Specific heat ratio

\newcommand{\dyadic}{\otimes}        %Symbol for dyadic product
%%%==============================%
%%% SYMBOLS
%%% Fluid Analysis
%%%==============================%
%%\newcommand{\sate}{w}
%%\newcommand{•}{•}
%%\newcommand{•}{•}
%%\newcommand{•}{•}
%%\newcommand{\statevec}{\vec{\state}}


%==============================%
% SYMBOLS
% Sturctural Analysis
%==============================%
\newcommand{\stiffmat}{\dmat{K}}               %Finite Element stiffness matrix
\newcommand{\disp}{u}                          %Displacement vector
\newcommand{\dispvec}{\dvec{\disp}}            %Disrete displacement vector
\newcommand{\motion}{\vec{x}}                  %Mesh motion
\newcommand{\ifacedisp}{d}                     %Interface Displacement
\newcommand{\ifacedispvec}{\dvec{\ifacedisp}}  %Interface Displacent 


%==========================================
% SYMBOLS
% Fluid Structure Interaction
%==========================================
\newcommand{\strucstateq}{\mathcal{P}}         %TODO
\newcommand{\fluidstateq}{\mathcal{F}}         %TODO
\newcommand{\mmsstateeq}{\mathcal{D}}          %TODO

%\def\abbrevation}[1]{~}

%==========================================
% ABBREVATIONS
% Abbrevations
%==========================================
%\abbrevation{SQP}      %Sequential Quatdratic Programming
%\abbrevation{GSE}      %Gloabal Sensitivity Equations
%\abbrevation{SA}       %Sensitivity Analysis
%\abbrevation{VOF}      %Volume of Fluid Methods
%\abbrevation{GFM}      %GFMP
%\abbrevation{GFMP}     %Ghost Fluid Method of the Poor
%\abbrevation{EOS}      %Equation of state
%\abbrevation{JWL}      %Jones-Wilkins-Lee
%\abbrevation{SG}       %Stiffend Gas
%\abbrevation{MUSCL}    %Monotonic Upwind scheme for Conservation Laws
%\abbrevation{FIVER}    %Finite Volume Method with exact two-phase Riemann integrals


%=============================================================================%
% Nassi-Schneidermann auxiliary commands                                      %
%=============================================================================%

%commands for the text part of the dioagrams
%the bold command is typcally used for TODO
%there is now automatic line wrapping in strtuktex, so the  second command is
% used if the text becomes too long
\newcommand{\nassitext}[2]{\makebox[#1\textwidth]{\textit{\textcolor{nassitextcolor}{\small #2}}\hfill}}
\newcommand{\nassitextbold}[2]{\makebox[#1\textwidth]{\textbf{\textit{\textcolor{nassitextcolor}{\small #2}}}\hfill}}
\newcommand{\nassitexttwolines}[3]{\nassitext{#1}{#2}\hspace*{\hsize}\linebreak \nassitext{\df2}{#3}}

%%%%%These commands are used to highlight specific parts of the diagramm, e.g.
%%%%%algorithm parts that have changed compared to a previous implementation
%%%%\usepackage{soul}
%%%%\definecolor{hlcolor}{RGB}{230,230,230}
%%%%\setulcolor{red}
%%%%\newsavebox\MBox
%%%%\newcommand\hlchange[2][red]{{\sbox\MBox{$#2$}%
%%%%  \rlap{\usebox\MBox}\color{#1}\rule[-1.7\dp\MBox]{\wd\MBox}{1.0pt}}}


%%%%% custom scale command; TODO usage unclear
%%%%\newcommand*\MyScale{1}
%%%%\newcommand*\Scale[1][1]{\renewcommand*\MyScale{#1}}
%%%%\tikzset{%
%%%%  every picture/.style={%
%%%%    scale=\MyScale,
%%%%  }
%%%%}


%\renewcommand\thesection{\roman{section}}
%\renewcommand\thesubsection{\thesection.\roman{subsection}}
%\renewcommand\thesubsubsection{\thesection.\thesubsection.\roman{subsection}}
