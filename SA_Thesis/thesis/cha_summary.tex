\chapter{Summary}\label{cha:summary}

In conclusion, the FETI-method has been introduced as a prominent domain decomposition method for structural mechanics problems. The underlying motivation for domain decomposition methods has been described in Chapter~\ref{cha:introduction} and a quick comparison to other DD methods has been drawn.\\
Chapter~\ref{cha:feti_formulation} introduced the standard FETI method(FETI-1), as first described by \cite{Farhat1991} and derived all relevant equations. It has then been shown, that the FETI-1 method shows some severe deficiencies, especially for the problems mentioned in Chapter~\ref{cha:numerical_assesment}.\\
Variations of the FETI-1 method, namely FETI-2 and FETI-S have therefore been described in Chapter~\ref{cha:feti_solvers} and their theoretical potentials have been discussed.
Inspired by\cite{Spillane2016} the concept of an Adaptive Multi Preconditioned Conjugate Gradient algorithm has then been adapted to the FETI-method leading to the FETI-A algorithm. To my knowledge this has not been done before.\\
Since the FETI-AS algorithm depends on a user specified threshold $\tau$ which is delicate to choose, a variation of the algorithm, the Fast Adaptive Siumltaneous FETI solver(FETI-AS) has been proposed in Section~\ref{sec:fetias}.\\
All five algorithms have been thoroughly discussed and summarized in Chapter~\ref{cha:feti_solvers}.\\
Chapter~\ref{cha:numerical_assesment} has then been dedicated to an intensive analysis and comparison of the five described FETI-algorithms. For that purpose four setups, covering the most challenging situations for DD methods, have been devised and solved with each algorithm. Several conclusions have been drawn.\\
First of all, the before mentioned inferiority of the FETI-1 methods has been confirmed in each setup. Moreover, it has been shown, that FETI-S and FETI-2(Geneo) show comparable results, with slight advantages for the FETI-S(Geneo) approach when it comes to the total iteration number. However, as described in Chapter~\ref{cha:feti_solvers} the Geneo approach involves the solution of eigenvalue problems on the substructure interfaces. The overall conclusion therefore was, that the slightly reduced iteration count of the Geneo approach, does not justify the large computational overhead as compared to the FETI-S algorithm, especially with the efficient implementation as outlined in Section~\ref{sec:fetis}.\\
Moreover, the examples generally confirmed that one usually can neglect a large part of the search directions in the FETI-S solver, and still get reasonable iteration numbers. This was the main idea behind the Adaptive Simultaneous FETI solvers. It has also been shown that the choice of an appropriate parameter $\tau$ in the FETI-AS algorithm is a delicate issue. While prescribing a desired contraction factor, instead of directly prescribing $\tau$ mediated the problem, a general recommendation for $\tau$ can still not be drawn.\\
Our simulations thus suggested that the FETI-FAS solver can be favoured as a more reasonable, easier to choice that performs very well in most cases.\\
\\
For the purpose of this thesis, an object oriented finite element code (FEMAC) has been written in Matlab\cite{FEMAC}.