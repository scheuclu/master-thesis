\chapter{Outlook}

The potentials of the FETI-As approaches have been vividly demonstrated in Chapter~\ref{cha:numerical_assesment}.\\
Subsequent research should thus be focused on the development of the Simultaneous FETI  methods. When it comes to FETI-AS, a closer look can be paid upon the appropriate choice of $\tau$. It seems reasonable that better rules can be derived.\\
WE also propose to focus on the herein newly introduced FETI-FAS method. The solver parameters can definitely still be tuned, and an application to dynamic problems remains to be done.\\
This paper focused on a simple small strain, linear finite element formulation. As a first step, the method should thus be generalized to a non-linear, dynamic formulation.\\
Since the herein used code was written in Matlab, meaningful CPU-time analyses could not have been carried out and thus remain to be done too.\\
Special attention should also be given to an profound analyses of the parallel scalability of the method.\\
Further analyses should be carried out on an efficient low-level implementation like C++ or FORTRAN.