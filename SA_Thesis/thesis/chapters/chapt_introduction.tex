\chapter{Introduction} a
\section{Motivation}
The field of high performance computing has seen a significant paradigm shift over the past 10-15 years. While Moores law \cite{Moore1965} is still valid to this day, the way progress is achieved has drastically changed. Single-core performance has been stagnating for years, which gave rise to the advent of multi-core processors. But as the architectures are changing, so do the algorithms. Amdahl's law\cite{Rodgers1985} poses a harsh reqirements on efiicient parallel algorithms. In order to reach satisfing scalability, inter processor communication, as well as serial algorithm parts, have to be kept at an absolute minimum.
An intuitive way of approaching that goal are domain decomposition (DD) methods. In particular, the Finite Element Tearing and Interconnection (FETI) or the Balanced Domain Decomposition (BDD) ar well-established methods for mechanical problems.
The basic idea of DD methods is to split the domain in to, ideally, equal-sized subdomains(substructures). Local Problems can then be solved for each substructure simultaneusly on multiple cores sequentially, before an iterative technique is required to connect the domains together by finding the common interface unknowns.\\
Generally, primal and dual DD algorithms can be differntiated by the type of interface quantity that are used for the connection. Primal DD methods use interface displacements, whereas dual DD methods solve for the interface forces. An profound overview over these methods can be found in \cite{Gosselet2006}
\section{Objective} a