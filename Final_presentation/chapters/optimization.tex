\section{Solution sensitivity}

\def\incrabsvars{\Delta \absvars}
\def\incrlagmultsneq{\Delta \lagmultsneq}
\def\incrlagmultseq{\Delta \lagmultseq}

\def\PPLagfuncBYabsvars{\ppdfrac{\Lagfunc}{\absvars}}
\def\PLagfuncBYabsvars{\pdfrac{\Lagfunc}{\absvars}}
\def\PneqctrBYabsvars{\pdfrac{\neqctr}{\absvar}}
\def\PeqctrBYabsvars{\pdfrac{\eqctr}{\absvar}}

\def\DoptcritJBYabsvarI{\frac{d \optcrit_{j}}{d \absvar_{i}}}
\def\PoptcritJBYabsvarI{\pdfrac{\optcrit_{j}}{\absvar_{i}}}

\def\PoptcritJBYstructdisp{\pdfrac{\optcrit_{j}}{\structdisp}}
\def\PoptcritJBYmms       {\pdfrac{\optcrit_{j}}{\mms}}
\def\PoptcritJBYfstate    {\pdfrac{\optcrit_{j}}{\fstate}}

\def\DoptcritJBYstructdisp{\tfrac{\optcrit_{j}}{\structdisp}}
\def\DoptcritJBYmms       {\tfrac{\optcrit_{j}}{\mms}}
\def\DoptcritJBYfstate    {\tfrac{\optcrit_{j}}{\fstate}}

\def\DstructdispBYabsvarI{\frac{d \structdisp}{d \absvar_{i}} }
\def\DmmsBYabsvarI       {\frac{d \mms       }{d \absvar_{i}} }
\def\DfstateBYabsvarI    {\frac{d \fstate    }{d \absvar_{i}} }

\def\PstructdispBYabsvarI{\pdfrac{d \structdisp}{d \absvar_{i}} }
\def\PmmsBYabsvarI       {\pdfrac{d \mms       }{d \absvar_{i}} }
\def\PfstateBYabsvarI    {\pdfrac{d \fstate    }{d \absvar_{i}} }

\def\PEOSstructBYabsvarI{\pdfrac{\EOSstruct} {\absvar_i}}
\def\PEOSmeshBYabsvarI  {\pdfrac{\EOSmesh}  {\absvar_i}}
   %\def\PEOSfluidBYabsvarI{\pdfrac{\EOSfluid}{\absvar_i}}

\def\DEOSstructBYabsvarI{\tfrac{\EOSstruct} {\absvar_i}}
\def\DEOSmeshBYabsvarI  {\tfrac{\EOSmesh}  {\absvar_i}}
  %\def\DEOSfluidBYabsvarI{\tfrac{\EOSfluid}{\absvar_i}}

\def\PEOSstructBYstructdisp{\pdfrac{\EOSstruct} {\structdisp}}
\def\PEOSstructBYmms{\pdfrac{\EOSstruct}{\mms}}
\def\PEOSstructBYfstate    {\pdfrac{\EOSstruct}{\fstate}}

\def\PEOSmeshBYstructdisp{\pdfrac{\EOSmesh}{\structdisp}}
\def\PEOSmeshBYmms       {\pdfrac{\EOSmesh} {\mms}}
\def\PEOSmeshBYfstate    {\vec{0}}

\def\PEOSfluidBYstructdisp{\vec{0}}
   %\def\PEOSfluidBYmms       {\pdfrac{\EOSfluid} {\mms}}
   %\def\PEOSfluidBYfstate    {\pdfrac{\EOSfluid} {\fstate}}

\def\PstructdispBYabsvarI{\pdfrac{\structdisp} {\absvar_i}}
\def\PmmsBYabsvarI       {\pdfrac{\mms}        {\absvar_i}}
\def\PfstateBYabsvarI    {\pdfrac{\fstate}     {\absvar_i}}

\begin{frame}
  \frametitle{PDE-constrained optimization}

  \begin{itemize}
    \item PDE-constrained optimization for steady problems
      \begin{equation*}
        \begin{aligned}
          & \underset{\ubold \in \Rbb^{N_\wbold},~\mubold \in \Rbb^{N_\mubold}}{\text{minimize}}
          & & \optcrit(\ubold, \mubold)\onslide<2->{\rightarrow \text{e.g. Lift-Drag ratio}} \\
          & \text{subject to}
          & & \Rbold(\ubold, \mubold) = 0 \onslide<3->{\rightarrow \text{e.g. geometry of engine mount}} \\
          & & & \cbold(\ubold, \mubold) \leq 0 \onslide<4->{\rightarrow \text{e.g. Lift-Drag ratio}}
        \end{aligned}
      \end{equation*}
    \item Nested approach
      \begin{equation*}
        \begin{aligned}
          & \underset{\mubold \in \Rbb^{N_\mubold}}{\text{minimize}}
          & & \optcrit(\ubold(\mubold), \mubold) \\
          & \text{subject to}
          & & \cbold(\ubold(\mubold), \mubold) \leq 0.
        \end{aligned}
      \end{equation*}
  \end{itemize}
\end{frame}

%------------------------------------------------------------------------------

\begin{frame}
  \frametitle{Computation of the gradient}
  
  \begin{itemize}  
    \item Gradients of the objective function
			\begin{align*}\label{eq:sensitivity_startingpoint}
			\DoptcritJBYabsvarI\bigg\rvert_{\fstate_0}=\setulcolor{primary_0}\setul{}{2pt}
			\underbrace{\PoptcritJBYabsvarI\bigg\rvert_{\fstate_0}}_{
			                                                        \substack{
			                                                                 \text{\ul{directly derived from}} \\
			                                                                 \text{\ul{the definition of} $\optcrit$}
			                                                                 } 
			                                                        }  +
			\underbrace{\PoptcritJBYfstate\bigg\rvert_{\fstate_0}}_ {\setulcolor{neutral_1}
			                                                        \substack{
			                                                                 \text{\ul{derived analytically or~~}}\\
			                                                                 \text{\ul{by} FD}
			                                                                 }
			                                                        }
			\underbrace{\DfstateBYabsvarI\bigg\rvert_{\fstate_0}}_  {\setulcolor{neutral_2}
			                                                        \substack{
			                                                                 \text{\ul{derived from dynamic}}\\
			                                                                 \text{\ul{fluid equilibrium}}
			                                                                 }
			                                                        }
			\end{align*}
%    \item If the flow residual is converged
%      \begin{equation*}
%        \Rbold(\ubold(\mubold), \mubold) = 0, \,\,\forall \mubold 
%        \quad \longrightarrow \quad
%        \oder{\Rbold}{\mubold} = \pder{\Rbold}{\mubold} + \pder{\Rbold}{\ubold}\pder{\ubold}{\mubold} = 0        
%      \end{equation*}
%
%      \begin{equation*}
%        \pder{\ubold}{\mubold} = -\left[\pder{\Rbold}{\ubold}\right]^{-1}\pder{\Rbold}{\mubold}
%      \end{equation*}
%    \item The derivative of the objective function becomes\\
%      \begin{center}
%      \begin{tikzpicture}[node distance = 1.2cm, auto]
%        \node [blocke] (init) {$\oder{\optcrit}{\mubold} = \pder{\optcrit}{\mubold} - 
%                              \pder{\optcrit}{\ubold}\left(\left[\pder{\Rbold}{\ubold}\right]^{-1}
%                              \pder{\Rbold}{\mubold}\right)$};
%      \end{tikzpicture}
%      \end{center}
  \end{itemize}

\end{frame}

%------------------------------------------------------------------------------

\begin{frame}
\frametitle{Derivation of $\pdfrac{\dfstate}{\absvar}$}
\begin{align*}
\cancelto{0}{\pdfrac{\bar{\dfstate_i}}{t}} +
\underbrace{
  \underbrace{\sum_{j \in \vertexset(i)} \fluxesnum_{ij}(\dfstate_{ij},\dfstate_{ji},\wnormal_{ij})}_{\dresidual^i} -
  \underbrace{\sum_{T_i \in \elementset(i)} \int_{T_j} \difftensor \nabla \dfstate \nabla \phi_j dx}_{\dresidual^v}      }_{\dresidual} =
\vec{0}
\end{align*}


\def\DdresidualBYabsvarI{ \tfrac{\dresidual}{\absvar_i} }
\def\PdresidualBYabsvarI{ \pdfrac{\dresidual}{\absvar_i} }
\def\PdresidualBYdfstate{ \pdfrac{\dresidual}{\dfstate}  }
\def\DdfstateBYabsvarI  { \tfrac{\dfstate}{\absvar_i}   }
\def\PdresidualBYdmms   { \pdfrac{\dresidual}{\dmpos}     }
\def\DdmmsBYabsvarI     { \tfrac{\dmpos}{\absvar_i}      }
\begin{align*}
\DdresidualBYabsvarI=\vec{0}=
\PdresidualBYabsvarI                      +
\PdresidualBYdfstate    \DdfstateBYabsvarI +
\PdresidualBYdmms       \DdmmsBYabsvarI
\end{align*}


\begin{align*}
\PdresidualBYdfstate    \mathunderline[neutral_2]{\displaystyle\DdfstateBYabsvarI}=
-\DdresidualBYabsvarI 
-\PdresidualBYdmms       \DdmmsBYabsvarI
\end{align*}


\end{frame}

%------------------------------------------------------------------------------
\def\DdresidualBYabsvarI{ \tfrac{\dresidual}{\absvar_i} }
\def\PdresidualBYabsvarI{ \pdfrac{\dresidual}{\absvar_i} }
\def\PdresidualBYdfstate{ \pdfrac{\dresidual}{\dfstate}  }
\def\DdfstateBYabsvarI  { \tfrac{\dfstate}{\absvar_i}   }
\def\PdresidualBYdmms   { \pdfrac{\dresidual}{\dmpos}     }
\def\DdmmsBYabsvarI     { \tfrac{\dmpos}{\absvar_i}      }
\begin{frame}
The interface component is determined by the structure. Having obtained this one, the interior component can be computed by solving an auxiliary, fictitious Dirichlet problem:
\begin{align*}
\tfrac{\mms_\Omega}{\absvar_i}=-\left[\inv{\fic{\stiffmat}_{\Omega\Omega}} \fic{\stiffmat}_{\Omega\Gamma}\right] \tfrac{\mpos_\Gamma}{\absvar_i}
\end{align*}



\begin{align*}
\tfrac{\optcrit_j}{\absvar_i}\bigg\rvert_{\dfstate_0} &=
-\tfrac{\optcrit_j}{\fstate}\bigg\rvert_{\dfstate_0}
\inv{\left[\PdresidualBYdfstate\bigg\rvert_{\fstate_0}\right]}
\left(
  \hle{neutral_1}{\PdresidualBYabsvarI\bigg\rvert_{\fstate_0} } +
  \begin{bmatrix}
    \alpha\pdfrac{\dresidual}{\dmms_\Omega}\bigg\rvert_{\dfstate_0}~
    \pdfrac{\dresidual}{\dmms_\Gamma}\bigg\rvert_{\dfstate_0}
  \end{bmatrix}
  \underbrace{\begin{bmatrix}
    \alpha \inv{\fic{\stiffmat}_{\Omega\Omega}} \fic{\stiffmat}_{\Omega\Gamma} \nonumber\\
    \eye
  \end{bmatrix}
  \underbrace{\tfrac{\dmms_\Gamma}{\absvar_i}}_{\text{SDESIGN}}   }_{\text{ALE mesh motion}}
\right)\\
\alpha&=
\begin{cases}
  1\text{  in ALE framework}\\
  0\text{  in Embedded framework}
\end{cases}
\end{align*}


\end{frame}

%------------------------------------------------------------------------------


\begin{frame}
\frametitle{Derivation of $\pdfrac{\dresidual}{\dfstate}$ }

\begin{align*}
    \frac{\partial \dresidual^{c,i}_{ij}}{\partial \dfstate_k} = 
    {\color{primary_0}{\frac{\partial \dresidual^{c,i}_{ij}}{\partial \dfstateprim_{ij}^\star}}} \,
    {\color{neutral_1}{\frac{\partial \dfstateprim_{ij}^\star}{\partial \dfstateprim_k}}} \,
    \frac{\partial \dfstateprim_k}{\partial \dfstate_k} +
    {\color{primary_0}{\frac{\partial \dresidual^{c,i}_{ij}}{\partial \dfstateprim_{ij}}}} \,
    {\color{neutral_2}{\frac{\partial \dfstateprim_{ij}}{\partial \dfstateprim_k}}} \,
    \frac{\partial \dfstateprim_k}{\partial \dfstate_k} +
    \underbrace{\pdfrac{\dresidual^{c,i}_{ij}}{\dmpos}\pdfrac{\dmpos}{\normal_{ij}}}_{=\vec{0}\text{ for embedded}}
\end{align*}

\begin{center}
	\begin{itemize}
	  \item {\color{primary_0} {Analytical Jacobian of the (Roe's) centering flux} }
	  \item {\color{neutral_1} {Analytical derivative of the solution of the 1D half-Riemann problem}}
	  \item {\color{neutral_2} {Analytical derivative of the MUSCL reconstruction and limitation}}
	\end{itemize}
\end{center}


\only<2>{$\rightarrow$ \textbf{We don't go into any more detail here!}}


\end{frame}

%------------------------------------------------------------------------------

\begin{frame}
\frametitle{Analytic derivatives}
\framesubtitle{$\pdfrac{\dresidual}{\absvar_i}$}

\begin{itemize}
\item{Approach}
	\begin{itemize}
		\item Separate treatment of inviscid and viscous contribution
		\item Re-use information from the intersector, obtained by FIVER, whenever possible
	\end{itemize}
\item Inviscid part

\end{itemize}

\begin{align*}
& \dresidual^{c,i}_{ij} = \fluxesnum^i_{ij}(\dfstateprim_{ij},\dfstateprim^{*}_{ij}(\absvar),\normal_{ij}) \\
& \pdfrac{\dresidual^{c,i}_{ij}}{\absvar}=\pdfrac{\dresidual^{c,i}_{ij}}{\dfstate^{\star}_{ij}}\pdfrac{\dfstate^{\star}_{ij}}{\absvar}
\end{align*}

\end{frame}

%------------------------------------------------------------------------------


\begin{frame}
\frametitle{Analytic derivatives}
\framesubtitle{$\pdfrac{\dresidual}{\absvar_i}$}

\begin{itemize}
\item Diffusive part
\end{itemize}

\begin{align*}
\dresidual_i^v=-\sum_{T_i \in \elementset(i)} \sum_{i=1}^{n_g} 
w_i \prim{\difftensor} \nabla \prim{\dfstate}(\dmpos_i) \nabla \phi_j(\dmpos_i) dx
\end{align*}
\begin{align*}
    \pdfrac{\dresidual^v(\absvars,\dfstateprim^a(\absvars),\dfstateprim^g(\dfstateprim^a(\absvars)),\mpos(\absvars))}{\absvars}=\\
    \underbrace{\pdfrac{\dresidual^v}{\dfstateprim^a}\pdfrac{\dfstateprim^a}{\absvars}}_{\substack{\text{can be re-used from ALE } \\ \text{after the ghost-point population}}}+
    \pdfrac{\dresidual^v}{\dfstateprim^g}\cdot
    \overbrace{\pdfrac{\dfstateprim^g}{\dfstateprim^a}}^{\substack{\text{obtained during the}\\ \text{population process} }} \pdfrac{\dfstateprim^a}{\absvars} \nonumber
    +
    \underbrace{\pdfrac{\dresidual}{\dmpos}\pdfrac{\dmpos}{\absvars}}_{=\vec{0}\text{ for embedded}}
\end{align*}
\end{frame}
