\documentclass[10pt, mathserif]{beamer}


\usepackage{hyperref}
\hypersetup{
    colorlinks=true,
    linkcolor=primary_0,
    citecolor=neutral_1,
    urlcolor=red,
    pdfauthor={Lukas Scheucher},
    pdfcreator={Lukas Scheucher},   % creator of the document
    pdfproducer={Lukas Scheucher}, % producer of the document
    linktoc=all
}

\usetheme{CambridgeUS}
\setbeamertemplate{navigation symbols}{}
\useoutertheme[subsection=false]{miniframes}

\setbeamercolor{block body}{bg=yellow!23}
\setbeamercolor{block title}{bg=purple!100!black!140, fg=white}
\setbeamercolor{footline}{fg=gray!100}
\setbeamercolor{item}{fg=purple!100!black!150,bg=purple!100!black!150}
\setbeamercolor*{title}{fg=purple!100!black!150,bg=lightgray!50}
\setbeamercolor*{author}{fg=blue}
\setbeamercolor*{author in head/foot}{fg=blue,bg=gray!15}
\setbeamercolor*{author in head/foot}{fg=black,bg=gray!15}
\setbeamercolor*{date in head/foot}{fg=black,bg=gray!25}
\setbeamercolor*{frametitle}{fg=purple!100!black!150, bg=gray!15}
\setbeamerfont{footl}{size=\tiny}
\usefonttheme[onlylarge]{structuresmallcapsserif}
\setbeamerfont{frametitle}{size=\normalsize}

\setbeamertemplate{footline}
{
  \leavevmode%
  \hbox{%
  \begin{beamercolorbox}[wd=.4\paperwidth,ht=1.6ex,dp=1ex,center]{author in head/foot}%
    \usebeamerfont*{footl}\insertshortauthor~~(\insertshortinstitute)
  \end{beamercolorbox}%
  \begin{beamercolorbox}[wd=.6\paperwidth,ht=1.6ex,dp=1ex,right]{date in head/foot}%
    \usebeamerfont*{footl}\insertshortdate{}\hspace*{2em}
    \insertframenumber{} / \inserttotalframenumber\hspace*{2ex}
  \end{beamercolorbox}}
  \vskip0pt%
}



\usepackage{amsmath, amstext, amssymb, latexsym, psfrag, soul, cancel, relsize}
\usepackage{graphicx, color, colortbl, epsfig, subfigure, booktabs, animate}
\usepackage{tikz}
\usetikzlibrary{shapes,arrows}
\usepackage[percent]{overpic}


\usetikzlibrary{positioning}
\usetikzlibrary{calc}
\usepackage{pgfplots}
\usepackage{pgfplotstable}
\usepgfplotslibrary{groupplots}

% Define block styles
\tikzstyle{decision} = [diamond, draw, fill=yellow!80, 
                        text width=4.5em, text badly centered, node distance=2cm, inner sep=0pt]
\tikzstyle{block}  = [rectangle, draw, fill=black!20, 
                     text width=8em, text centered, rounded corners, minimum height=1em]
\tikzstyle{block1} = [rectangle, draw, fill=blue!30, 
                     text width=8em, text centered, rounded corners, minimum height=1em]
\tikzstyle{block3} = [rectangle, draw, fill=red!50, 
                     text width=8em, text centered, rounded corners, minimum height=1em]
\tikzstyle{block2} = [rectangle, draw, fill=green!40, 
                     text width=8em, text centered, rounded corners, minimum height=1em]
\tikzstyle{block4} = [rectangle, draw, fill=yellow!80, 
                     text width=8em, text centered, rounded corners, minimum height=1em]
\tikzstyle{blocke} = [rectangle, draw, fill=yellow!25, 
                     text width=16em, text centered, rounded corners, minimum height=2em]
\tikzstyle{line} = [draw, -latex']


%%%%%%%%%%%%%%%%%%%%%%%%%%%%%%%%%%
\newcommand{\Rbb}{\ensuremath{\mathbb{R} }}
\newcommand\mubold{{\ensuremath{\boldsymbol{\mu}}}}
\newcommand{\Jcal}{\ensuremath{\mathcal{J}}}
\newcommand{\func}[3]{\ensuremath{#1 : #2 \rightarrow #3}}
\newcommand{\Dcal}{\ensuremath{\mathcal{D}}}
\newcommand{\Fcal}{\ensuremath{\mathcal{J}}}

\newcommand\cbold{\ensuremath{\mathbf{c}}}
\newcommand{\pder}[2]{\ensuremath{\dfrac{\partial #1}{\partial #2}}} %1st partial derivative
\newcommand{\oder}[2]{\ensuremath{\dfrac{\mathrm{d} #1}{\mathrm{d} #2}}} %1st ordinary derivative

\newcommand\Rbold{\ensuremath{\mathbf{R}}}
\newcommand\ubold{\ensuremath{\mathbf{u}}}
\newcommand\wbold{\ensuremath{\mathbf{w}}}
\newcommand\Fbold{\ensuremath{\mathbf{F}}}
\newcommand\fbold{\ensuremath{\mathbf{f}}}
\newcommand\xbold{\ensuremath{\mathbf{x}}}
\newcommand\nbold{\ensuremath{\mathbf{n}}}
\newcommand\C{\mathcal{C}}
\newcommand\dC{\partial\mathcal{C}}
\newcommand\n{\boldsymbol{n}}
\newcommand\K{\mathcal{K}}

\newcommand{\Et}{E^t}
\newcommand{\IdxM} {\mathbb{I}}
\newcommand{\vel}{\boldsymbol{v}}
\newcommand{\mm}{\boldsymbol{m}}
\newcommand{\dsddt}[1] {\dfrac{\partial{#1}}{\partial{t}}}
\newcommand{\Div}[1] { \boldsymbol{\nabla} \!\dotprod #1 }
\newcommand{\vzero}{\boldsymbol{0}}
\newcommand{\wall}{\mathrm{wall}}
%%%%%%%%%%%%%%%%%%%%%%%%%%%%%%%%%%

\title[]{\large{Analytical and Numerical Approaches for the Computation of Aeroelastic Sensitivities Using the Direct and Adjoint Methods}}
\author[Scheucher]{Lukas Scheucher}
\institute[Stanford]{Stanford University}
\date[SMaster thesis]{October 2016 - June 2017}
\newcommand{\backupbegin}{
   \newcounter{finalframe}
   \setcounter{finalframe}{\value{framenumber}}
}
\newcommand{\backupend}{
   \setcounter{framenumber}{\value{finalframe}}
}


\usepackage{hhline}


\usepackage{color}
\usepackage[table]{xcolor}
\usepackage{soul}
\definecolor{hlcolorblue}{RGB}{126,215,255}
\definecolor{hlcolorgreen}{RGB}{126,255,175}
\definecolor{hlcolororange}{RGB}{246,194,115}
\definecolor{hlcoloryellow}{RGB}{255,243,113}
\definecolor{hlcolorpurple}{RGB}{215,163,232}
\DeclareRobustCommand{\hltblue}[1]{{\sethlcolor{hlcolorblue}\hl{#1}}}
\DeclareRobustCommand{\hltgreen}[1]{{\sethlcolor{hlcolorgreen}\hl{#1}}}
\DeclareRobustCommand{\hltorange}[1]{{\sethlcolor{hlcolororange}\hl{#1}}}
\DeclareRobustCommand{\hltyellow}[1]{{\sethlcolor{hlcoloryellow}\hl{#1}}}
\DeclareRobustCommand{\hltpurple}[1]{{\sethlcolor{hlcolorpurple}\hl{#1}}}
\newcommand{\hleblue}[1]{\colorbox{hlcolorblue}{$\displaystyle{#1}$}}
\newcommand{\hlegreen}[1]{\colorbox{hlcolorgreen}{$\displaystyle{#1}$}}
\newcommand{\hleorange}[1]{\colorbox{hlcolororange}{$\displaystyle{#1}$}}
\newcommand{\hleyellow}[1]{\colorbox{hlcoloryellow}{$\displaystyle{#1}$}}
\newcommand{\hlepurple}[1]{\colorbox{hlcolorpurple}{$\displaystyle{#1}$}}


\def\maathunderline#1#2{\color{#1}\underline{{\color{black}#2}}\color{black}}

\newsavebox\MBox
\newcommand\mathunderline[2][red]{{\sbox\MBox{$#2$}%
  \rlap{\usebox\MBox}\color{#1}\rule[-1.2\dp\MBox]{\wd\MBox}{2.0pt}}}

\DeclareRobustCommand{\hlt}[2]{{\sethlcolor{#1}\hl{#2}}}
\newcommand{\hle}[2]{\colorbox{#1}{$\displaystyle{#2}$}}